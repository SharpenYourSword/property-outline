\section{Private Land Use Controls: The Law of Servitudes}

\subsection{Easements}

\subsubsection{Background}

\begin{enumerate}
    \item Servitudes create interests in land that bind and benefit the 
    parties to an agreement and their successors.\footnote{Casebook p. 763.}
    \item There is much functional overlap between the doctrinal categories of 
    servitudes. But is it necessary?
    \item Modern servitudes are products of the nineteenth century, following 
    the enclosure movement and urbanization.
    \item \textbf{Affirmative easement}: the right to perform an act on 
    servient land.\footnote{Casebook p. 767.}
    \item \textbf{Negative easement}: forbids a landowner from doing something 
    that might harm his neighbor.
    \item \textbf{Easement appurtenant}: grants a right to the owner of the 
    land that benefits from the easement to make use of another's land, i.e., 
    it benefits the land belonging to the easement owner. Usually 
    transferable.
    \item \textbf{Easement in gross}: grants a right to some person, 
    regardless of ownership of land, to make use of another's land, i.e., it 
    benefits the easement owner directly. 
    \item If the parties' intent is unclear, courts resolve the ambiguity in 
    favor of creating an easement appurtenant.
    \item \textbf{Dominant tenement}: the property that the easement benefits.
    \item \textbf{Servient tenement}: the property to which the easement 
    applies.
\end{enumerate}

\subsubsection{Creation of Easements}

\paragraph{\emph{Willard v. First Church of Christ, Scientist}}
~\\\\
Unlike at common law, grantors can create easements in third parties.

\begin{enumerate}
    \item % FIXME 768
\end{enumerate}

\paragraph{Licenses}

\begin{enumerate}
    \item \textbf{License}: permission to allow the licensee to do something 
    that would otherwise be a trespass---e.g., plumbers, dinner guests.
    \item Licenses are revocable. Easements are not.
    \item However, there are two circumstances in which a license is not 
    revocable:
    \begin{enumerate}
        \item A license coupled with an interest, e.g., the right to harvest 
        timber from the grantor's land.
        \item A license can become irrevocable under the rules of estoppel. 
        See \emph{Holbrook} below.
    \end{enumerate}
\end{enumerate}

\paragraph{\emph{Holbrook v. Taylor}}
~\\\\
If the licensee has made substantial improvements in reliance on the license, 
the licensor is estopped from revoking the license.

\begin{enumerate}
    \item % FIXME 774
\end{enumerate}

\paragraph{\emph{Shepard v. Purvine}}

\begin{enumerate}
    \item % FIXME 777
\end{enumerate}

\paragraph{\emph{Henry v. Dalton}}

\begin{enumerate}
    \item % FIXME 777
\end{enumerate}

\paragraph{\emph{Van Sandt v. Royster}}
~\\\\
If the previous use is apparent (though not necessarily visible), the grantee 
can have a quasi-easement.

\begin{enumerate}
    \item % FIXME 779
\end{enumerate}

\paragraph{Implied Easements}

\begin{enumerate}
    \item % FIXME 785
\end{enumerate}

\paragraph{\emph{Othen v. Rosier}}

\begin{enumerate}
    \item % FIXME 786
\end{enumerate}

\paragraph{Easements by Necessity}

\begin{enumerate}
    \item % FIXME 792
\end{enumerate}

\paragraph{Problems on Easements by Necessity}

\begin{enumerate}
    \item % FIXME 793
    \item \footnote{Casebook p. 793 problem 2.}
\end{enumerate}

\subsubsection{Easements by Prescription}

\begin{enumerate}
    \item % FIXME 794
\end{enumerate}

\paragraph{Problems on Easements by Prescription}

\begin{enumerate}
    % FIXME 794
    \item \footnote{Casebook p. 794 note 3.}
    % FIXME 796
    \item \footnote{Casebook p. 796, problems at the end of note 1.}
    % FIXME 797
    \item \footnote{Casebook p. 797, problem 4 (first paragraph only).}
\end{enumerate}

\subsubsection{Negative Easements}

% FIXME 842-45

\paragraph{\emph{Fontainebleu Hotel Corp. v. Forty-Five Twenty-Five Inc.}}

\begin{enumerate}
    \item % FIXME supp 215
\end{enumerate}


\subsection{Covenants Running with the Land}

% TODO 847 ff.
