\section{Servitudes}

\subsection{Easements}

\subsubsection{Background}

\begin{enumerate}
    \item Servitudes create interests in land that bind and benefit the 
    parties to an agreement and their successors.\footnote{Casebook p. 763.}
    \item There is much functional overlap between the doctrinal categories of 
    servitudes. Is it necessary?
    \item Modern servitudes are products of the nineteenth century, following 
    the enclosure movement and urbanization.
    \item \textbf{Affirmative easement}: the right to perform an act on 
    servient land.\footnote{Casebook p. 767.}
    \item \textbf{Negative easement}: forbids a landowner from doing something 
    that might harm his neighbor.
    \item \textbf{Easement appurtenant}: grants a right to the owner of the 
    land that benefits from the easement to make use of another's land, i.e., 
    it benefits the land belonging to the easement owner. Usually 
    transferable.
    \item \textbf{Easement in gross}: grants a right to some person, 
    regardless of ownership of land, to make use of another's land, i.e., it 
    benefits the easement owner directly. 
    \item If the parties' intent is unclear, courts resolve the ambiguity in 
    favor of creating an easement appurtenant.
    \item \textbf{Dominant tenement}: the property that the easement benefits.
    \item \textbf{Servient tenement}: the property to which the easement 
    applies.
\end{enumerate}

\subsubsection{Express Easements}

\begin{enumerate}
    \item % TODO
\end{enumerate}

\paragraph{Express Easement for a Third Party: \emph{Willard v. First Church 
of Christ, Scientist}}
~\\\\
Unlike at common law, grantors can create easements in third parties.

\begin{enumerate}
    \item McGuigan owned lots 19 and 20. She allowed churchgoers to park in 
    lot 20. She sold lot 19 to Peterson. Peterson wanted to resell it, so he 
    listed it with Willard, a realtor. Willard wanted to buy lots 19 and 20, 
    so Peterson conveyed both to him by deed in fee simple.
    \item But, Peterson did not own lot 20 when he agreed to convey it to 
    Willard, so he offered to buy it from McGuigan. She was willing to sell 
    as long as the church could continue to use the lot for parking. The deed 
    from McGuigan to Peterson for lot 20 included an easement for church 
    purposes.
    \item Willard then received the deed for lot 20 from Peterson, which did 
    not contain the language creating the easement, though Peterson told 
    Willard that the church would want to use the lot for parking.
    \item Willard learned of the easement several months later. He sued to 
    quiet title against the church.
    \item The trial court held that McGuigan and Peterson intended to convey 
    an easement, but that it was invalid because of the ``common law rule that 
    one cannot `reserve' an interest in property to a stranger to the 
    title.''\footnote{Casebook p. 769.} The rule was based on a feudal desire 
    to limit conveyances by deed in favor of conveyances by livery of seisin.
    \item Many courts have found ways of circumventing this rule, and at least 
    two states have discarded it.\footnote{Casebook p. 771.}
    \item Willard argued that the old rule should nonetheless apply because 
    grantees and title insurers relied on it, but the court found no evidence 
    to support his claim.
    \item ``~.~.~.~in the instant case the balance falls in favor of the 
    grantor's intent, and the old common law rule may not be applied to defeat 
    her intent.''\footnote{Casebook p. 772.}
\end{enumerate}

\subsubsection{Licenses}

\begin{enumerate}
    \item \textbf{License}: permission to allow the licensee to do something 
    that would otherwise be a trespass---e.g., plumbers, dinner guests.
    \item Licenses are revocable. Easements are not.
    \item However, there are two circumstances in which a license is not 
    revocable:
    \begin{enumerate}
        \item A license coupled with an interest, e.g., the right to harvest 
        timber from the grantor's land.
        \item A license can become irrevocable under the rules of estoppel. 
        See \emph{Holbrook} below.
    \end{enumerate}
\end{enumerate}

\subsubsection{Irrevocable Licenses by Estoppel: \emph{Holbrook v. Taylor}}
~\\\\
If the licensee has made substantial improvements in reliance on the license, 
the licensor is estopped from revoking the license.

\begin{enumerate}
    \item Facts:
    \begin{enumerate}
        \item 1942: The Holbrooks bought the property in question.
        \item 1944: The Holbrooks granted permission for a road to be cut to 
        haul coal from a mine.
        \item 1949: The mine closed and the road was no longer used.
        \item 1957: The Holbrooks built a house and leased it. The Holbrooks 
        and the tenants used the road.
        \item 1961: The tenant house burned down and was not rebuilt.
        \item 1964: The Taylors bought property next to the Holbrooks' and 
        built a house. Until 1965, the Holbrooks gave permission for the 
        Taylors to use the road.
    \end{enumerate}
    \item Precedent supported the view that ``a right to the use of a roadway 
    over the lands of another may be established by 
    estoppel.''\footnote{Casebook p. 775.}
    \item After 1965, it was disputed whether the Taylors used the road by 
    permission or claim of right. The Taylors argued that it was used by them 
    and others without the Holbrooks' permission.
    \item In 1970, the Holbrooks apparently wanted to require the Taylors to 
    buy the land containing the road for \$500.
    \item The court held that several factors---the right to get to their 
    home, the use of the road for construction, and improvements---granted an 
    irrevocable license by estoppel to the Taylors to use the 
    road.\footnote{Casebook p. 777.}
\end{enumerate}

\paragraph{\emph{Shepard v. Purvine}}

\begin{enumerate}
    \item % FIXME 777
\end{enumerate}

\paragraph{\emph{Henry v. Dalton}}

\begin{enumerate}
    \item % FIXME 777
\end{enumerate}

\subsubsection{Easements Implied by Prior Existing Use}

\begin{enumerate}
    \item \textbf{Easement implied from a prior existing use}, i.e., a 
    quasi-easement: apparent and continuous use. The easement protects the 
    probable intentions of the grantor and grantee to create an 
    easement.\footnote{Casebook p. 785.} \emph{Van Sandt v. Royster.}
    \item % TODO UP 532
\end{enumerate}

\paragraph{Easement Implied by Prior Existing Use: \emph{Van Sandt v. Royster}}
~\\\\
An easement by reservation is valid when the prior use was reasonably 
apparent.

\begin{enumerate}
    \item Bailey owned three adjacent lots. She built an underground sewer to 
    connect the lots to a public sewer on the street. Van Sandt, the 
    plaintiff, acquired the lot closest to the road (lot 19). Royster, the 
    defendant, acquired the adjacent lot (lot 20).
    \item Van Sandt discovered that his basement was flooded with sewage. He 
    sued to prevent the defendants from draining their sewage across his land.
    \item The trial court found an appurtenant easement, holding for 
    defendants.
    \item On appeal, Van Sandt argued that there was never an easement 
    created in his land, and if there was, he took the premises free from any 
    easement because there was no actual or constructive 
    notice.\footnote{Casebook p. 780.}
    \item Royster argued that Bailey (the original owner of the lots) created 
    an easement by implied reservation when she severed the lots. Royster's 
    backup theory was that he had a valid easement by prescription.
    \item \textbf{Easement by grant}: an owner grants an easement to another. 
    It favors the grantee.\footnote{See \emph{Understanding Property} p. 530.}
    \item \textbf{Easement by reservation}: an owner conveys part of his 
    property to a grantee but retains an easement by reservation. It favors 
    the grantor.
    \item The court considered various approaches to the differences between 
    easements by implied grant and implied reservation.
    \begin{enumerate}
        \item Some courts: unless implied by necessity, a grantor should not 
        be able to reserve an easement by implication.
        \item Other courts: there should \emph{never} be an implied easement 
        by reservation.
        \item Still others: there is no difference between an implied 
        reservation and a grant.
        \item This court: necessity is a factor in determining whether the 
        easement was implied by reservation, but it is not determinative. 
        Here, the implied reservation was permissible.
    \end{enumerate}
    \item The court here held that an owner cannot have an easement in his 
    own land, but he can have a ``quasi-easement'' if one part of the land is 
    necessary for the benefit of another. It held that an implied reservation 
    exists when the grantee knew or reasonably should have known of the prior 
    use. It doesn't matter whether the use was visible to the naked eye as 
    long as the use was reasonably apparent (e.g., underground sewers).
    \item Because the court held that there was a valid easement by 
    reservation, it did not reach the question of whether there was a 
    prescriptive easement.
    \item Affirmed.
\end{enumerate}

\subsubsection{Easements by Necessity}

\begin{enumerate}
    \item The easement is necessary for the enjoyment of the land, and the 
    necessity arose upon severance of the dominant and servient 
    parcels.\footnote{Casebook p. 785.} \emph{Othen v.  Rosier.}
    \item Necessity might be a factor in establishing whether an easement is 
    implied from an existing use, but the two are distinct.
    \begin{enumerate}
        \item Old rule: ``strict necessity is required for implied easements 
        in favor of the grantor.''\footnote{Casebook p. 785.} At least one 
        jurisdiction has required prior use, not recognizing necessity as 
        sufficient.
        \item Most jurisdictions: only reasonable necessity is required, 
        regardless of whether the easement is implied in favor of the grantor 
        or grantee.
    \end{enumerate}
    \item An easement by necessity is implied either because of \textbf{public 
    policy} (access to landlocked land) or \textbf{the parties' intent} 
    (presumably the grantor intended the land to be 
    accessible).\footnote{Casebook pp. 792--93.}
\end{enumerate}

\paragraph{Easement by Necessity: \emph{Othen v. Rosier}}

\begin{enumerate}
    \item Othen owned land that could only access the public highway by 
    crossing Rosier's land. The Rosiers maintained a road which they used for 
    farming needs and which Othen used to access his land.
    \item The Rosiers constructed a levee that flooded the road, rendering it 
    unusable. Othen sued for an injunction to recognize an easement.
    \item The trial court held that Othen had an easement by necessity.
    \item The appellate court reversed, finding that there was no easement.
    \item Before the Supreme Court of Texas, Othen claimed an easement both of 
    necessity and by prescription. (An easement by prescription arises from 
    adverse use for a certain period---like adverse possession, but creating 
    an easement rather than a possessory right; see \emph{Van Valkenburgh}).
    \begin{enumerate}
        \item \emph{Easement of necessity}: no. An easement of necessity 
        arises when an owner severs two parcels, making it necessary to cross 
        one parcel to access the other. It \emph{only} arises when the 
        sevrance causes the land to be landlocked. In this case, Hill owned 
        Othen's and the Rosiers' land as well as the surrounding land. Upon 
        severance, there might have been several other ways to access Othen's 
        lot. Othen failed to show that use of the road on the Rosiers' lot was 
        \emph{necessary}, rather than merely 
        \emph{convenient}.\footnote{Casebook p. 789.}
        \item \emph{Easement by prescription}: no. An easement by prescription 
        arises from adverse use. Othen's use of the Rosiers' road was not 
        adverse because it was permissive and therefore constituted a 
        license.\footnote{Casebook p. 791.}
    \end{enumerate}
    \item Othen did not have an easement because he failed to prove that 
    access to the Rosiers' road was necessary. Affirmed.
    \item Othen could have successfully argued (1) that there was an implied 
    easement by reservation on the basis of prior use (but Texas does not 
    allow easements by implied reservation without necessity), or (2) that he 
    had a license to use the road that had become irrevocable through 
    estoppel.
\end{enumerate}

\paragraph{Problems on Easements by Necessity}

\begin{enumerate}
    \item A owns five tracts of land. Lots 1--4 surround lot 5. A purchased 
    each lot from O in separate transactions. A dies intestate, leaving lot 1 
    to B, 2 to C, 3 to D, 4 to E, and 5 to F (so F has the landlocked lot). 
    There is no mention of an easement for F on lot 5. F sues the owners of 
    lots 1--4, claiming an easement by necessity. What 
    result?\footnote{Casebook p. 793 problem 2.}
    \begin{enumerate}
        \item % TODO
    \end{enumerate}
    \item % TODO \footnote{Casebook p. 794 note 2.}
\end{enumerate}

\subsubsection{Easements by Prescription}

\begin{enumerate}
    \item Similar to adverse possession, but leading to use, not possession.
    \item \emph{Fiction of the lost grant}: ``[i]f use was shown to have 
    existed for 20 years, it was presumed that a grant of easement had been 
    made and that the grant had been lost.'' In other words, it granted an 
    easement after 20 years of adverse use on the fiction that the owner had 
    made a grant.\footnote{Casebook p. 795.} The lost grant theory confuses 
    acquiescence and permission.
\end{enumerate}

\paragraph{Problems on Easements by Prescription}

\begin{enumerate}
    \item Say A built a road across O's land in 1982. In 1994, O writes a 
    letter telling A to stop using the road. A ignores O's letter and 
    continues to use the road until 2002. The prescriptive period is 20 
    years.\footnote{Casebook p. 796.}
    \begin{enumerate}
        \item Does A have a prescriptive easement under the lost grant theory?
        \begin{enumerate}
            \item % TODO ~\\\\\\\\\\\\
        \end{enumerate}
        \item Does A have a prescriptive easement under the adverse use 
        theory?
        \begin{enumerate}
            \item ``In a jurisdiction not following the fiction of the lost 
            grant, to prevent a prescriptive easement from being acquired, the 
            owner must effectively interrupt or stop the adverse 
            use.''\footnote{Casebook p. 796.} O's letter is insufficient 
            to interrupt A's adverse use, so A acquires a prescriptive 
            easement.
        \end{enumerate}
        \item What if instead of writing a letter, O put up a fence blocking 
        the road?
        \begin{enumerate}
            \item O's action would have interrupted A's use. Prescriptive 
            easements, like adverse possession, require continuous use. A 
            would not have a prescriptive easement. 
        \end{enumerate}
    \end{enumerate}
    \item A owns a house next to a golf course. Every day, several golf balls 
    end up on his lawn, and golfers come to retrieve them.\footnote{Casebook 
    p. 797.}
    \begin{enumerate}
        \item If this continues, will the golf course acquire a prescriptive 
        easement?
        \begin{enumerate}
            \item Yes, because course's licensees will have met the 
            requireements.
        \end{enumerate}
        \item What could A do to prevent a prescriptive easement?
        \begin{enumerate}
            \item He could post a sign giving golfers permission to retrieve 
            their balls, which would negate the adverse use requirement. 
      \end{enumerate}
    \end{enumerate}
\end{enumerate}

\subsubsection{Negative Easements}

\begin{enumerate}
    \item ``A negative easement is the right of the dominant owner to stop the 
    servient owner from doing something on the servient 
    land.''\footnote{Casebook p. 842.}
    \item English courts recognized four types of negative easements, 
    preventing a neighbor from blocking windows, blocking air, removing 
    support from a building, and interfering with water flow.
    \item American courts accepted the English restrictions on the creation of 
    new negative easements, though there are persuasive policy reasons in 
    favor of the opposite approach.\footnote{Casebook p. 843--45.}
    \item American courts have recognized a few new negative easements, 
    including the right to an unobstructed view and the right to sunlight (for 
    solar energy).
\end{enumerate}

\paragraph{\emph{Fontainebleu Hotel Corp. v. Forty-Five Twenty-Five Inc.}}

\begin{enumerate}
    \item % FIXME supp 215
\end{enumerate}

\subsubsection{Scope}

\begin{enumerate}
    \item % TODO UP 545
\end{enumerate}

\subsubsection{Transfer}

\begin{enumerate}
    \item % TODO UP 548
\end{enumerate}

\subsubsection{Termination}

\begin{enumerate}
    \item There are several ways an easement can end:\footnote{Casebook p. 
    841--42.}
    \begin{enumerate}
        \item The easement owner can agree to \textbf{release} it.
        \item The easement can \textbf{expire} at the end of a stated period 
        or upon the occurrence of a state event (as with a defeasible 
        easement).
        \item It can \textbf{merge} if the easement owner acquires the 
        servient estate.
        \item It can end when the \textbf{necessity ends}.
        \item \textbf{Estoppel}.
        \item \textbf{Abandonment}.
        \item It can end by \textbf{condemnation} if the government exercises 
        its eminent domain power to take a fee interest in the servient 
        estate.
        \item It can end by \textbf{prescription} if the servient owner 
        wrongfully and physically prevents the easement from being used for 
        the prescriptive period.
    \end{enumerate}
    \item Traditionally, courts could not modify easements according to 
    changed conditions (though they could change real covenants and equitable 
    servitudes), though this is changing (e.g., with the Restatement Third).
\end{enumerate}

\subsection{Real Covenants}

\subsubsection{Basic Fact Pattern}

\begin{enumerate}
    \item A owns lots 1 and 2. A sells lot 2 to B with a deed in which B 
    promises to use the land only for residential purposes; or, A and B reach 
    a contractual agreement.
    \item B conveys his lot to C; or, A conveys his lot to D; or both.
\end{enumerate}

\subsubsection{Definition}

\begin{enumerate}
    \item A real covenant is a land use promise that benefits and burdens the 
    original parties and their successors. It is enforceable in an action for 
    damages.
    \item It runs with an estate in land, not the land itself.
    \item If a promise meets the requirement, it can be enforced either as a 
    real covenant or an equitable servitude.
    \item It can be affirmative (a promise to act) or negative (a promise to 
    refrain).
    \item It shares many features with the negative easement, including the 
    promise to refrain and damages as the remedy. But courts recognize only a 
    few types of negative easements---promises to not block windows, air, or 
    water, and to not remove support for buildings---while real covenants are 
    not limited in scope.
\end{enumerate}

\subsubsection{History}

\begin{enumerate}
    \item At early common law, A could enforce the contract against B as a 
    personal covenant. But the benefits and burdens of the personal covenant 
    did not apply to the successors of A or B because contract rights and 
    duties were not assignable.
    \item Courts developed real covenants and equitable servitudes to extend 
    the benefits and burdens of land use covenants to successors.
    \begin{enumerate}
        \item Real covenants developed in courts at \emph{law} for actions for 
        \emph{damages}.
        \item Equitable servitudes developed in courts in \emph{equity} for 
        actions for \emph{injunctions}.
    \end{enumerate}
    \item English law courts were hostile to restraints on land, so the law of 
    real covenants developed into an ``unspeakable 
    quagmire.''\footnote{\emph{Understanding Property} 557.} Equity courts 
    were willing to tolerate land use restraints in the interest of fairness, 
    so the law of equitable servitudes was relatively straightforward.
    \item American courts have blurred the distinctions between real covenants 
    and equitable servitudes. The Restatement (Third) combines both into the 
    ``servitude.''
\end{enumerate}

\paragraph{Egan, ``The Serene Fortress''}

\begin{enumerate}
    \item Americans increasingly live in gated communities, governed by 
    private covenants.\footnote{Casebook p. 922.}
\end{enumerate}

\subsubsection{Creation and Enforcement}

\paragraph{Scenario 1: Original Promisee vs. Promisor's Successor}

\begin{enumerate}
    \item The original promisee (A) seeks to enforce a promise against the 
    promisor's  successor (C). Does the \emph{burden} run?
    \item It must be in \textbf{writing}. A written declaration from a 
    subdivider is enough.
    \item It must show \textbf{intent to bind successors}. Intent can be 
    explicit (``to his heirs and assigns'') or inferred (``the land shall be 
    used only for residential purposes'').
    \item It must \textbf{``touch and concern''} the land.
    \item The original covenanting parties (A and B) must be in 
    \textbf{horizontal privity}. American courts follow three standards:
    \begin{enumerate}
        \item \textbf{Mutual interests}: A and B are (1) landlord-tenant or 
        (2) owners of the dominant and servient tenements of an easement. A's 
        conveyance to B in fee simple absolute does \emph{not} meet this 
        requirement.
        \item \textbf{Successive interests}: horizontal privity arises when 
        the covenant is created from a transaction conveying land from A to 
        B---e.g., A conveys one of his lots to B with a deed containing a 
        covenant. This is the US majority rule.
        \item \textbf{No horizontal privity required}.
    \end{enumerate}
    \item \textbf{Vertical privity} is required for the \emph{burden} to bind 
    the promisor's successors (here, D). It arises when the promisor conveys 
    his \emph{entire estate}, but not less than that (e.g., a life estate or 
    term of years).
    \item Successors must have \textbf{notice} (actual, record, inquiry, or 
    imputed).
\end{enumerate}

\paragraph{Scenario 2: Promisee's Successor vs. Original Promisor}

\begin{enumerate}
    \item The promisee's successor (C) seeks to enforce a promise against the 
    original promisor (B). Does the \emph{benefit} run?
    \item The original covenant must be \textbf{in writing}.
    \item The original parties must \textbf{intend to benefit successors}.
    \item The covenant must \textbf{``touch and concern''} the land.
    \item There must be \textbf{vertical privity}.
    \begin{enumerate}
        \item There is vertical privity even if the promisee's successor 
        received less than the promisee's entire interest.
    \end{enumerate}
    \item \textbf{Horizontal privity} and \textbf{notice} are \textbf{not 
    required.}
    \item Example: A and B agree in writing to keep 90\% of their front yards 
    as grass. A sells his lot to C. B paves his yard. C can win damages 
    against B.
\end{enumerate}

\paragraph{Scenario 3: Promisee's Successor vs. Promisor's Successor}

\begin{enumerate}
    \item The promisee's successor (C) seeks to enforce a promise against the 
    promisor's successor (D). Do \emph{both} the benefit and burden run?
    \item The analysis is the same as above.
\end{enumerate}

\subsubsection{Scope}

\begin{enumerate}
    \item There are no limits as long as the agreement touches and concerns 
    the land.
    \item Real covenants can only be enforced in actions for damages.
\end{enumerate}

\subsubsection{Termination}

\begin{enumerate}
    \item \textbf{Release}: the parties can agree on an expiration date or 
    agree to release their rights.
    \item \textbf{Eminent domain} and other government actions.
    \item \textbf{Merger} (e.g., the owner of the benefited land gains 
    possession of the burdened land).
    \item \textbf{Abandonment}: the person entitled to the benefit 
    demonstrates intent to abandon the covenant.
    \item \textbf{Changed conditions}: the covenant becomes unenforceable when 
    the benefits cannot be realized.
    \item Anti-discrimination statutes.
\end{enumerate}

\subsubsection{Remedies}

\begin{enumerate}
    \item Only damages.
\end{enumerate}

\subsubsection{Problem on Horizontal Privity}

\begin{enumerate}
    \item A, B, C, and D, neighboring landowners, decide that they will 
    mutually restrict their lots to single-family residential use only. They 
    sign an agreement wherein each promises on behalf of himself, his heirs 
    and assigns that his lot will be used for single-family residential 
    purposes only. This agreement is recorded in the county courthouse under 
    the name of each signer. A sells his lot to E. E builds an apartment house 
    on his lot.\footnote{Assignment sheet \#14, 4/15/2013.}
    \begin{enumerate}
        \item B, C, and D sue for damages. What result?
        \begin{enumerate}
            \item If the jurisdiction requires privity of estate, B, C, and D 
            will lose. Privity of estate can be created only through a 
            conveyance. A contract isn't enough.\footnote{See problem from pp. 
            851--52, above.}
        \end{enumerate}
        \item Suppose that C rather than E had built the apartment house. Is E 
        entitled to damages against C?
        \begin{enumerate}
            \item ~\\\\\\\\ % TODO 
        \end{enumerate}
        \item Suppose that A, B, C, and D, in order to preserve their views 
        over hillside lots, had agreed that no building taller than 20 feet 
        would be erected on any lot. A sells his lot to E, who erects a 
        30-foot building. C argues that the agreement creates a negative 
        easement. What result in a suit by C against E for damages?    
        \begin{enumerate}
            \item ~\\\\\\\\ % TODO
        \end{enumerate}
    \end{enumerate}
\end{enumerate}

\subsubsection{Policy}

\begin{enumerate}
    \item English law courts restricted real covenants out of fear that 
    restraints on land would limit productive use. Today, courts recognize 
    that land use restraints can enhance productive use---e.g., limiting 
    neighborhoods to residential use keeps property value high. Still, land 
    use restrictions can limit productive use---e.g., 99 lots have been 
    converted to office buildings, while 1 adjacent lot can only be used for 
    agriculture.
    \item Real covenants respect parties' liberty and autonomy.
    \item The horizontal privity requirement is obsolete, and anyway, parties 
    can easily evade it through a straw.
    \item The ``touch and concern'' requirement frustrates the intent of the 
    parties. On the other hand, it promotes efficient use by preventing 
    burdens that impair marketability, and it protects owners' expectations by 
    ensuring a relationship between the benefit and the burden.
    \item Most agree that the vertical privity requirement should be relaxed. 
    Should a successor be able to avoid the burden because he has a 99-year 
    lease, rather than an estate in FSA? On the other hand, it might make 
    sense not to burden a tenant with a one-month term of years.
\end{enumerate}

\subsection{Equitable Servitudes}

\subsubsection{Basic Fact Pattern}

\begin{enumerate}
    \item Same as with real covenants. When do the benefits and burdens of a 
    promise run to successors?
    \item Enforced in actions for injunctions, not damages.
\end{enumerate}

\subsubsection{Definition}

\begin{enumerate}
    \item An equitable servitude is a land use agreement that benefits and 
    burdens the original parties and their successors. It is enforceable in 
    actions in equity (for injunctions).
    \item Equitable servitude vs. real covenant:
    \begin{enumerate}
        \item The standard for enforcing an equitable servitude is easier to 
        meet.
        \item A broader range of defenses applies to equitable servitudes (see 
        defenses below).
        \item The remedy is an injunction, not damages.
    \end{enumerate}
    \item Equitable servitudes are interests in land. Unlike real covenants, 
    they attach to the land itself, not to the estate in land (an equitable 
    servitude ``sinks its tentacles into the soil'').\footnote{Casebook p. 
    857.}
\end{enumerate}

\subsubsection{History}

\begin{enumerate}
    \item See real covenants.
\end{enumerate}

\paragraph{Origin: \emph{Tulk v. Moxhay}}

\begin{enumerate}
    \item Tulk owned the vacant piece of ground in Leicester Square and 
    several surrounding houses. In 1808, he sold the garden to Elms with a 
    deed containing a covenant that Elms and his heirs and assigns would 
    maintain the land as a garden and not build any buildings.
    \item After several conveyances, Moxhay gained possession of the land. His 
    deed said nothing about the covenant, but he admitted that he had 
    purchased the land with notice of the original covenant.
    \item Moxhay wanted to build on the land. Tulk sued for an injunction, 
    which the Master of the Rolls granted.
    \item Moxhay argued that the burden of Elms's covenant did not pass to 
    subsequent owners.
    \item The court here held that if the original covenant is to have any 
    value, it must be enforceable against subsequent purchasers.
\end{enumerate}

\subsubsection{Creation and Enforcements}

\paragraph{Scenario 1: Original Promisee vs. Promisor's Successor}

\begin{enumerate}
    \item The original promisee (A) seeks to enforce a promise against the 
    promisor's  successor (C). Does the \emph{burden} run?
    \item The promise must be \textbf{in writing or implied from a common 
    plan}.
    \item The original parties must \textbf{intend to bind successors}.
    \item The promise must \textbf{``touch and concern''} the land.
    \item The successor must have \textbf{notice} (actual, record, imputed, 
    inquiry).
    \begin{enumerate}
        \item In England, promises are unenforceable against successors who 
        lack notice. \emph{Tulk v. Moxhay}.
        \item Inquiry notice can be sufficient. \emph{Sanborn v. McClean}.
    \end{enumerate}
    \item \textbf{Horizontal and vertical privity are not required}.
\end{enumerate}

\paragraph{Scenario 2: Promisee's Successor vs. Original Promisor}

\begin{enumerate}
    \item The promisee's successor (C) seeks to enforce a promise against the 
    original promisor (B). Does the \emph{benefit} run?
    \item The promise must be \textbf{in writing or implied from a common 
    plan}.
    \item The original parties must \textbf{intend to bind successors}.
    \item The promise must \textbf{``touch and concern''} the land.
    \item There are \textbf{no notice or privity requirements}.
\end{enumerate}

\paragraph{Scenario 3: Promisee's Successor vs. Promisor's Successor}

\begin{enumerate}
    \item The promisee's successor (C) seeks to enforce a promise against the 
    promisor's successor (D). Do \emph{both} the benefit and burden run?
    \item The analysis is the same as above.
\end{enumerate}

\subsubsection{Scope}

\begin{enumerate}
    \item The promise must touch and concern the land. It must be enforced in 
    an action for an injunction (not damages).
\end{enumerate}

\subsubsection{Subdivisions}

\begin{enumerate}
    \item A subdivision developer wants to bind all lots to a covenant 
    restricting use to residential purposes. He includes the covenant in each 
    deed. But what if he forgets to include the covenant in some of the deeds? 
    Can it still be enforced as an implied covenant?
    \begin{enumerate}
        \item A common scheme creates an implied reciprocal covenant that 
        binds successors. \emph{Sanborn v. McClean}.
        \item Some states reject the common scheme approach. In those states, 
        successors can still enforce the covenant as third party 
        beneficiaries. \emph{Snow v. Van Dam}.
    \end{enumerate}
\end{enumerate}

\paragraph{Implied Burden, the Implied Reciprocal Negative Equitable 
Servitude, and the ``Common Plan'': \emph{Sanborn v. McClean}}
~\\\\
The first lot sold in the subdivision contained a covenant in the deed. But 
the deeds for the remaining lots do no mention the covenant. Can residents 
enforce the implied covenant against an owner who had no notice?

The court here said yes. This is the majority rule. ``If the owner of two or 
more lots, so situated as to bear the relation, sells one with restrictions of 
benefit to the land retained, the servitude becomes mutual, and, during the 
period of restraint, the owner of the lot or lots retained can do nothing 
forbidden to the owner of the lot sold.''\footnote{Casebook p.  860.}

\begin{enumerate}
    \item The McLeans wanted to build a gas station on the back of their lot. 
    Their neighbors, the plaintiffs, alleged that the gas station would 
    violate the restriction on the lots on the street that they could only be 
    used for residential purposes, as evidenced by restrictions on 53 of the 
    91 lots. They argue that defendants' lot was subject to a reciprocal 
    negative easement.\footnote{Casebook pp. 859--60.}
    \item The McLeans argued that no restrictions appeared in their chain of 
    title and that they purchased without notice of a reciprocal negative 
    easement.
    \item \textbf{Reciprocal negative easement} (but really, an implied 
    reciprocal servitude): ``If the owner of two or more lots, so situated as 
    to bear the relation, sells one with restrictions of benefit to the land 
    retained, the servitude becomes mutual, and, during the period of 
    restraint, the owner of the lot or lots retained can do nothing forbidden 
    to the owner of the lot sold.''\footnote{Casebook p.  860.} It must start 
    from a common owner.
    \item Although his title was silent on restrictions, Mr. McLean was put on 
    inquiry notice, i.e., he could not avoid noticing the uniformity of the 
    residences on the street, and ``the least inquiry'' would have alerted him 
    to the existence of a reciprocal negative easement.
    \item Held: The McLeans were not allowed to build their gas station. Any 
    construction so far had to be torn down or repurposed for residential use.
\end{enumerate}

\paragraph{Implied Benefit and Third Party Beneficiaries: \emph{Snow v. Van 
Dam}}
~\\\\
A subdivider, S, sells a lot to A in 2012, to B in 2013, and to C in 2014. Each 
deed contains a covenant to benefit ``S and his successors.'' If C breaches, 
can A win an injunction?

In states that follow the ``common plan'' approach from \emph{Sanborn}, the 
implied covenant binds all owners in the subdivision, so A wins.

But some states (e.g., Massachusetts and California) reject the common plan 
approach. The \emph{Van Dam} solution holds that A can win as a third party 
beneficiary of the covenant between S and C.

\begin{enumerate}
    \item Facts:
    \begin{enumerate}
        \item September 5, 1906: Luce acquired title. Soon after, he 
        transferred it to Shackleton.
        \item The northern part of the land (including the part later owned by 
        Van Dam) was marshy and deemed ``unsuitable for building and 
        worthless.''\footnote{Reader p. 236.}
        \item 1907: the lower tract was subdivided into around a hundred lots 
        for summer residences, all owned by the plaintiffs.
        \item July 18, 1907--January 23, 1923: almost all of the lower lots 
        were sold. Almost without exceptions, the deeds contained restrictions 
        that ``only one dwelling house shall be erected or maintained thereon 
        at any given time which building shall cost not less than \$2500 and 
        no outbuilding containing a privy shall be erected or maintained on 
        said parcel without the consent in writing of the grantor or their 
        heirs.''\footnote{Reader p. 236.}
        \item January 23, 1923: Shackelford conveyed the three northern marshy 
        lots (C, D, and E) to Robert C. Clark, subject to similar 
        restrictions.
        \item February 18, 1923: Clark conveyed lot D to Van Dam. 
        \item June 15, 1923: Shackelford conveyed the remaining southern lots 
        to J. Richard Clark, subject to similar restrictions as the other 
        southern lots.
        \item 
    \end{enumerate}
    \item Van Dam built a building to sell ice cream, etc. The plaintiffs sued 
    for an injunction.
    \item The trial court enjoined Van Dam from building non-residential 
    buildings on its land.\footnote{Reader p. 235.}
    \item To be attached to land, a restriction must be intended to benefit a 
    dominant estate. If there's no dominant estate, it's a personal contract 
    that does not attach to the land.
    \item The restriction in the conveyance of lot D did not specify whether 
    it benefited a dominant estate. In the absence of express statements, 
    courts can infer an intention that a restriction benefit a dominant 
    estate. That inference can arise from ``a scheme or plan for restricting 
    the lots in a tract undergoing development to obtain substantial 
    uniformity in building and use. The existence of such a building scheme 
    has often been relied on to show an intention that the restrictions 
    imposed upon the several lots shall be appurtenant to every other lot in 
    the tract included in the scheme.''\footnote{Reader p. 237.}
    \item A scheme existed in this case. Because of the scheme, all plaintiffs 
    are entitled to relief.
    \item The scheme included the northern land because every entrant to the 
    lower land must pass through it. Also, all of the plans for the housing 
    development included the northern land.
\end{enumerate}

\subsubsection{Termination}

\begin{enumerate}
    \item \textbf{Anti-discrimination protections}: no racial covenants. Some 
    jurisdictions refuse to enforce single-family home covenants.
    \item \textbf{Changed conditions}: the restriction is unenforceable if it 
    would no longer give substantial benefit to the dominant estate.
    \item Others: release, abandonment, merger, eminent domain. See real 
    covenants above.
    \item Still others: acquiescence, estoppel, laches, relative hardship, 
    unclean hands.\footnote{See \emph{Understanding Property} pp. 590--91.}
\end{enumerate}

\subsubsection{Remedies}

\begin{enumerate}
    \item Only injunctions.
\end{enumerate}

\subsubsection{Policy}

\begin{enumerate}
    \item The requirements for equitable servitudes are less controversial 
    than those for real covenants.
    \item The Restatement (Third) combines both into the 
    ``servitude.''\footnote{See \emph{Understanding Property} pp. 592--94.}
\end{enumerate}

\subsubsection{Problems on Equitable Servitudes}

\begin{enumerate}
    \item Barnes owns a lot immediately to the south of Zamiarski's lot. 
    Barnes sells his lot to Lewek, subject to a restriction that he will not 
    build anything within ten feet of the northern property line (bordering 
    Zamiarski's lot). Later, Kozial takes possession of Lewek's lot. He wants 
    to build within ten feet of the line. Zamiarski sues for an injunction. 
    What result?\footnote{Reader p. 239 note 2.} % TODO verify
    \begin{enumerate}
        \item It depends on whether the court requires privity of estate to 
        enforce the covenant. Zamiarski and Kozial are not in horizontal or 
        vertical privity. If the court does not require privity, Zamiarski 
        wins.
    \end{enumerate}
\end{enumerate}
