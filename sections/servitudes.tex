\section{Private Land Use Controls: The Law of Servitudes}

\subsection{Easements}

\subsubsection{Background}

\begin{enumerate}
    \item Servitudes create interests in land that bind and benefit the 
    parties to an agreement and their successors.\footnote{Casebook p. 763.}
    \item There is much functional overlap between the doctrinal categories of 
    servitudes. But is it necessary?
    \item Modern servitudes are products of the nineteenth century, following 
    the enclosure movement and urbanization.
    \item \textbf{Affirmative easement}: the right to perform an act on 
    servient land.\footnote{Casebook p. 767.}
    \item \textbf{Negative easement}: forbids a landowner from doing something 
    that might harm his neighbor.
    \item \textbf{Easement appurtenant}: grants a right to the owner of the 
    land that benefits from the easement to make use of another's land, i.e., 
    it benefits the land belonging to the easement owner. Usually 
    transferable.
    \item \textbf{Easement in gross}: grants a right to some person, 
    regardless of ownership of land, to make use of another's land, i.e., it 
    benefits the easement owner directly. 
    \item If the parties' intent is unclear, courts resolve the ambiguity in 
    favor of creating an easement appurtenant.
    \item \textbf{Dominant tenement}: the property that the easement benefits.
    \item \textbf{Servient tenement}: the property to which the easement 
    applies.
\end{enumerate}

\subsubsection{Creation of Easements}

\paragraph{\emph{Willard v. First Church of Christ, Scientist}}
~\\\\
Unlike at common law, grantors can create easements in third parties.

\begin{enumerate}
    \item McGuigan owned lots 19 and 20. She allowed churchgoers to park in 
    lot 20. She sold lot 19 to Peterson. Peterson wanted to resell it, so he 
    listed it with Willard, a realtor. Willard wanted to buy lots 19 and 20, 
    so Peterson conveyed both to him by deed in fee simple.
    \item But, Peterson did not own lot 20 when he agreed to convey it to 
    Willard, so he offered to buy it from McGuigan. She was willing to sell 
    as long as the church could continue to use the lot for parking. The deed 
    from McGuigan to Peterson for lot 20 included an easement for church 
    purposes.
    \item Willard then received the deed for lot 20 from Peterson, which did 
    not contain the language creating the easement, though Peterson told 
    Willard that the church would want to use the lot for parking.
    \item Willard learned of the easement several months later. He sued to 
    quiet title against the church.
    \item The trial court held that McGuigan and Peterson intended to convey 
    an easement, but that it was invalid because of the ``common law rule that 
    one cannot `reserve' an interest in property to a stranger to the 
    title.''\footnote{Casebook p. 769.} The rule was based on a feudal desire 
    to limit conveyances by deed in favor of conveyances by livery of seisin.
    \item Many courts have found ways of circumventing this rule, and at least 
    two states have discarded it.\footnote{Casebook p. 771.}
    \item Willard argued that the old rule should nonetheless apply because 
    grantees and title insurers relied on it, but the court found no evidence 
    to support his claim.
    \item ``~.~.~.~in the instant case the balance falls in favor of the 
    grantor's intent, and the old common law rule may not be applied to defeat 
    her intent.''\footnote{Casebook p. 772.}
\end{enumerate}

\paragraph{Licenses}

\begin{enumerate}
    \item \textbf{License}: permission to allow the licensee to do something 
    that would otherwise be a trespass---e.g., plumbers, dinner guests.
    \item Licenses are revocable. Easements are not.
    \item However, there are two circumstances in which a license is not 
    revocable:
    \begin{enumerate}
        \item A license coupled with an interest, e.g., the right to harvest 
        timber from the grantor's land.
        \item A license can become irrevocable under the rules of estoppel. 
        See \emph{Holbrook} below.
    \end{enumerate}
\end{enumerate}

\paragraph{Irrevocable License by Estoppel: \emph{Holbrook v. Taylor}}
~\\\\
If the licensee has made substantial improvements in reliance on the license, 
the licensor is estopped from revoking the license.

\begin{enumerate}
    \item Facts:
    \begin{enumerate}
        \item 1942: The Holbrooks bought the property in question.
        \item 1944: The Holbrooks granted permission for a road to be cut to 
        haul coal from a mine.
        \item 1949: The mine closed and the road was no longer used.
        \item 1957: The Holbrooks built a house and leased it. The Holbrooks 
        and the tenants used the road.
        \item 1961: The tenant house burned down and was not rebuilt.
        \item 1964: The Taylors bought property next to the Holbrooks' and 
        built a house. Until 1965, the Holbrooks gave permission for the 
        Taylors to use the road.
    \end{enumerate}
    \item Precedent supported the view that ``a right to the use of a roadway 
    over the lands of another may be established by 
    estoppel.''\footnote{Casebook p. 775.}
    \item After 1965, it was disputed whether the Taylors used the road by 
    permission or claim of right. The Taylors argued that it was used by them 
    and others without the Holbrooks' permission.
    \item In 1970, the Holbrooks apparently wanted to require the Taylors to 
    buy the land containing the road for \$500.
    \item The court held that several factors---the right to get to their 
    home, the use of the road for construction, and improvements---granted an 
    irrevocable license by estoppel to the Taylors to use the 
    road.\footnote{Casebook p. 777.}
\end{enumerate}

\paragraph{\emph{Shepard v. Purvine}}

\begin{enumerate}
    \item % FIXME 777
\end{enumerate}

\paragraph{\emph{Henry v. Dalton}}

\begin{enumerate}
    \item % FIXME 777
\end{enumerate}

\paragraph{Implied Reservation: \emph{Van Sandt v. Royster}}
~\\\\
If the previous use is apparent (though not necessarily visible), the grantee 
can have a quasi-easement.

\begin{enumerate}
    \item Bailey owned three adjacent lots. She built an underground sewer to 
    connect the lots to a public sewer on the street. Van Sandt, the 
    plaintiff, acquired the lot closest to the road (lot 19). Royster, the 
    defendant, acquired the adjacent lot (lot 20).
    \item Van Sandt discovered that his basement was flooded with sewage. He 
    sued to prevent the defendatns from draining their sewage across his land.
    \item The trial court found an appurtenant easement, holding for 
    defendants.
    \item On appeal, Van Sandt argued that there was never an easement 
    created in his land, and if there was, he took the premises free from any 
    easement because there was no actual or constructive 
    notice.\footnote{Casebook p. 780.}
    \item Royster argued that Bailey (the original owner of the lots) created 
    an easement by implied reservation when she severed the lots.
    \item The court here held that an owner cannot have an easement in his 
    own land, but he can have a ``quasi-easement'' if one part of the land is 
    necessary for the benefit of another. After surveying conflicting 
    authority, the court held that an implied reservation exists when the 
    grantee knew or reasonably should have known of the prior use. It doesn't 
    matter whether the use was visible to the naked eye as long as the use was 
    reasonably apparent (e.g., underground sewers).
    \item Affirmed.
\end{enumerate}

\paragraph{Implied Easements}

\begin{enumerate}
    \item % FIXME 785
\end{enumerate}

\paragraph{\emph{Othen v. Rosier}}

\begin{enumerate}
    \item % FIXME 786
\end{enumerate}

\paragraph{Easements by Necessity}

\begin{enumerate}
    \item % FIXME 792
\end{enumerate}

\paragraph{Problems on Easements by Necessity}

\begin{enumerate}
    \item % FIXME 793
    \item \footnote{Casebook p. 793 problem 2.}
\end{enumerate}

\subsubsection{Easements by Prescription}

\begin{enumerate}
    \item % FIXME 794
\end{enumerate}

\paragraph{Problems on Easements by Prescription}

\begin{enumerate}
    % FIXME 794
    \item \footnote{Casebook p. 794 note 3.}
    % FIXME 796
    \item \footnote{Casebook p. 796, problems at the end of note 1.}
    % FIXME 797
    \item \footnote{Casebook p. 797, problem 4 (first paragraph only).}
\end{enumerate}

\subsubsection{Negative Easements}

% FIXME 842-45

\paragraph{\emph{Fontainebleu Hotel Corp. v. Forty-Five Twenty-Five Inc.}}

\begin{enumerate}
    \item % FIXME supp 215
\end{enumerate}


\subsection{Covenants Running with the Land}

% TODO 847 ff.
