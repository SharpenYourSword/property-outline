\section{Future Interests}

\begin{enumerate}
    \item ``A future interest is not a mere expectancy, like the hope of a 
    child to inherit from a parent. A future interest gives legal rights to 
    its owner. It is a \textbf{presently protected property interest}, 
    protected by the court as such.''\footnote{Casebook p. 254.}
    \item Five types: remainders, executory interests, reversions, 
    possibilities of reverter, and rights of entry.
    \begin{enumerate}
        \item The Restatement (Third) simply categorizes all five as ``future 
        interests.''\footnote{Casebook p. 272.}
    \end{enumerate}
    % TODO: other restatement third reforms: pp. 272--73
\end{enumerate}

\subsection{Future Interests in the Transferor}

\subsubsection{Reversion}

\begin{enumerate}
    \item A reversion is ``the interest remaining in the grantor, or in the 
    successor in interest of a testator, who transfers a \textbf{vested estate 
    of a lesser quantum} than that of the vested estate which he 
    has.''\footnote{Casebook p. 255.}
    \item ``[A]ll reversions are retained interests, which remain vested in 
    the transferor.''\footnote{Casebook p. 255.}
    \item Example: if O has a fee simple and conveys a life estate to A, O has 
    a reversion, which takes effect when A's life estate ends.
    \item There is no such thing as a ``possibility of reversion.''
\end{enumerate}

\subsubsection{Possibility of Reverter}

\begin{enumerate}
    \item ``A possibility of reverter arises when an owner carves out of his 
    estate a \textbf{determinable estate of the same 
    quantum}.''\footnote{Casebook p. 256.}
    \item It almost always arises when carving a fee simple determinable out 
    of a fee simple absolute.
    \item Example: O conveys property ``so long as it is used for school 
    purposes.'' When it is no longer used for school purposes, it reverts to 
    O.
\end{enumerate}

\subsubsection{Right of Entry}

\begin{enumerate}
    \item ``When an owner transfers an estate subject to condition subsequent 
    and retains the power to cut short or terminate the estate, the transferor 
    has a right of entry.''\footnote{Casebook p. 257.}
    \item Example: O conveys land ``to the school board, but if it ceases to 
    use the land for school purposes, O has the right to reenter and retake 
    the premises.
\end{enumerate}

\subsection{Future Interests in Transferees}

\subsubsection{Remainders}

\begin{enumerate}
    \item A remainder is ``a future interest that waits politely until the 
    termination of the preceding possessory estate, at which time the 
    remainder moves into possession if it is then vested.''\footnote{Casebook 
    p. 258.}
\end{enumerate}

\subsubsection{Vested and Contingent Remainders}

\begin{enumerate}
    \item \textbf{Vested}: if (1) given to an ascertained person and (2) not 
    subject to a condition precedent.
    \begin{enumerate}
        \item \emph{Indefeasibly vested}:  ``to A for life, then to B and her 
        heirs.'' B has an indefeasibly vested remainder because it is certain 
        to become possessory and cannot be divested.
        \item Remainders can also be vested but subject to being divested if 
        an event happens. For instance, ``to A for life, then to B and her 
        heirs, but if B does not survive A, to C and his 
        heirs.''\footnote{Casebook p. 261 example 8.} B has a vested remainder 
        in fee simple subject to divestment.
        \item If later-born children are entitled to share in the gift, the 
        remainder is \emph{vested subject to open} (or \emph{vested subject to 
        partial divestment}).\footnote{Casebook p. 260.} For instance, ``to A 
        for life, then to A's children and their heirs.'' A has one child, B. 
        B has a vested remainder subject to open.
    \end{enumerate}
    \item \textbf{Contingent}: if (1) given to an unascertained person or (2) 
    contingent upon an event other than the natural termination of the 
    preceding estates.
    \begin{enumerate}
        \item Example: ``to A for life, then to the heirs of B.'' The 
        remainder is contingent because the heirs of B cannot be ascertained 
        until B dies.\footnote{Casebook p. 260 example 5.}
        \item Example: ``to A for life, then to B and his heirs if B survives 
        A.''\footnote{Casebook p. 261 example 6.} The condition precedent is 
        B's survival of A. The remainder is contingent on satisfying this 
        condition.
    \end{enumerate}
    % TODO: examples 6 + 7 on p. 261; see reader
\end{enumerate}

\subsubsection{Executory Interests}

\begin{enumerate}
    \item ``An executory interest is a future interest in a transferee that 
    must, in order to become possessory,
    \begin{enumerate}
        \item divest or cut short some interest in another \emph{transferee} 
        (this is known as a \textbf{shifting executory interest}), or
        \item divest the \emph{transferor} in the future (this is known as a 
        \textbf{springing executory interest}).''\footnote{Casebook p. 264.}
    \end{enumerate}
    \item Example: O conveys ``to A and his heirs, but if A dies without issue 
    surviving him, to B and his heirs.'' B has an executory interest, which 
    can only become possessory by divesting A.\footnote{Casebook p. 268.}
    \item ``The difference between taking possession as soon as the prior 
    estate ends and divesting the prior estate is the essential difference 
    between a remainder and an executory interest.''\footnote{Casebook p. 259.}
    \item An executory interest can be created only in a 
    transferee.\footnote{Casebook p. 269.}
    \item Executory interests are always vest automatically.
\end{enumerate}

% TODO: review problems p. 271.

\subsection{The Trust}

\begin{enumerate}
    \item Trusts distinguish between legal and equitable titles. The trustee 
    holds legal title to the trust property and manages it for the benefit of 
    the beneficiaries, who hold equitable title.
    \item Settlors of trusts can protect the beneficiaries' interests by 
    making the trust inalienable. ``~.~.~.~trusts can be drafted in such a way 
    that trust beneficiaries have no power to transfer or borrow against their 
    trust interests, and creditors have no power to reach those interests to 
    satisfy beneficiaries' debts.''\footnote{Casebook p. 275.}
\end{enumerate}

\subsubsection{\emph{Broadway Natl. Bank v. Adams}}

\begin{enumerate}
    \item % TODO 276-278
\end{enumerate}

\subsubsection{Gray, ``Restraints on Alienation of Property''}

\begin{enumerate}
    \item % TODO 278-79
\end{enumerate}

\subsection{Rules Furthering Marketability by Destroying Contingent Future 
Interests}

\subsubsection{Destructability of Contingent Remainders}

\begin{enumerate}
    \item ``A legal remainder in land is destroyed if it does not vest at or 
    before the termination of the preceding freehold 
    estate.''\footnote{Casebook p. 281.} Destroying contingent remainders 
    increases alienability.
    \item Not all jurisdictions follow the DDCR.
    \item Example: O conveys to A for life, then to B if B reaches 21. Upon 
    A's death, if B is not 21, B's interest is destroyed.
    \item \textbf{Merger}: ``if the life estate and the next vested estate in 
    fee simple come into the hands of one person, the lesser estate is merged 
    into the larger.''\footnote{Casebook p. 282.}
    \begin{enumerate}
        \item Example: O conveys to A for life, remainder to B. If A conveys 
        her life estate to B, the life estate merges into B's remainder, 
        giving B a fee simple.
        \item Example: O conveys to A for life, and then to B if B survives A. 
        A conveys his life estate to O. The life estate merges with O's fee 
        simple, giving O a fee simple and destroying B's contingent remainder.
    \end{enumerate}
\end{enumerate}

\subsubsection{The Doctrine of Worthier Title}

\begin{enumerate}
    \item % TODO 284-85
\end{enumerate}

\subsubsection{The Rule Against Perpetuities}
% TODO: practice problems in reader, p. 87
\paragraph{The Common Law Rule}

\begin{enumerate}
    \item The rule resulted from a struggle between landowners and judges. 
    They settled on a period of ``lives in being plus 21 years 
    thereafter.''\footnote{Casebook p. 285.}
    \item Gray: ``No interest is good unless it must vest, if at all, not 
    later than twenty-one years after some life in being at the creation of 
    the interest.''\footnote{Casebook p. 285.}
    \item The rule applies only to interests that are \textbf{not vested at 
    the time of conveyance}: contingent remainders, executory 
    interests, and class gifts.
    \item The interest must \emph{vest or terminate} within the perpetuities 
    period (life in being plus 21 years).\footnote{Casebook p. 286.}
    \item Applying the rule:
    \begin{enumerate}
        \item Is the interest vested at the time of the conveyance? The rule 
        applies only to interests that are \textbf{not vested at the time of 
        conveyance}. In other words, the rule applies to only three interest: 
        contingent remainders, executory interests, and class gifts.
        \item Will the interest vest within the perpetuity period of 
        \emph{any} \textbf{life in being plus 21 years}? If you can 
        \emph{prove} that it will, then the interest is valid. But if you 
        cannot prove the existence of scenario where it would vest, the 
        interest is invalid.
    \end{enumerate}
    \item The 21 year period is pegged to a \textbf{validating life} (or 
    measuring life or life in being), defined as ``someone who can affect 
    vesting or termination of the interest.''\footnote{Casebook pp. 286--87.} 
    The person used as the validating life \textbf{must be alive at the time 
    the interest is created.}
    \item Basic examples (see the end of this section for more complicated 
    problems):
    \begin{enumerate}
        \item Valid: O transfers land ``in trust for A to life, then to A's 
        first child to reach 21.'' A is the validating life. The interest in 
        A's first child to reach 21 will necessarily vest prior to A's life 
        plus 21 years. Since you can prove that the interest must vest within 
        this period, the remainder is valid.\footnote{Casebook pp. 286--87.}
        \item Invalid: O transfers land ``in trust to A for life, then to A's 
        first child to reach 25.'' The interest will not necessarily vest 
        before after A's life plus 21 years. Thus, the remainder is invalid.
        \item Invalid: to A and his heirs so long as used for school purposes, 
        and then to B and his heirs. The interest will not necessarily vest or 
        terminate within A or B's lifetimes.
    \end{enumerate}
    \item The ``presumption of life fertility'' assumes that anyone can have a 
    child at any time---hence the ``fertile octogenarian'' and the 
    ``precocious toddler.''\footnote{Casebook p. 288.}
    \item ``A dozen healthy babies'': grantors will name a dozen infants in 
    the hopes that at least one of them will live long, thereby providing a 
    measuring life that maximizes the total perpetuities period.
    \item Class gifts follow the \textbf{all-or-nothing rule}, which holds 
    that ``if a gift to one member of the class might vest too remotely, the 
    whole class gift is void.'' % TODO: fill in after revising p. 260; and 
    % then outline 288-89
\end{enumerate}

\paragraph{RAP Examples}

\begin{enumerate}
    \item ``To A and his heirs so long as used for school purposes, and then 
    to B and his heirs.'' The conveyance is invalid because it may not vest 
    until more than 21 years after the deaths of A and B.
        % TODO: what interest does B have?
\end{enumerate}

\paragraph{Analyzing a RAP Problem}

\begin{enumerate}
    \item Identify the interests.
    \item Is each interest valid under the RAP?
    \item If one or more interests is invalid, what is the remedy for the 
    violation of the RAP?\footnote{Reader p. 73.}
\end{enumerate}

\paragraph{Possibilities of Reverter under the RAP: \emph{Brown v. Independent 
Baptist Church of Woburn}}
~\\\\
The RAP does not apply to reversionary interests, including possibilities of 
reverter.

\begin{enumerate}
    \item Sarah Converse died in 1849. Her will included two provisions 
    regarding the land she wanted to devise:
    \begin{enumerate}
        \item \emph{Executory devise}: to the Independent Baptist Church of 
        Woburn, so long as they keep their faith and use the land as a church, 
        and then to ``my legatees hereinafter named.''\footnote{Reader p. 73.} 
        \item \emph{Residuary gift}: everything else to her husband for life, 
        and then to her legatees.
    \end{enumerate}
    \item The husband died in 1864. The church stopped functioning as a church 
    in 1939.
    \item The parties agreed (1) the church had a fee simple determinable and 
    (2) the executory devise to the legatees was void for remoteness, because 
    it might not come into being until long after any life in being plus 21 
    years.
    \item Since the executory interest was void, the question was what should 
    become of the possibility of reverter upon the termination of the church's 
    determinable fee.
    \item The court held that Converse's possibility of reverter passed to the 
    legatees under the residuary gift. ``~.~.~.~the rule against perpetuities 
    does not apply to reversionary interests of this general type, including 
    possibilities of reverter.''\footnote{Reader p. 76.}
    \item Simes and Smith argue that she had no possibility of reverter to 
    devise when she died, because the possibility of reverter is not created 
    until the determinable fee is created.\footnote{Reader p. 73.}
    \item Leach: it is wrong the RAP should apply to executory interests but 
    not possibilities of reverter. Here, the exemption of the possibility of 
    reverter from the RAP meant that the title was unmarketable, which was 
    likely a significant factor in the church's demise.\footnote{Casebook p. 
    77.}
\end{enumerate}

\paragraph{``Six Feet Under and Overbearing''}
\begin{enumerate}
    \item The ``incentive trust'' allow wealthy benefactors to control the 
    behavior of benefactees---for instance, awarding an inheritance only when 
    the benefactee gets married.\footnote{Reader p. 79.}
\end{enumerate}

\paragraph{\emph{City of Klamath Falls v. Flitcraft}}

\begin{enumerate}
    \item Facts:
    \begin{enumerate}
        \item 1925: The Daggett-Schallock Investment Company conveyed land to 
        the city of Klamath Falls for use as a site for a city library (``so 
        long as''---a fee simple determinable). Upon termination, the land 
        should pass to Fred Schallock and Floy Daggett (or their heirs and 
        assigns).
        \item 1927: The corporation was dissolved.
        \item 1969: The city stopped using the land for a library. The city 
        asked the court to adjudicate the rights of the parties under the 
        deed.
        \item 1970: The defendants conveyed their interests to Flitcraft.
    \end{enumerate}
    \item Issue: ``Does the title to the land remain in the city or did the 
    termination of use as a library cause title to pass to the descendants of 
    the shareholders of the donor-corporation (now 
    dissolved)?''\footnote{Reader p. 91.}
    \item The trial court held that the title was vested in the city because 
    Schallock and Daggett's interest was void under the RAP. The appellate 
    court affirmed, holding that the gift over to Schallock and Daggett was 
    void because the city \emph{could} have maintained a library on the site 
    forever (i.e., longer than any life in being plus 21 years).
    \item The next question was whether the grantor corporation retained a 
    possibility of reverter. The appellate court held that although Schallock 
    and Daggett's executory interest was void under the RAP, the corporation 
    retained a possiblity of reverter, which was \emph{not} void (\emph{Brown 
    v. Independent Baptist Church of Woburn} and others).
    \item But ``the corporation was civilly dead without a successor to whom 
    the possibility of reverter could descend.''\footnote{Reader p. 93.} The 
    court held that Daggett and Schallock, as the corporation's sole 
    shareholders, were entitled to receive the corporation's remaining 
    assets---including the possibility of reverter in question. Therefore, the 
    possibility of reverter in the land belongs to the defendants (here, all 
    rights were consolidated to Marjorie Flitcraft).
    \item In 1925, how should Daggett-Schallock's lawyers have advised the 
    company to carry out its desire to have the land forfeited back to 
    Schallock and Daggett after it was no longer used for library purposes?
    \begin{enumerate}
        \item They could have used a two-step transaction: (1) create a FSD 
        and (2) transfer the possibility of reverter to Schallock and Daggett.
        \begin{enumerate}
            \item This is a major loophole. Why should you be able to get 
            around the RAP like this? Maybe interests in the transferor 
            (reversion, possibility of reverter, right of entry) should also 
            be subject to the RAP.
        \end{enumerate}
    \end{enumerate}
\end{enumerate}

\paragraph{\emph{Jee v. Audley}}

% TODO: verify with supplement

\begin{enumerate}
    \item In his will, Edward Audley left \pounds 1000 to his wife, and upon 
    her death, to his niece Mary Hall ``and the issue of her body.'' Then it 
    should go to the daughters of John and Elizabeth Jee.
    \item Audley's wife died. Mary Hall was about 40, with no husband or 
    children. The daughters of the Jees sued to secure the \pounds 1000 upon 
    Hall's death.
    \item The question was ``whether the limitation to the daughters of John 
    and Elizabeth Jee was not void as being too remote.''\footnote{Reader p. 
    95.}
    \item Can any of the following be valid measuring lives?
    \begin{enumerate}
        \item \emph{Edward Audley}: no. Mary Hall might die more than 21 years 
        after Audley.
        \item \emph{Edward Audley's wife}: no. Mary Hall might die more than 
        21 years after Audley's wife's death.
        \item \emph{Mary Hall}: No. The Jees might have 
        \item \emph{John or Elizabeth Jee}:
        \item \emph{One of the four Jee daughters}:
    \end{enumerate}
\end{enumerate}

\paragraph{\emph{The Symphony Space, Inc. v. Pergola Properties, Inc.}}

\begin{enumerate}
    \item 1978: Broadwest sold a building to Symphony Space for the 
    below-market price of \$10,010. Broadwest leased back the commercial 
    portions of the building for \$1 per year, while Symphony retained the 
    theater portion. The purpose of the agreement was to allow Symphony 
    Space, a nonprofit, to claim a property tax exemption for the entire 
    building.
    \item The parties signed several other documents dated December 1, 1978:
    \begin{enumerate}
        \item A deed conveying the property from Broadwest to Symphony.
        \item A lease to Broadwest from 1979 to 2003 for \$1 per year.
        \item A 25-year mortgage between the two parties.
        \item An option agreement giving Broadwest the exclusive right to 
        repurchase all of the property.
    \end{enumerate}
    \item The option agreement was in question here. It allowed Broadwest to 
    repurchase the property up to 2003, or 24 years after the initial 
    agreement.
    \item 1981: Broadwest transferred its interests to Pergola.
    \item 1985: Pergola tried to exercise its option to buy back the property. 
    Symphony claimed the option was void under the RAP. (The property had been 
    designated a historical landmark and its value had increased 
    significantly.)
    \item The trial court held that the option was void under the RAP.
    \item The court here reasoned that applying the RAP to commercial land 
    options hinders the owner's incentive to improve the property and hinders 
    its alienability [---why?]. Nonetheless, the RAP had to apply to such 
    options unless the legislature decided otherwise.\footnote{Casebook p. 
    297.}
    \item Because the option could have vested beyond the perpetuities period, 
    it was invalid.
    \item (The USRAP abolished the application of the RAP to options and other 
    commercial transactions.\footnote{Casebook p. 304.})
\end{enumerate}

\paragraph{``Postscript on the Rule Against Perpetuities''}

\begin{enumerate}
    \item % TODO supp
\end{enumerate}

\paragraph{L. Simes, ``Public Policy and the Dead Hand''}

\begin{enumerate}
    \item % TODO supp
\end{enumerate}

\paragraph{The Perpetuity Reform Movement}

\begin{enumerate}
    \item % TODO 305-310
\end{enumerate}

\paragraph{Dukeminier \& Krier, ``The Rise of the Perpetual Trust''}

\begin{enumerate}
    \item % TODO 310-315
\end{enumerate}

\paragraph{Silverman, ``Amid Congressional Scrutiny, Huge Sums Pour into 
States that Allow \enquote{Dynasty Trusts}''}

\begin{enumerate}
    \item % TODO 315-317
\end{enumerate}

\paragraph{Reforming the Rule Against Perpetuities}

\begin{enumerate}
    \item What concerns is the RAP designed to address? Are these concerns 
    serious?\footnote{See syllabus 2/11/2013.}
    \begin{enumerate}
        % TODO
        \item What's wrong with ``dead hand control''?
        \begin{enumerate}
            \item Dead hand control is the ability of landowners (or property 
            owners) to control their possessions after they die. Such control 
            can hinder the property's marketability by creating uncertainty in 
            title. There is also a moral dilemma in allowing a single, mortal 
            person to control property for all time. Finally, dead hand 
            control can serve to consolidate wealth and perpetuate income 
            inequality, although estate taxes might mitigate this effect.
        \end{enumerate}
    \end{enumerate}
    \item How close is the fit between the RAP's provisions and the concerns 
    it is designed to address?
    \begin{enumerate}
        \item It generally achieves its goal, but sometimes it leads to absurd 
        results, like the cases of fertile octogenarian and the precocious 
        toddler. It can also lead to results that are clearly contrary to the 
        conveyor's intent, as in the case of the unborn widow.
        \item Why is the possibility of remote vesting problematic?
        \begin{enumerate}
            \item % TODO
        \end{enumerate}
        \item What does the ``vested'' or ``contingent'' nature of a future 
        interest have to do with concerns that underlie the RAP?
        \begin{enumerate}
            \item % TODO
            \item Vested interests (remainders, rights of entry, possibilities 
            of reverter) raise the same concerns, but they are exempted from 
            the common law RAP.
        \end{enumerate}
    \end{enumerate}
    \item How could the concerns underlying the RAP be addressed in other 
    ways? What would an alternative rule look like?
    \begin{enumerate}
        \item % TODO
        % TODO \item ``a dozen healthy babies'': reform the rule to require lives in being that are logically connected to the conveyance.
    \end{enumerate}
\end{enumerate}

\paragraph{Problems on the Rule Against Perpetuities}
% TODO: practice problems, reader p. 87
\begin{enumerate}
    \item O conveys ``to A for life, then to B if B attains the age of 30.'' B 
    is now two years old. Is the interest valid?\footnote{Casebook p. 289 
    problem 1.}
    \begin{enumerate}
        \item Yes. You can use B as the measuring life because B was alive 
        when the interest was created. If you use B as the measuring life, the 
        interest will vest within the period of the life in being plus 21 
        years.
    \end{enumerate}
    \item O conveys ``to A for life, then to the first child of A to reach the 
    age of 30.'' A's oldest child is B, age two. Is the interest 
    valid?\footnote{Syllabus 2/6/2013 problem 2 (p. 464, problem 2). See 
    reader p. 73.}
    \begin{enumerate}
        \item No. A might die tomorrow, which would 
        mean that B's interest would not vest until A's life plus 28 
        years.
    \end{enumerate}
    \item O conveys ``to A for life, then to A's widow, if any, for life, then 
    to A's issue then living.'' Is the conveyance valid?\footnote{Syllabus 
    2/6/2013 problem 2 (p. 464, problem 4). See reader p. 74.}
    \begin{enumerate}
        \item No. The only relevant lives are A and A's widow:
        \begin{enumerate}
            \item \emph{A}: no. A's widow could live more than 21 years, which 
            would mean that A's issues' interest would vest after A's life 
            plus 21 years.
            \item \emph{A's widow}: no. This is the \textbf{unborn widow 
            problem}. A's widow may not be alive at the time the interest is 
            created, so A's widow is not a valid measuring life.
        \end{enumerate}
    \end{enumerate}
    \item Assume the same facts as the previous question, but suppose the 
    conveyance were ``to A for life, then to A's widow, if any, for life, then 
    to B and his heirs.''\footnote{Syllabus 2/6/2013 problem 2 (p. 464, 
    problem 4). See reader p. 74.}
    \begin{enumerate}
        \item Yes. B has a vested remainder in fee simple absolute. The RAP 
        only applies to non-vested interests.
    \end{enumerate}
    \item O conveys land ``to the school board so long as it is used for 
    school purposes.''\footnote{Casebook p. 291 ex. 33.}
    % TODO: other problems on pp. 291-92
    \begin{enumerate}
        \item The school board has a FSD. O has a possibility of reverter, 
        which is exempt from the RAP. Future interests retained by the 
        transferor---reversions, possibilities of reverter, and rights of 
        entry---are considered vested and are exempt.
    \end{enumerate}
\end{enumerate}

\paragraph{Sample Exam Question}
~\\\\
O, owner of Blackacre in fee simple absolute, devised Blackacre ``to A and his 
heirs so long as the property is used for a non-profit lending library for 
children, but if the property is not used for such a library or ceases to be 
used for such a library, then B and his heirs, successors or assigns have the 
right to re-enter and re-take the premises.''

B would have been O's sole heir if O had died intestate.

Blackacre is a 10-acre parcel of land with a house on 1/4 of an acre. After O 
died, A converted the house into a non-profit lending library for children and 
used it as such a library for 5 years. A then sold the property to C, who 
converted the building back into a residence and used it for residential 
purposes. C was an elderly man who seldom emerged from his home and never used 
the rest of the property.

Twenty years after C acquired the property from A, C died, devising Blackacre 
``to D for life, then to E and his heirs if E survives D, and if E does not 
survive D, to F and his heirs.''

In this jurisdiction, the statutory period for recovery of possession of real 
property is 15 years, and the common-law Rule Against Perpetuities has not 
been modified.

Who owns what interests in Blackacre?

\begin{enumerate}
    \item O's initial conveyance to A was an \textbf{ambiguous conveyance}. It 
    could have given A either a FSD or FSSEL. Courts generally resolve 
    ambiguous conveyances in favor of optional rather than automatic 
    forfeiture. But in this case, both possibilities involve automatic 
    forfeiture, so the court would have to use its discretion to choose one or 
    the other. Then, it would rewrite the instrument to remove the ambiguity.
    \item Option 1: \textbf{A has a FSD} because of the durational language 
    (``so long as the property is no longer used as a non-profit lending 
    library'').
    \begin{enumerate}
        \item A's interest is valid under the RAP because it is vested.
        \item O's devise to A in FSD gave O a possibility of reverter. Upon 
        O's death, O's future interest passed to B, his sole heir. So, 
        \textbf{B has a possibility of reverter}. The facts here are analogous 
        to \emph{Brown v. Independent Baptist Church of Woburn}, where the 
        court held that possibilities of reverter are exempt from the RAP. 
        Thus, B's possibility of reverter is valid under the RAP.
        \item After five years, A sold the property to C. The title 
        automatically transferred to B because of B's possibility of reverter. 
        But, B did not take physical possession of the property. C remained in 
        physical possession of 1/4 of an acre (out of 10 acres total). C 
        possessed the property for twenty years. The statutory period for 
        adverse possession was fifteen years. Did C meet the requirements for 
        adverse possession?
        \begin{enumerate}
            \item \emph{Actual entry giving exclusive possession}: yes---C 
            acquired what he thought was a title to the land, and he 
            physically occupied it and used it as his home.
            \item \emph{Open and notorious possession}: yes---although C was 
            reclusive, he openly used the 1/4 of an acre as a residence. The 
            policy behind this requirement is to penalize owners for sleeping 
            on their rights. B had a duty to check whether anyone was living 
            in the building.
            \item \emph{Possession that is adverse}: yes---C occupied 
            Blackacre without B's consent.
            \item \emph{Possession that is continuous for the statutory 
            period}: yes.
            \item Depending on the jurisdiction's rules, C may have been 
            required to pay property taxes.
            \item \textbf{C adversely possessed Blackacre}. Because he 
            occupied the land under color of title, constructive adverse 
            possession entitled him to ownership of the entire 10 acres, with 
            a few possible exceptions:
            \begin{enumerate}
                \item If the jurisdiction has a ``reasonable proportion'' 
                requirement, C may not be entitled to ownership of the full 10 
                acres.
                \item We don't know the details of C's deed. If the 
                jurisdiction follows the objective test, C's understanding of 
                the deed and his intentions are irrelevant. But if the 
                jurisdiction follows the subjective test, we would have to 
                address his understanding of the deed and his intentions. Some 
                courts (like the \emph{Lutz} court) require a ``mentality of 
                thievery,'' but others require good faith.
            \end{enumerate}
        \end{enumerate}
    \end{enumerate}
    \item Option 2: \textbf{A has a FSSEL} because the devise created a 
    future interest in a third party. \textbf{B has an executory interest.}
    \begin{enumerate}
        \item B's executory interest is \textbf{not valid under the RAP}. A 
        and B are the only possible measuring lives, and it's not possible to 
        prove that B's interest will vest within the life of either A or B 
        plus 21 years. For instance, the property may be used as a non-profit 
        lending library for the next two hundred years.
        \item The court would strike the invalid clause, after which \textbf{A 
        owns Blackacre in FSA.}
        \item After five years, A sold Blackacre to C. \textbf{C owns 
        Blackacre in FSA}. It doesn't matter how C used the property.
    \end{enumerate}
    \item Under both options, \textbf{C owns all of Blackacre in FSA.}
    \item C died, conveying Blackacre to D for life, then to E and his 
    heirs if E survives D, and if E does not survive D, to F and his 
    heirs.
    \item \textbf{D has a life estate} and \textbf{E and F have contingent 
    remainders}. Both E and F's remainders are valid under the RAP because 
    both are certain to vest or terminate within 21 years of D's death.
\end{enumerate}
