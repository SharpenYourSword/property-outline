\section{Future Interests}

\begin{enumerate}
    \item ``A future interest is not a mere expectancy, like the hope of a 
    child to inherit from a parent. A future interest gives legal rights to 
    its owner. It is a \textbf{presently protected property interest}, 
    protected by the court as such.''\footnote{Casebook p. 254.}
\end{enumerate}

\subsection{Future Interests in the Transferor}

\subsubsection{Reversion}

\begin{enumerate}
    \item A reversion is ``the interest remaining in the grantor, or in the 
    successor in interest of a testator, who transfers a \textbf{vested estate 
    of a lesser quantum} than that of the vested estate which he 
    has.''\footnote{Casebook p. 255.}
    \item ``[A]ll reversions are retained interests, which remain vested in 
    the transferor.''\footnote{Casebook p. 255.}
    \item Example: if O has a fee simple and conveys a life estate to A, O has 
    a reversion, which takes effect when A's life estate ends.
    \item There is no such thing as a ``possibility of reversion.''
\end{enumerate}

\subsubsection{Possibility of Reverter}

\begin{enumerate}
    \item ``A possibility of reverter arises when an owner carves out of his 
    estate a \textbf{determinable estate of the same 
    quantum}.''\footnote{Casebook p. 256.}
    \item It almost always arises when carving a fee simple determinable out 
    of a fee simple absolute.
    \item Example: O conveys property ``so long as it is used for school 
    purposes.'' When it is no longer used for school purposes, it reverts to 
    O.
\end{enumerate}

\subsubsection{Right of Entry}

\begin{enumerate}
    \item ``When an owner transfers an estate subject to condition subsequent 
    and retains the power to cut short or terminate the estate, the transferor 
    has a right of entry.''\footnote{Casebook p. 257.}
    \item Example: O conveys land ``to the school board, but if it ceases to 
    use the land for school purposes, O has the right to reenter and retake 
    the premises.
\end{enumerate}

\subsection{Future Interests in Transferees}

\subsubsection{Remainders}

\begin{enumerate}
    \item % TODO 259
\end{enumerate}

\subsubsection{Vested and Contingent Remainders}

\begin{enumerate}
    \item % TODO 262-63
\end{enumerate}

\subsubsection{Executory Interests}

\begin{enumerate}
    \item % TODO 264-273
\end{enumerate}

\subsection{The Trust}

\subsubsection{\emph{Broadway Natl. Bank v. Adams}}

\begin{enumerate}
    \item % TODO 276-278
\end{enumerate}

\subsubsection{Gray, ``Restraints on Alienation of Property''}

\begin{enumerate}
    \item % TODO 278-79
\end{enumerate}

\subsection{Rules Furthering Marketability by Destroying Contingent Future 
Interests}

\subsubsection{Destructability of Contingent Remainders}

\begin{enumerate}
    \item % TODO 281-283
\end{enumerate}

\subsubsection{The Doctrine of Worthier Title}

\begin{enumerate}
    \item % TODO 284-85
\end{enumerate}

\subsubsection{The Rule Against Perpetuities}

\paragraph{The Common Law Rule}

\begin{enumerate}
    \item The rule resulted from a struggle between landowners and judges. 
    They settled on a period of ``lives in being plus 21 years 
    thereafter.''\footnote{Casebook p. 285.}
    \item Gray: ``No interest is good unless it must vest, if at all, not 
    later than twenty-one years after some life in being at the creation of 
    the interest.''\footnote{Casebook p. 285.}
    \item Applying the rule:
    \begin{enumerate}
        \item Is the interest vested at the time of the conveyance? The rule 
        applies only to interests that are \textbf{not vested at the time of 
        conveyance}. In other words, the rule applies to only three interest: 
        contingent remainders, executory interests, and class gifts.
        \item Will the interest vest within the perpetuity period of 
        \textbf{lives in being plus 21 years}? If you can \emph{prove} that it 
        will, then the interest is valid. But if there is any possible 
        scenario where it would not vest, the interest is invalid.
    \end{enumerate}
    \item The 21 year period is pegged to a \textbf{validating life}, defined 
    as ``someone who can affect vesting or termination of the 
    interest.''\footnote{Casebook pp. 286--87.}
    \item Basic examples:
    \begin{enumerate}
        \item Valid: O transfers land ``in trust for A to life, then to A's 
        first child to reach 21.'' A is the validating life. The interest in 
        A's first child to reach 21 will necessarily vest prior to A's life 
        plus 21 years. Since you can prove that the interest must vest within 
        this period, the remainder is valid.\footnote{Casebook pp. 286--87.}
        \item Invalid: O transfers land ``in trust to A for life, then to A's 
        first child to reach 21.'' The interest will not vest until after A's 
        life plus 21 years. Thus, the remainder is invalid.
        \item The ``presumption of life fertility'' assumes that anyone can 
        have a child at any time---hence the ``fertile octogenarian'' and the 
        ``precocious toddler.''\footnote{Casebook p. 288.}
    \end{enumerate}
    \item Class gifts follow the \textbf{all-or-nothing rule}, which holds 
    that ``if a gift to one member of the class might vest too remotely, the 
    whole class gift is void.'' % TODO: fill in after revising p. 260; and 
    % then outline 288-89
\end{enumerate}

\paragraph{Problems}

\begin{enumerate}
    \item O conveys ``to A for life, then to B if B attains the age of 30.'' B 
    is now two years old. Is the interest valid?\footnote{Casebook p. 289 
    problem 1. See reader p.}
    \begin{enumerate}
        \item No. If A died tomorrow, B's interest would not vest until the 
        life of A plus 28 years. The rule limit is life plus 21 years. Thus, 
        B's interest is invalid.
    \end{enumerate}
    \item O conveys ``to A for life, then to the first child of A to reach the 
    age of 30.'' A's oldest child is B, age two. Is the interest valid?
    \begin{enumerate}
        \item No. As in the previous period, A might die tomorrow, which would 
        mean that B's interest would not vest until A's life plus 28 
        years.\footnote{Syllabus 1/6/2013 problem 2. See reader p.  TODO.}
    \end{enumerate}
    \item O conveys ``to A for life, then to A's widow, if any, for life, then 
    to A's issue then living.'' Is the Rule against Perpetuities 
    violated?\footnote{Syllabus 1/6/2013 problem 2. See reader p.  TODO.}
    \begin{enumerate}
        \item % TODO
    \end{enumerate}
    \item Assume the same facts as the previous question, but suppose the 
    conveyance were ``to A for life, then to A's widow, if any, for life, then 
    to B and his heirs.''
    \begin{enumerate}
        \item % TODO
    \end{enumerate}
\end{enumerate}

\subsubsection{\emph{Brown v. Independent Baptist Church of Woburn}}

\begin{enumerate}
    \item % TODO reader
\end{enumerate}

\subsubsection{``Six Feet Under and Overbearing''}
% TODO cmos capitalization?
\begin{enumerate}
    \item % TODO 
\end{enumerate}


