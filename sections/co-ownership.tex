\section{Co-Ownership and Marital Interests}

\begin{enumerate}
    \item Focus: \emph{concurrent} interests in present or future possession.
\end{enumerate}

\subsection{Common Law Concurrent Interests}

\subsubsection{Types, Characteristics, and Creation}

\begin{enumerate}
    \item Three main types:
    \begin{enumerate}
        \item \textbf{Tenants in common}: separate but undivided 
        interests.\footnote{Casebook p. 319.}
        \begin{enumerate}
            \item Example: O conveys property to A and B.
            \item The interest of each is descendable and may be conveyed by 
            deed or will.
            \item No survivorship rights---i.e., the interests are distinct 
            even if one owner dies.
            \item Each tenant owns an undivided share of the whole.
        \end{enumerate}
        \item \textbf{Joint tenants}: regarded as a single 
        owner.\footnote{Casebook p. 320.}
        \begin{enumerate}
            \item Both have a right of survivorship---so if one dies, that 
            tenant's interest is simply extinguished.\
            \item Four unities are required:
            \begin{enumerate}
                \item \emph{Time}: all interests must be acquired or vest 
                simultaneously.
                \item \emph{Title}: all interests must be acquired by the same 
                instrument or by joint adverse possession.
                \item \emph{Interest}: all must have equal undivided shares 
                and identical interests measured by duration.
                \item \emph{Possession}: each must have a right to possession 
                of the whole. After the joint tenancy is created, one joint 
                tenant can voluntarily give exclusive possession to another.
            \end{enumerate}
            \item If any unity is lacking, a tenancy in common is created.
            \item If any unities are later severed, the joint tenancy becomes 
            a tenancy in common. Any one joint tenant can unilaterally convert 
            a joint tenancy into a tenancy in common by conveying his interest 
            to a third party. 
        \end{enumerate}
        \item \textbf{Tenancy by the entirety}: created only in husband and 
        wife.\footnote{Casebook p. 321.}
        \begin{enumerate}
            \item Requires the four unities, plus a fifth---marriage.
            \item Today it exists in less than half the states.
        \end{enumerate}
    \end{enumerate}
    \item Common law favored joint tenancies over tenancies in common. Today, 
    the situation is reversed.
    \item Historically, A and B could not hold joint tenancy if they occupied 
    unequal shares. This rule is mostly ignored today.\footnote{Casebook p. 
    324.}
\end{enumerate}

\paragraph{Problems on Creation of Joint Tenancies}

\begin{enumerate}
    \item O conveys property to A, B, and C as joint tenants. Subsequently A 
    conveys his interest to D. Then B dies intestate, leaving H as his 
    heir.\footnote{Casebook p. 322 problem 1.}
    \begin{enumerate}
        \item What is the state of the title? % TODO
        \item What if B had died leaving a will devising his interest to H?
        % TODO
    \end{enumerate}
    \item % TODO assignment sheet 6 feb 15
\end{enumerate}

\subsubsection{Severance of Joint Tenancies}

\paragraph{Problems on Severance of Joint Tenancies}

\begin{enumerate}
    \item Is unilateral severance fair? What sort of notice should be 
    required?
    \item % TODO prob 3 p. 334
\end{enumerate}

\paragraph{Unilateral Termination of Joint Tenancy: \emph{Riddle v. Harmon}}
~\\\\
Joint tenants can unilaterally sever the joint tenancy without resorting to 
common law rituals.

\begin{enumerate}
    \item Mrs. Riddle owned property in joint tenancy with her husband. She 
    tried to sever the joint tenancy by conveying to herself a one-half 
    interest.
    \item The trial court refused to recognize the severance. It quieted title 
    to the property in her husband.
    \item ``An indisputable right of each joint tenant is the power to convey 
    his or her separate estate by way of gift or otherwise without the 
    knowledge or consent of the other joint tenant and to thereby terminate 
    the joint tenancy.''\footnote{Casebook p. 325.}
    \item Some common law jurisdictions required the joint tenant to convey 
    his interest to a third party ``strawman'' (e.g., a lawyer), who would 
    then convey it back.
    \item Held: ``there is little virtue in steadfastly adhering to cumbersome 
    feudal law requirements.''\footnote{Casebook p. 327.} The severance was 
    valid.
\end{enumerate}

\paragraph{Mortgage and Severance: \emph{Harms v. Sprague}}
~\\\\
Mortages do not sever joint tenancies because they do not destroy unity of 
title.

\begin{enumerate}
    \item William and John Harms were joint tenants. During his life, John 
    Harms mortgaged his interest in the joint tenancy to Carl and Mary 
    Simmons. On his death, John Harms devised all of his property to Charles 
    Sprague.
    \item William Harms brought suit against Sprague to quiet title, asserting 
    his right of survivorship. He also named the Simmonses as defendants. 
    Sprague counterclaimed, claiming a right to the property as a co-tenant.
    \item The main issue was whether John Sprague's mortgage of his interest 
    severed the joint tenancy. A secondary issue was whether the mortgage 
    survived John Harms's death as a lien on the property.
    \item The trial court held that the mortgage severed the joint tenancy and 
    that it survived as a lien. The appellate court reversed, finding that 
    William Harms owned the entire property.
    \item The court's reasoning turned on whether the mortgage was a lien or 
    title. Precedent held that a lien would not sever a joint tenancy but a 
    title would. The court held that mortgages are liens. ``~.~.~.~we find 
    that a joint tenancy is not severed when one joint tenant executes a 
    mortgage on his interest in the property, since the unity of title is 
    preserved.''\footnote{Casebook p. 333.} Further, the mortgage did not 
    survive, because it rested on John Harms's interest, which expired at his 
    death.
\end{enumerate}

\subsubsection{Relations Among Concurrent Owners}

\paragraph{\emph{Delfino v. Vealencis}}

\begin{enumerate}
    \item % TODO % TODO 338
\end{enumerate}

\paragraph{\emph{Spiller v. Mackereth}}

\begin{enumerate}
    \item % TODO 348
\end{enumerate}

\paragraph{Co-Tenants' Right to Lease Their Shares: \emph{Swartzbaugh v. 
Sampson}}
~\\\\
A co-tenant (either a joint tenant or tenant in common)  can lease to another 
without the other co-tenants' consent, but the lessee's interest cannot exceed 
the lessor's interest as a co-tenant.

\begin{enumerate}
    \item John and Lola Swartzbaugh were joint tenants of sixty acres on 
    Orange County. In February 1934, John Swartzbaugh leased part of the 
    property to Sampson, over the objections of Lola Swartzbaugh. Sampson 
    gained exclusive possession of the property.
    \item In June 1934, Lola Swartzbaugh brought suit against John Swartzbaugh 
    and Sampson. The question before the court was, ``[c]an one joint tenant 
    who has not joined in the leases executed by her cotenant and another 
    maintain an action to cancel the leases where the lessee is in exclusive 
    possession of the leased property?''\footnote{Casebook p. 352.}
    \item In England, at least, a lease by one joint tenant destroys unity of 
    title and possession, thereby severing the joint tenancy. But the adoption 
    of this rule in the US ``seems doubtful.''\footnote{Casebook p. 352.}
    \item \emph{Stark v. Barrett}: ``conveyance by one tenant of a parcel of a 
    general tract, owned by several, is inoperative to impair any of the 
    rights of his cotenants.'' The conveyance to the grantee is valid, but it 
    does not supersede the interests of the other joint tenants---so, for 
    instance, if the land is partitioned and the grantee's tract is then no 
    longer controlled by the grantor, the grantee's interest 
    evaporates.\footnote{Casebook p. 353.}
    \item Thus, leases by co-tenants (both joint tenants and tenants in 
    common) are valid to the extent that the lessee's interests do not exceed 
    the lessor's as a joint tenant.\footnote{Casebook p. 354.}
    \item ``~.~.~.~the foregoing authorities force the conclusion that the 
    leases from Swartzbaugh to Sampson are not null and void but valid and 
    existing contracts giving to Sampson the same right to the possession of 
    the leased property that Swartzbaugh had. It follows that they cannot be 
    cancelled by plaintiff in this action.''\footnote{Casebook p. 354.}
    \item What could Mrs. Swartzbaugh's lawyer have done?
    \begin{enumerate}
        \item Actions against Sampson:
        \begin{enumerate}
            \item Seek partition---but that would allow Sampson to stay on the 
            land.
            \item Assert her right as a joints tenant and move in with him.
            \item Allow people onto the property---e.g., undermine his 
            business by letting people in for free.
            \item If he ousts her, she can win an injunction or damages.
        \end{enumerate}
        \item Actions against Mr. Swartzbaugh:
        \begin{enumerate}
            \item Prove that he was mentally incompetent, which would void the 
            lease to Sampson.
            \item Share in the rent under the Statute of Anne---but Sampson 
            only paid \$15/month, so she would only get \$7.50.
            \item Partition, either in kind or sale. It would terminate the 
            joint tenancy and destroy her right of survivorship. If it were 
            partitioned in kind and Sampson's portion happened to end up in 
            Mrs. Swartzbaugh's half, his lease would go poof---but courts are 
            unlikely to take that route.
        \end{enumerate}
    \end{enumerate}
    Lesson: pick your co-owners (and your spouses) with care.
\end{enumerate}

\paragraph{\emph{Baird v. Moore}}

\begin{enumerate}
    \item % TODO supp
    % occupying co-tenant is responsible for a larger share of maintenance 
    % expenses---though the Baird court did not apply this rule (the Mastbaum 
    % rule) -- see pp. 357--58
    % p. 357: do you agree with the statute of anne?
        % if co-owners don't have to pay rent to other co-owners, why do they 
        % have to divvy their rental income? and, if co-owners sell their 
        % interest, they don't have to distribute *that* income.
        % --- on the other hand, there's a visceral reaction to one owner 
        %  cashing in.
\end{enumerate}

\paragraph{Accounting for Benefits, Recovering Costs}

\begin{enumerate}
    \item If a co-tenant leases to a third party, he must distribute payments 
    received to the other co-tenants. But what if the co-tenant leases only 
    his share, rather than giving the lessee exclusive 
    occupancy?\footnote{Casebook p. 357, note 2.}
    ~\\\\\\\\ % TODO
    \item When repairs are necessary, some jurisdictions allow a co-tenant to 
    make the repairs and collect contributions from his co-tenants as long as 
    he gave notice. But in most jurisdictions, he has no right to contribution 
    unless there is an agreement between co-tenants, because the other 
    co-tenants have a right to participate in determining how much to spend, 
    etc. Given that justification, ``how is it that the cotenant receives a 
    credit for reasonable repairs in a partition or accounting 
    action~.~.~.~?'' \footnote{Casebook p. 358, note 4.}
    ~\\\\\\\\ % TODO
    \item  % TODO other quesetions p 358 n 4
    % TODO co-tenant --> cotenant
    \item % TODO \footnote{Casebook p. 347, note 7.}
        % they're tenants in common; it can't be partitioned in kind, so 
        % there'll be a partition sale; say they both attach equal value to 
        % it; so -- the one with more money will probably get it by paying 
        % more for it -- a ``wealth effect''

        % TODO: peterson said all [jointly] inherited property --> TiC. why?
        % todo: when should partition be allowed?
\end{enumerate}

\subsection{Marital Interests}

\begin{enumerate}
    \item Two systems of marital property emerged out of medieval Europe:
    \begin{enumerate}
        \item \emph{English}: husband and wife have separate ownership 
        interests.
        \item \emph{Continental}: community property. Husband and wife are 
        one, with indivisible interests.
    \end{enumerate}
    \item Most states adopted the English common law system. Ten states 
    adopted a community property system. During the twentieth century, the 
    trend was towards community property.
\end{enumerate}

\subsubsection{The Common Law Marital Property System}

\paragraph{During Marriage (The Fiction that Husband and Wife Are One)}

\begin{enumerate}
    \item At common law, a woman moved under \emph{cover} at marriage. Husband 
    and wife became one. ``~.~.~.~the husband had the right of possession to 
    all of the wife's lands during marriage, including land acquired after 
    marriage.''\footnote{Casebook p. 360.}
    \item Married Women's Property Acts removed the disabilities of coverture, 
    giving married women control of their property.
\end{enumerate}

\paragraph{Beyond the Reach of Creditors: \emph{Sawada v. Endo}}
~\\\\
An estate by the entireties is immune from the claims of each spouse's 
creditors.

\begin{enumerate}
    \item Facts:
    \begin{enumerate}
        \item November 30, 1968: Kokichi Endo injured Masako and Helen Sawada 
        in a car accident.
        \item Summer/fall 1969: the Sawadas sued Endo.
        \item July 26, 1969: Kokichi Endo and his wife conveyed their house, 
        which they owned as tenants by the entirety, to their sons. They no 
        longer had an interest in the property, but they continued to live 
        there.
        \item January 19, 1971: the Sawadas won damages, but Endo refused to 
        pay.  The Sawadas brought suit to set aside the conveyance of the Endo 
        home.
    \end{enumerate}
    \item The trial court refused to set aside the conveyance.
    \item The issue on appeal was ``whether the interest of one spouse in real 
    property, held in tenancy by the entireties, is subject to levy and 
    execution by his or her individual creditors.''\footnote{Casebook p. 362.}
    \item The court surveyed different states' treatment of tenancy by the 
    entireties, dividing them into four groups:
    \begin{enumerate}
        \item \emph{Group 1}: same as common law. The husband has exclusive 
        control.
        \item \emph{Group 2}: the interest of the debtor spouse can be 
        conveyed, subject to the other spouse's contingent right of 
        survivorship.
        \item \emph{Group 3}: following the Married Women's Property Acts, the 
        estate is not subject to the debts of either spouse. An attempted 
        conveyance by either spouse is void.
        \item \emph{Group 4}: each spouse's right of survivorship is alienable 
        and attachable by creditors.
    \end{enumerate}
    \item The court here decided to join group 3, holding that neither 
    spouse's interest in an estate by the entireties is subject to the claims 
    of creditors of either spouse. The reason is that husband and wife are 
    legally unifie. They have a single ownership interest in a tenancy by the 
    entirety. ``Neither husband nor wife has a separate divisible interest in 
    the property held by the entirety that can be conveyed or reached by 
    execution.''\footnote{Casebook p. 364.}
    \item This is not unfair to creditors because in extending credit, ``the 
    creditor presumably had notice of the characteristics of the estate which 
    limited his right to reach the property.''\footnote{Casebook p. 365.}
    \item Endo's conveyance to his sons was valid becuase it was not subject 
    to attachment by his creditors.
\end{enumerate}

\paragraph{Problems on Marital Property During Marriage}

\begin{enumerate}
    \item % TODO 366
\end{enumerate}

\paragraph{Homestead Rights}

\begin{enumerate}
    \item ``Most American states have laws designed to preserve the family 
    home (or homestead) from the claims of creditors.~.~.~.~These laws have 
    the purpose of securing shelter for the family and giving it some measure 
    of stability and independence; to achieve this purpose the rights of 
    creditors are sacrificed.''\footnote{Supplement p. 125.}
\end{enumerate}

\paragraph{The Policy of Exempting a Tenancy by the Entirety from Creditors}

\begin{enumerate}
    \item In most states, creditors of one spouse cannot reach a tenancy by 
    the entirety becuase one spouse cannot assign his or her interest.
    \item What policies does this rule serve?
    \begin{enumerate}
        \item The rule protects the interests of one spouse from the mistakes 
        of another. It wouldn't be fair to hold one spouse liable for 
        the other's costly errors.
    \end{enumerate}
    \item Is it fair to creditors?
    \begin{enumerate}
        \item The court in \emph{Sawada} argued that the rule is not unfair to 
        creditors because creditors should have known the status of the estate 
        before extending credit. This policy makes sense when the creditor 
        voluntary extends credit. In \emph{Sawada}, however, the Sawadas 
        inadvertently became Endo's creditors because he injured them in a car 
        accident and they won tort damages against him. His home appears to 
        have been his only significant asset, and the Sawadas did not have the 
        opportunity to assess his creditworthiness before ecoming his 
        creditors. In this case, the fact that Endo's estate was unreachable 
        left the Sawadas with no compensation for their injuries.
    \end{enumerate}
\end{enumerate}

\paragraph{Termination of Marriage by Death of One Spouse}

\begin{enumerate}
    \item Common law:
    \begin{enumerate}
        \item \emph{Dower}: a surviving widow takes one-third of the all of 
        the husband's freehold land.\footnote{Casebook p. 385.}
        \item \emph{Curtesy}: a widower is entitled to a life estate in his 
        wife's real property.
        \item Dower and curtesy exists in only four states, and in three of 
        them curtesy has been abolished and dower has been extended to 
        husbands.
    \end{enumerate}
    \item The modern elective share:
    \begin{enumerate}
        \item Forced share legislation gave the surviving spouse an elective 
        share in all proprerty that the decedent spouse owned at death. 
        Conventionally, the surviving spouse can renounce the will and elect 
        to take a statutory share, usually one-half or one-third.
        \item The Uniform Probate Code allows the spouse to keeo the property 
        that the will devised to him or her.
    \end{enumerate}
\end{enumerate}

\paragraph{The Uniform Probate Code on Elective Forced Share}

\begin{enumerate}
    \item The typical forced share statute ignored the duration of the 
    marriage when allocating property upon the death of a 
    spouse.\footnote{Reader p. 125.}
    \item Community property, by contrast, rewards are proportional to the 
    length of the marriage.
    \item 1990 amendments to the Uniform Probate Code brought it closer to the 
    community property system.
    \item \S\ 2-201 created a sliding scale from 3 percent for one-year 
    marriages to 50 percent for 15-year marriages.
    \item \S\ 2-202 defined the ``augmented estate,'' which 
    ``basically~.~.~.~totals up to the couple's combines 
    assets~.~.~.~''\footnote{Reader p. 125.}
    \item \S\ 2-207 ``provides that the surviving spouse's share is 
    collectible out of probate assets first and, if insufficient, then out of 
    nonprobate assets.''
\end{enumerate}

\paragraph{Problems on Termination of Marriage by Death of One Spouse}

\begin{enumerate}
    \item During O's marriage to W, O conveys property to A and B as joint 
    tenants.\footnote{Casebook p. 386 problem 1.}
    \begin{enumerate}
        \item O dies. Is O's widow, W, entitled to dower?
        \begin{enumerate}
            \item Yes, but not in the property O conveyed to A and B.
        \end{enumerate}
        \item Suppose that A dies survived by his wife, X. Does X have a right 
        to dower in the property?
        \begin{enumerate}
            \item Yes. Dower covers all of the husband's freehold land.
        \end{enumerate}
        \item A conveys his interest in the property to C. A dies survived by 
        his wife, X, who did not join in the deed to C. Is X entitled to dower 
        in the property? If C dies, is C's widow, Y, entitled to dower?
        \begin{enumerate}
            \item % TODO
        \end{enumerate}
    \end{enumerate}
    \item % TODO: problem re the elective forced share--see syllabus 2/25/2013
\end{enumerate}

\paragraph{Termination of Marriage by Divorce}

\begin{enumerate}
    \item % TODO 369
\end{enumerate}

\paragraph{\emph{In re Marriage of Graham}}

\begin{enumerate}
    \item % TODO 371
\end{enumerate}

\paragraph{\emph{Elkus v. Elkus}}

\begin{enumerate}
    \item % TODO 376
\end{enumerate}

\paragraph{J. Thomas Oldham, ``Putting Asunder in the 1990s''}

\begin{enumerate}
    \item % TODO 383
\end{enumerate}

\subsubsection{The Community Property System}

% TODO 387 ff.

\paragraph{Community Property Compared with Common Law Concurrent Interests}

\begin{enumerate}
    \item % TODO 389
\end{enumerate}

\paragraph{Management of Community Property}

\begin{enumerate}
    \item % TODO 391
\end{enumerate}

\paragraph{Mixing Community Property with Separate Property Problems}

\begin{enumerate}
    \item % TODO 393
\end{enumerate}

\paragraph{Migrating Couples}

\begin{enumerate}
    \item % TODO 393
\end{enumerate}
