\section{Co-Ownership and Marital Interests}

\begin{enumerate}
    \item Focus: \emph{concurrent} interests in present or future possession.
\end{enumerate}

\subsection{Common Law Concurrent Interests}

\subsubsection{Types, Characteristics, and Creation}

\begin{enumerate}
    \item Three main types:
    \begin{enumerate}
        \item \textbf{Tenants in common}: separate but undivided 
        interests.\footnote{Casebook p. 319.}
        \begin{enumerate}
            \item Example: O conveys property to A and B.
            \item The interest of each is descendable and may be conveyed by 
            deed or will.
            \item No survivorship rights---i.e., the interests are distinct 
            even if one owner dies.
            \item Each tenant owns an undivided share of the whole.
        \end{enumerate}
        \item \textbf{Joint tenants}: regarded as a single 
        owner.\footnote{Casebook p. 320.}
        \begin{enumerate}
            \item Both have a right of survivorship---so if one dies, that 
            tenant's interest is simply extinguished.\
            \item Four unities are required:
            \begin{enumerate}
                \item \emph{Time}: all interests must be acquired or vest 
                simultaneously.
                \item \emph{Title}: all interests must be acquired by the same 
                instrument or by joint adverse possession.
                \item \emph{Interest}: all must have equal undivided shares 
                and identical interests measured by duration.
                \item \emph{Possession}: each must have a right to possession 
                of the whole. After the joint tenancy is created, one joint 
                tenant can voluntarily give exclusive possession to another.
            \end{enumerate}
            \item If any unity is lacking, a tenancy in common is created.
            \item If any unities are later severed, the joint tenancy becomes 
            a tenancy in common. Any one joint tenant can unilaterally convert 
            a joint tenancy into a tenancy in common by conveying his interest 
            to a third party. 
        \end{enumerate}
        \item \textbf{Tenancy by the entirety}: created only in husband and 
        wife.\footnote{Casebook p. 321.}
        \begin{enumerate}
            \item Requires the four unities, plus a fifth---marriage.
            \item Today it exists in less than half the states.
        \end{enumerate}
    \end{enumerate}
    \item Common law favored joint tenancies over tenancies in common. Today, 
    the situation is reversed.
    \item Historically, A and B could not hold joint tenancy if they occupied 
    unequal shares. This rule is mostly ignored today.\footnote{Casebook p. 
    324.}
\end{enumerate}

\paragraph{Problems on Creation of Joint Tenancies}

\begin{enumerate}
    \item O conveys property to A, B, and C as joint tenants. Subsequently A 
    conveys his interest to D. Then B dies intestate, leaving H as his 
    heir.\footnote{Casebook p. 322 problem 1.}
    \begin{enumerate}
        \item What is the state of the title? % TODO
        \item What if B had died leaving a will devising his interest to H?
        % TODO
    \end{enumerate}
    \item % TODO assignment sheet 6 feb 15
\end{enumerate}

\subsubsection{Severance of Joint Tenancies}

% TODO 324-335

\paragraph{Problems on Severance of Joint Tenancies}

\begin{enumerate}
    \item % TODO prob 3 p. 334
\end{enumerate}


\paragraph{\emph{Riddle v. Harmon}}

\begin{enumerate}
    \item % TODO 324
\end{enumerate}

\paragraph{\emph{Harms v. Sprague}}

\begin{enumerate}
    \item % TODO 330
\end{enumerate}

\subsubsection{Relations Among Concurrent Owners}

\paragraph{\emph{Delfino v. Vealencis}}

\begin{enumerate}
    \item % TODO % TODO 338
\end{enumerate}

\paragraph{\emph{Spiller v. Mackereth}}

\begin{enumerate}
    \item % TODO 348
\end{enumerate}

\paragraph{Co-Tenants' Right to Lease Their Shares: \emph{Swartzbaugh v. 
Sampson}}
~\\\\
A co-tenant (either a joint tenant or tenant in common)  can lease to another 
without the other co-tenants' consent, but the lessee's interest cannot exceed 
the lessor's interest as a co-tentant.

\begin{enumerate}
    \item John and Lola Swartzbaugh were joint tenants of sixty acres on 
    Orange County. In February 1934, John Swartzbaugh leased part of the 
    property to Sampson, over the objections of Lola Swartzbaugh. Sampson 
    gained exclusive possession of the property.
    \item In June 1934, Lola Swartzbaugh brought suit against John Swartzbaugh 
    and Sampson. The question before the court was, ``[c]an one joint tenant 
    who has not joined in the leases executed by her cotenant and another 
    maintain an action to cancel the leases where the lessee is in exclusive 
    possession of the leased property?''\footnote{Casebook p. 352.}
    \item In England, at least, a lease by one joint tenant destroys unity of 
    title and possession, thereby severing the joint tenancy. But the adoption 
    of this rule in the US ``seems doubtful.''\footnote{Casebook p. 352.}
    \item \emph{Stark v. Barrett}: ``conveyance by one tenant of a parcel of a 
    general tract, owned by several, is inoperative to impair any of the 
    rights of his cotenants.'' The conveyance to the grantee is valid, but it 
    does not supersede the interests of the other joint tenants---so, for 
    instance, if the land is partitioned and the grantee's tract is then no 
    longer controlled by the grantor, the grantee's interest 
    evaporates.\footnote{Casebook p. 353.}
    \item Thus, leases by co-tenants (both joint tenants and tenants in 
    common) are valid to the extent that the lessee's interests do not exceed 
    the lessor's as a joint tenant.\footnote{Casebook p. 354.}
    \item ``~.~.~.~the foregoing authorities force the conclusion that the 
    leases from Swartzbaugh to Sampson are not null and void but valid and 
    existing contracts giving to Sampson the same right to the possession of 
    the leased property that Swartzbaugh had. It follows that they cannot be 
    cancelled by plaintiff in this action.''\footnote{Casebook p. 354.}
\end{enumerate}

\paragraph{Accounting for Benefits, Recovering Costs}

\begin{enumerate}
    \item % TODO 356
\end{enumerate}

\paragraph{\emph{Baird v. Moore}}

\begin{enumerate}
    \item % TODO supp
\end{enumerate}

\paragraph{Problems on Relations Among Concurrent Owners}

\begin{enumerate}
    \item If a co-tenant leases to a third party, he must distribute payments 
    received to the other co-tenants. But what if the co-tenant leases only 
    his share, rather than giving the lessee exclusive 
    occupancy?\footnote{Casebook p. 357, note 2.}
    ~\\\\\\\\ % TODO
    \item When repairs are necessary, some jurisdictions allow a co-tenant to 
    make the repairs and collect contributions from his co-tenants as long as 
    he gave notice. But in most jurisdictions, he has no right to contribution 
    unless there is an agreement between co-tenants, because the other 
    co-tenants have a right to participate in determining how much to spend, 
    etc. Given that justification, ``how is it that the cotenant receives a 
    credit for reasonable repairs in a partition or accounting 
    action~.~.~.~?'' \footnote{Casebook p. 358, note 4.}
    ~\\\\\\\\ % TODO
    \item  % TODO other quesetions p 358 n 4
    % TODO co-tenant --> cotenant
    \item % TODO \footnote{Casebook p. 347, note 7.}
\end{enumerate}





%%%%%%%%%%%%%%%%%%%%%

\subsubsection{Marital Interests}

\begin{enumerate}
    \item Two systems of marital property emerged out of medieval Europe:
    \begin{enumerate}
        \item \emph{English}: husband and wife have separate ownership 
        interests.
        \item \emph{Continental}: community property. Husband and wife are 
        one, with indivisible interests.
    \end{enumerate}
    \item Most states adopted the English common law system. Ten states 
    adopted a community property system. During the twentieth century, the 
    trend was towards community property.
\end{enumerate}

\paragraph{During Marriage (The Fiction that Husband and Wife Are One)}

\begin{enumerate}
    \item At common law, a woman moved under \emph{cover} at marriage. Husband 
    and wife became one. ``~.~.~.~the husband had the right of possession to 
    all of the wife's lands during marriage, including land acquired after 
    marriage.''\footnote{Casebook p. 360.}
    \item Married Women's Property Acts removed the disabilities of coverture, 
    giving married women control of their property.
\end{enumerate}

\paragraph{Beyond the Reach of Creditors: \emph{Sawada v. Endo}}
~\\\\
An estate by the entireties is immune from the claims of each spouse's 
creditors.

\begin{enumerate}
    \item Facts:
    \begin{enumerate}
        \item November 30, 1968: Kokichi Endo injured Masako and Helen Sawada 
        in a car accident.
        \item Summer/fall 1969: the Sawadas sued Endo.
        \item July 26, 1969: Kokichi Endo and his wife conveyed their house, 
        which they owned as tenants by the entirety, to their sons. They no 
        longer had an interest in the property, but they continued to live 
        there.
        \item January 19, 1971: the Sawadas won damages, but Endo refused to 
        pay.  The Sawadas brought suit to set aside the conveyance of the Endo 
        home.
    \end{enumerate}
    \item The trial court refused to set aside the conveyance.
    \item The issue on appeal was ``whether the interest of one spouse in real 
    property, held in tenancy by the entireties, is subject to levy and 
    execution by his or her individual creditors.''\footnote{Casebook p. 362.}
    \item The court surveyed different states' treatment of tenancy by the 
    entireties, dividing them into four groups:
    \begin{enumerate}
        \item \emph{Group 1}: same as common law. The husband has exclusive 
        control.
        \item \emph{Group 2}: the interest of the debtor spouse can be 
        conveyed, subject to the other spouse's contingent right of 
        survivorship.
        \item \emph{Group 3}: following the Married Women's Property Acts, the 
        estate is not subject to the debts of either spouse. An attempted 
        conveyance by either spouse is void.
        \item \emph{Group 4}: each spouse's right of survivorship is alienable 
        and attachable by creditors.
    \end{enumerate}
    \item The court here decided to join group 3, holding that neither 
    spouse's interest in an estate by the entireties is subject to the claims 
    of creditors of either spouse. The reason is that husband and wife are 
    legally unifie. They have a single ownership interest in a tenancy by the 
    entirety. ``Neither husband nor wife has a separate divisible interest in 
    the property held by the entirety that can be conveyed or reached by 
    execution.''\footnote{Casebook p. 364.}
    \item This is not unfair to creditors because in extending credit, ``the 
    creditor presumably had notice of the characteristics of the estate which 
    limited his right to reach the property.''\footnote{Casebook p. 365.}
    \item Endo's conveyance to his sons was valid becuase it was not subject 
    to attachment by his creditors.
\end{enumerate}
