\section{Co-Ownership and Marital Interests}

\subsection{Common Law Concurrent Interests}

\subsubsection{Types, Characteristics, and Creation}

\begin{enumerate}
    \item Three main types:
    \begin{enumerate}
        \item \textbf{Tenants in common}: separate but undivided 
        interests.\footnote{Casebook p. 319.}
        \begin{enumerate}
            \item Example: O conveys property to A and B.
            \item The interest of each is descendable and may be conveyed by 
            deed or will.
            \item No survivorship rights---i.e., the interests are distinct 
            even if one owner dies.
            \item Each tenant owns an undivided share of the whole.
        \end{enumerate}
        \item \textbf{Joint tenants}: regarded as a single 
        owner.\footnote{Casebook p. 320.}
        \begin{enumerate}
            \item Both have a right of survivorship---so if one dies, that 
            tenant's interest is simply extinguished.
            \item Four unities are required:
            \begin{enumerate}
                \item \emph{Time}: all interests must be acquired or vest 
                simultaneously.
                \item \emph{Title}: all interests must be acquired by the same 
                instrument or by joint adverse possession.
                \item \emph{Interest}: all must have equal undivided shares 
                and identical interests measured by duration.
                \item \emph{Possession}: each must have a right to possession 
                of the whole. After the joint tenancy is created, one joint 
                tenant can voluntarily give exclusive possession to another.
            \end{enumerate}
            \item If any unity is lacking, a tenancy in common is created.
            \item If any unities are later severed, the joint tenancy becomes 
            a tenancy in common. Any one joint tenant can unilaterally convert 
            a joint tenancy into a tenancy in common by conveying his interest 
            to a third party. 
        \end{enumerate}
        \item \textbf{Tenancy by the entirety}: created only in husband and 
        wife.\footnote{Casebook p. 321.}
        \begin{enumerate}
            \item Requires the four unities, plus a fifth---marriage.
            \item Today it exists in less than half the states.
        \end{enumerate}
    \end{enumerate}
    \item Common law favored joint tenancies over tenancies in common. Today, 
    the situation is reversed.
    \item Historically, A and B could not hold joint tenancy if they occupied 
    unequal shares. This rule is mostly ignored today.\footnote{Casebook p. 
    324.}
\end{enumerate}

\paragraph{Problems on Creation of Joint Tenancies}

\begin{enumerate}
    \item O conveys property to A, B, and C as joint tenants. Subsequently A 
    conveys his interest to D. Then B dies intestate, leaving H as his 
    heir.\footnote{Casebook p. 322 problem 1.}
    % TODO: verify; google group `Severance of joint tenancy' and sprankling 
    % p. 144.
    \begin{enumerate}
        \item What is the state of the title?
        \begin{enumerate}
            \item When A conveyed his interest to D, he severed the joint 
            tenancy, creating a tenancy in common among B, C, and D. When B 
            died intestate, his interest passed to his heir, H. So C, D, and H 
            have a tenancy in common.
        \end{enumerate}
        \item What if B had died leaving a will devising his interest to H?
        \begin{enumerate}
            \item The result should be the same. A's conveyance severed the 
            joint tenancy, creating a tenancy in common among B, C, and D. B 
            then devised his interest to H, leaving a tenancy in common among 
            C, D, and H.
        \end{enumerate}
    \end{enumerate}
    \item A and B own Blackacre in joint tenancy. A conveys a 10-year term of 
    years in Blackacre to C. After five years, A dies, devising all of his 
    property to D. What are B's rights?\footnote{Syllabus 2/15/2013.}
    % TODO verify; google group `Severance of joint tenancy #2'
    \begin{enumerate}
        \item Courts are divided on whether conveying a lease severs the joint 
        tenancy. In \emph{Tehnet v. Boswell}, the California Supreme Court 
        held that the lease is valid but does not sever the joint tenancy. In 
        this case, the \emph{Boswell} approach would mean that C's lease would 
        end upon A's death, and D would have nothing because A had no separate 
        interest to convey. Thus, B would have sole 
        possession.\footnote{\emph{Understanding Property} p. 144.}
        \item In other states, a lease does sever the joint tenancy. In this 
        case, after the lease, A and B would own Blackacre as tenants in 
        common. Upon A's death, A's interest passes to D, leaving B and C as 
        tenants in common. C's lease would remain, with D as the landlord.
    \end{enumerate}
\end{enumerate}

\subsubsection{Severance of Joint Tenancies}

\paragraph{Problems on Severance of Joint Tenancies}

\begin{enumerate}
    \item Is unilateral severance fair? What sort of notice should be 
    required?
    \begin{enumerate}
        \item Unilateral severance respects each co-owner's rights by allowing 
        each to control how his share is used. However, unilateral severance 
        creates opportunities for fraud. For instance, one tenant could 
        secretly sever the joint tenancy by conveying his interest to a third 
        party; but if the other joint tenant dies first, the survivor destroys 
        the secret deed and claims sole title.
        \item Therefore, joint tenants should be required to provide notice of 
        severance to the other joint tenants.
    \end{enumerate}
    \item Suppose that A and B (who are joint tenants) sign a written 
    agreement giving B the rentals from and possession of the land for her 
    life. Does the agreement destroy the unity of 
    possession?\footnote{Casebook p. 334 problem 3.}
    \begin{enumerate}
        \item ``~.~.~.~an agreement between joint tenants that merely provides 
        that one of them will occupy the common property does not effect a 
        severance.''\footnote{\emph{Understanding Property} p. 145.}
        \item A and B remain joint tenants. Upon A's death, B becomes the sole 
        owner.
    \end{enumerate}
\end{enumerate}

\paragraph{Unilateral Termination of Joint Tenancy: \emph{Riddle v. Harmon}}
~\\\\
Joint tenants can unilaterally sever the joint tenancy without resorting to 
common law rituals.

\begin{enumerate}
    \item Mrs. Riddle owned property in joint tenancy with her husband. She 
    tried to sever the joint tenancy by conveying to herself a one-half 
    interest.
    \item The trial court refused to recognize the severance. It quieted title 
    to the property in her husband.
    \item ``An indisputable right of each joint tenant is the power to convey 
    his or her separate estate by way of gift or otherwise without the 
    knowledge or consent of the other joint tenant and to thereby terminate 
    the joint tenancy.''\footnote{Casebook p. 325.}
    \item Some common law jurisdictions required the joint tenant to convey 
    his interest to a third party ``strawman'' (e.g., a lawyer), who would 
    then convey it back.
    \item Held: ``there is little virtue in steadfastly adhering to cumbersome 
    feudal law requirements.''\footnote{Casebook p. 327.} The severance was 
    valid.
\end{enumerate}

\paragraph{Mortgage and Severance: \emph{Harms v. Sprague}}
~\\\\
Mortages do not sever joint tenancies because they do not destroy unity of 
title.

\begin{enumerate}
    \item William and John Harms were joint tenants. During his life, John 
    Harms mortgaged his interest in the joint tenancy to Carl and Mary 
    Simmons. On his death, John Harms devised all of his property to Charles 
    Sprague.
    \item William Harms brought suit against Sprague to quiet title, asserting 
    his right of survivorship. He also named the Simmonses as defendants. 
    Sprague counterclaimed, claiming a right to the property as a cotenant.
    \item The main issue was whether John Sprague's mortgage of his interest 
    severed the joint tenancy. A secondary issue was whether the mortgage 
    survived John Harms's death as a lien on the property.
    \item The trial court held that the mortgage severed the joint tenancy and 
    that it survived as a lien. The appellate court reversed, finding that 
    William Harms owned the entire property.
    \item The court's reasoning turned on whether the mortgage was a lien or 
    title. Precedent held that a lien would not sever a joint tenancy but a 
    title would. The court held that mortgages are liens. ``~.~.~.~we find 
    that a joint tenancy is not severed when one joint tenant executes a 
    mortgage on his interest in the property, since the unity of title is 
    preserved.''\footnote{Casebook p. 333.} Further, the mortgage did not 
    survive, because it rested on John Harms's interest, which expired at his 
    death.
\end{enumerate}

\subsubsection{Relations Among Concurrent Owners}

\begin{enumerate}
    \item On the one hand, the law treats cotenants as independent actors. For 
    instance, one cotenant cannot enter into a contract on behalf of another.
    \item On the other hand, cotenants sometimes owe fiduciary duties to other 
    cotenants. But generally, cotenants do not have a duty to safeguard the 
    rights of other cotenants.
\end{enumerate}

\paragraph{Right to Possession}

\begin{enumerate}
    \item Each cotenant has an \textbf{equal right to possession of the whole 
    property}.
    \item The one major exception is \textbf{ouster}. Ousted cotenants can 
    recover their pro rata share of the fair rental value of the use by the 
    cotenant in possession. \emph{Spiller v. Mackereth}.
    \item A cotenant (either a joint tenant or tenant in common)  can lease to 
    another without the other cotenants' consent, but the lessee's interest 
    cannot exceed the lessor's interest as a cotenant.  \emph{Swartzbaugh v. 
    Sampson}.
    \item Some argue that it makes more sense today to require cotenants in 
    possession to pay rent to the other cotenants under some circumstances, 
    e.g., if the cotenants acquire the property by devise or intestate 
    succession and the other cotenants are already living somewhere 
    else.\footnote{\emph{Understanding Property} p. 140--41.}
\end{enumerate}

\paragraph{Right to Rents and Profits}

\begin{enumerate}
    \item If a third person pays to rent the land, \textbf{each cotenant is entitled 
    to a pro rata share of rents}.
    \begin{enumerate}
        \item The \textbf{Statute of Anne} provided that ``a tenant in common 
        actually receiving rents, issues and profits might be compelled to 
        account for the excess over his proper share.''\footnote{Reader p. 
        122.}
    \end{enumerate}
    \item If a cotenant refuses to pay, the other cotenants can bring an 
    \textbf{accounting action} to recover their shares.
    \item Cotenants are entitled to pro rata shares of profits from natural 
    resources.
\end{enumerate}

\paragraph{Liability for Mortgage and Tax Payments}

\begin{enumerate}
    \item \textbf{All cotenants are obliged} to pay their share of mortages, 
    taxes, assessments, and other payments that could give rise to a lien on 
    the property.
    \item If one tenant pays more than his share, he can recover the excess in 
    a \textbf{contribution action}.
    \begin{enumerate}
        \item Should the amount a cotenant can collect in contributions be 
        offset by the value of his use of the property? The court in 
        \emph{Baird v. Moore} said no. given the circumstances.
    \end{enumerate}
    \item However, in most states, a cotenant in sole possession cannot 
    recover for these payments unless they exceed the reasonable rental value. 
    So if the cotenant in possession spends \$20,000 per year on mortgage 
    payments, but the fair rental value for the year is \$30,000, the cotenant 
    cannot win contributions.
\end{enumerate}

\paragraph{Liability for Repair and Improvement Costs}

\begin{enumerate}
    \item \textbf{Cotenants cannot recover contributions for repairs or 
    improvements} because (1) cotenants may disagree on the scope and 
    necessity of the work and (2) if the law allowed contribution actions for 
    repairs or maintenance, courts would have to adjudicate minor disputes.
    \begin{enumerate}
        \item The \textbf{\emph{Mastbaum} rule}: ``A tenant in common who is 
        in sole possession of the common property is under a duty to his 
        co-tenants to preserve the property by making needful, ordinary 
        repairs, and paying taxes, mortgage interest and insurance 
        premiums.''\footnote{Reader p. 121.} But the general rule is that the  
        cotenant who made the repairs cannot collect contributions from other 
        cotenants.
        \item Some courts will allow contribution as justice requires. 
        \emph{Baird v. Moore}.
    \end{enumerate}
    \item Upon \textbf{partition or an accounting for rent}, a cotenant can 
    recover credit for the \textbf{excess costs} of repairs (i.e., costs 
    beyond his share). For improvements, 
    courts will try to give the improved portion of the property to the 
    cotenant to paid for it, and if that is not possible, it will award a 
    cotenant a credit for the \textbf{added property value} (but not for the 
    actual expense).
    \begin{enumerate}
        \item Example: A and B are cotenants. A builds a building at a cost of 
        \$10,000. Upon partition, the property is sold for \$55,000, with the 
        land worth \$30,000 and the building worth \$25,000. If they split it 
        down the middle, each would get \$27,500. But A should get the entire 
        value of the building. So, they each get half of the value of the land 
        (\$15,000), and A gets the entire value of the building in addition 
        (\$25,000). In total, A gets \$40,000 and B gets \$15,000.
        \item Courts favor partition in kind over a partition sale if it is 
        practical and serves the parties' interests. \emph{Delfino v.  
        Vealencis}.
    \end{enumerate}
\end{enumerate}

\paragraph{Liability for Waste}

\begin{enumerate}
    \item Cotenants are liable to other cotenants for waste.
    \item Courts treat natural resource profits as sources of income to be 
    divided among cotenants.
\end{enumerate}

\paragraph{Partition in Kind: \emph{Delfino v. Vealencis}}
~\\\\
Partition in kind should take priority over a partition sale if it is 
practical and serves the parties' interests.

\begin{enumerate}
    \item The plaintifs, Angelo and William Delfino, owned property as tenants 
    in common with the defendant, Helen Vealencis. The Delfinos sought a 
    partition sale.
    \item The trial court held that a partition in kind would cause ``material 
    injury'' and ordered a sale.
    \item The Supreme Court of Connecticut held that a partition sale should 
    happen only if partition in kind is unworkable.
    \item In this case, ``a partition in kind would clearly be 
    practicable.~.~.~.''\footnote{Casebook p. 341.}
    \item The trial court also held that a partition in kind would prejudice 
    the parties. The Delfinos argued that Vealencis's garbage business would 
    hurt the prospects of using the land for residential development. The 
    court here disagreed.
    \item Reversed.
\end{enumerate}

\paragraph{Problem on Partition in Kind}

\begin{enumerate}
    \item A and B are heirs of their father, who owned one item both A and B 
    very much want---his old rocking chair. They cannot agree who is to have 
    the chair. A brings a partition action. What relief should the court 
    award?\footnote{Casebook p. 347 note 7.}
    \begin{enumerate}
        \item A and B own the chair as tenants in common. The chair cannot be 
        partitioned in kind, so there must be a partition sale. Both attach 
        equal value to the chair, so the one with more money will probably get 
        it by paying more for it.
    \end{enumerate}
\end{enumerate}

\paragraph{Ouster: \emph{Spiller v. Mackereth}}
~\\\\Cotenants are not liable for rent to other cotenants unless they deny the 
other cotenants the right to enter.

\begin{enumerate}
    \item John Spiller and Hettie Mackereth owned property as tenants in 
    common. Spiller began using the structure as a warehouse. Mackereth 
    demanded that Spiller vacate half of the warehouse or pay half of the 
    rental value. Spiller refused.
    \item The trial court awarded the rental value to Mackereth.
    \item The general rule is that ``in absence of an agreement to pay rent or 
    an ouster of a cotenant, a cotenant in possession is not liable to his 
    cotenants for the value of his use and occupation of the 
    property.''\footnote{Casebook p. 348.}
    \item Here, there was no prior agreement to pay rent. Was there also 
    ouster? The plaintiff argued that the defendant's refusal to pay after 
    receiving the letter demanding rent counted as ouster. The court followed 
    the majority rule in holding that a cotenant cannot be liable for rent 
    until he denies his cotenants the right to enter.
    \item There was no ouster. Reversed.
\end{enumerate}

\paragraph{Cotenants' Right to Lease Their Shares: \emph{Swartzbaugh v. 
Sampson}}
~\\\\
A cotenant (either a joint tenant or tenant in common)  can lease to another 
without the other cotenants' consent, but the lessee's interest cannot exceed 
the lessor's interest as a cotenant.

\begin{enumerate}
    \item John and Lola Swartzbaugh were joint tenants of sixty acres on 
    Orange County. In February 1934, John Swartzbaugh leased part of the 
    property to Sampson, over the objections of Lola Swartzbaugh. Sampson 
    gained exclusive possession of the property.
    \item In June 1934, Lola Swartzbaugh brought suit against John Swartzbaugh 
    and Sampson. The question before the court was, ``[c]an one joint tenant 
    who has not joined in the leases executed by her cotenant and another 
    maintain an action to cancel the leases where the lessee is in exclusive 
    possession of the leased property?''\footnote{Casebook p. 352.}
    \item In England, at least, a lease by one joint tenant destroys unity of 
    title and possession, thereby severing the joint tenancy. But the adoption 
    of this rule in the US ``seems doubtful.''\footnote{Casebook p. 352.}
    \item \emph{Stark v. Barrett}: ``conveyance by one tenant of a parcel of a 
    general tract, owned by several, is inoperative to impair any of the 
    rights of his cotenants.'' The conveyance to the grantee is valid, but it 
    does not supersede the interests of the other joint tenants---so, for 
    instance, if the land is partitioned and the grantee's tract is then no 
    longer controlled by the grantor, the grantee's interest 
    evaporates.\footnote{Casebook p. 353.}
    \item Thus, leases by cotenants (both joint tenants and tenants in 
    common) are valid to the extent that the lessee's interests do not exceed 
    the lessor's as a joint tenant.\footnote{Casebook p. 354.}
    \item ``~.~.~.~the foregoing authorities force the conclusion that the 
    leases from Swartzbaugh to Sampson are not null and void but valid and 
    existing contracts giving to Sampson the same right to the possession of 
    the leased property that Swartzbaugh had. It follows that they cannot be 
    cancelled by plaintiff in this action.''\footnote{Casebook p. 354.}
    \item What could Mrs. Swartzbaugh's lawyer have done?
    \begin{enumerate}
        \item Actions against Sampson:
        \begin{enumerate}
            \item Seek partition---but that would allow Sampson to stay on the 
            land.
            \item Assert her right as a joints tenant and move in with him.
            \item Allow people onto the property---e.g., undermine his 
            business by letting people in for free.
            \item If he ousts her, she can win an injunction or damages.
        \end{enumerate}
        \item Actions against Mr. Swartzbaugh:
        \begin{enumerate}
            \item Prove that he was mentally incompetent, which would void the 
            lease to Sampson.
            \item Share in the rent under the Statute of Anne---but Sampson 
            only paid \$15/month, so she would only get \$7.50.
            \item Partition, either in kind or sale. It would terminate the 
            joint tenancy and destroy her right of survivorship. If it were 
            partitioned in kind and Sampson's portion happened to end up in 
            Mrs. Swartzbaugh's half, his lease would go poof---but courts are 
            unlikely to take that route.
        \end{enumerate}
    \end{enumerate}
    \item Lesson: pick your co-owners (and your spouses) with care.
\end{enumerate}

\paragraph{Contribution for Maintenance: \emph{Baird v. Moore}}
~\\\\When a cotenant in sole possession spends money to maintain the 
property---repairs, taxes, mortgage payments, and so on---can she recover 
contributions from the other cotenants? Many courts have said no, but this 
court said yes.

\begin{enumerate}
    \item The plaintiff owned a tenancy in common with her brother, 
    Herbert---an ``impecunious young man.''\footnote{Reader p. 121.} Herbert 
    died. The defendant was Herbert's wife, who became the administratrix of 
    his estate.
    \item The plaintiff had maintained the property and paid off a mortgage. 
    (She had been taking care of their mother and maintaining the house while 
    Herbert cavorted.) She sued for contribution from Herbert's estate. 
    The defendant argued that Herbert should not be liable for maintenance 
    costs after he left, because at that point plaintiff derived value as the 
    sole cotenant in possession.
    \item The trial court denied contribution for the plaintiff's expenses 
    after the point when she became the sole cotenant in possession.
    \item The trial court followed the \textbf{\emph{Mastbaum} rule}: ``A 
    tenant in common who is in sole possession of the common property is under 
    a duty to his co-tenants to preserve the property by making needful, 
    ordinary repairs, and paying taxes, mortgage interest and insurance 
    premiums.''\footnote{Reader p. 121.} It read the rule to mean that a 
    cotenant in sole possession could not recover contributions from other 
    cotenants. The court here reversed: ``the mere fact of possession by the 
    cotenant making advances for the benefit of the common estate should not 
    preclude reimbursement by contribution from the cotenants sharing in the 
    benefits by preservation of the common property.''\footnote{Reader p.  
    122.}
    \item Should the cotenant in possession's recovery be offset by the value 
    of his use and occupation?
    \begin{enumerate}
        \item The \textbf{Statute of Anne} provided that ``a tenant in common 
        actually receiving rents, issues and profits might be compelled to 
        account for the excess over his proper share.''\footnote{Reader p. 
        122.} The rule traditionally applied narrowly to scenarios where 
        the cotenant in possession actually collected rent from third parties.  
        \item A cotenant need not pay other cotenants to occupy the property. 
        But when the cotenant in possession seeks to recover contribution for 
        maintaining the property, ``many courts deemed it equitable that the 
        occupying tenant give credit for the value of his use and 
        occupation.''\footnote{Reader p. 123.}
        \item But the setoff rule is not absolute. First, the cotenant in 
        possession has not ousted the others. Second, why should you have to 
        pay rent for property you own? Third, payments (repairs, taxes, etc.) 
        benefit all cotenants.
        \item So, does \emph{Mastbaum} allow the plaintiff (the cotenant in 
        possession) to recover for expenses, or should there be a setoff for 
        the value of her use of the property?
        \item The circumstances of this case would make it ``grievously 
        inequitable to require plaintiff to offset against the just and 
        ratable contribution by defendant toward the expenses and maintenance 
        of the property after [his departure]~.~.~.~''\footnote{Reader p. 124.}
    \end{enumerate}
\end{enumerate}

\paragraph{Problems on Accounting for Benefits and Recovering Costs}

\begin{enumerate}
    \item If a cotenant leases to a third party, the Statute of Anne requires 
    him to distribute payments received to the other cotenants in proportion 
    to their shares. But what if the cotenant leases only his share, rather 
    than giving the lessee exclusive occupancy?\footnote{Casebook p. 357, note 
    2.}
    \begin{enumerate}
        \item Under the Statute of Anne, the leasing co-tenant is not required 
        to compensate the other co-tenants for more than his share. So if he 
        only rents his share, he is not required to compensate his co-tenants.
        % TODO: q for peterson
    \end{enumerate}
    \item When repairs are necessary, some jurisdictions allow a cotenant to 
    make the repairs and collect contributions from his cotenants as long as 
    he gave notice. But in most jurisdictions, he has no right to contribution 
    unless there is an agreement between cotenants, because the other 
    cotenants have a right to participate in determining how much to spend, 
    etc. Given that justification, ``how is it that the cotenant receives a 
    credit for reasonable repairs in a partition or accounting 
    action~.~.~.~?'' \footnote{Casebook p. 358, note 4.}
    \begin{enumerate}
        \item % TODO google group `Accounting to other co-tenants for repair'
              % see also q for peterson
    \end{enumerate}
\end{enumerate}

\subsection{Marital Interests}

\begin{enumerate}
    \item Two systems of marital property emerged out of medieval Europe:
    \begin{enumerate}
        \item \emph{English}: husband and wife have separate ownership 
        interests.
        \item \emph{Continental}: community property. Husband and wife are 
        one, with indivisible interests.
    \end{enumerate}
    \item Most states adopted the English common law system. Ten states 
    have adopted a community property system. During the twentieth century, 
    the trend was towards community property.
\end{enumerate}

\subsubsection{The Common Law Marital Property System}

\paragraph{During Marriage (The Fiction that Husband and Wife Are One)}

\begin{enumerate}
    \item At common law, a woman moved under \emph{cover} at marriage. Husband 
    and wife became one. ``~.~.~.~the husband had the right of possession to 
    all of the wife's lands during marriage, including land acquired after 
    marriage.''\footnote{Casebook p. 360.}
    \item Married Women's Property Acts removed the disabilities of coverture, 
    giving married women control of their property.
\end{enumerate}

\paragraph{Married Women's Property Acts}

\begin{enumerate}
    \item Typical acts provided that a property a women acquired before or 
    during marriage remained her own.\footnote{Reader p. 126.}
    \item MWPAs often fail to bridge gender gaps, however, because women are 
    still more likely to stay at home, or if they work, to earn significantly 
    less then men. Despite MWPAs, women are still more likely to have fewer 
    property interests than men.\footnote{\emph{Understanding Property} p. 
    151.}
\end{enumerate}

\paragraph{Effects of the MWPA on Tenancy by the Entirety: \emph{Sawada v. Endo}}
~\\\\
What effects do the Married Women's Property Act have on common law marital 
property rights?

Held: an estate held in tenancy by the entirety is immune from the claims of 
each spouse's separate creditors. So, you can protect your family home from 
creditors by holding it in a tenancy by the entirety.

% TODO: can you unilaterally sever a tenency by the entirety?
% see google group `Tenancy by the entirety'

\begin{enumerate}
    \item Facts:
    \begin{enumerate}
        \item November 30, 1968: Kokichi Endo injured Masako and Helen Sawada 
        in a car accident. Endo was uninsured.
        \item Summer/fall 1969: the Sawadas sued Endo.
        \item July 26, 1969: Kokichi Endo and his wife conveyed their house, 
        which they owned as tenants by the entirety, to their sons. They no 
        longer had an interest in the property after the conveyance, but they 
        continued to live there.
        \item January 19, 1971: the Sawadas won damages, but Endo refused to 
        pay. The Sawadas brought second suit to set aside the conveyance of 
        the Endo home as fraudulent.
    \end{enumerate}
    \item The trial court refused to set aside the conveyance.
    \item The issue on appeal was ``whether the interest of one spouse in real 
    property, held in tenancy by the entireties, is subject to levy and 
    execution by his or her individual creditors.''\footnote{Casebook p. 362.}
    \item The court surveyed different states' treatment of tenancy by the 
    entireties, dividing them into four groups:
    \begin{enumerate}
        \item \emph{Group 1}: same as common law. The husband has exclusive 
        control. In these states, the MWPA did not apply to tenancy by the 
        entirety. (These states have all since changed their approach.)
        \item \emph{Group 2}: the MWPA was meant to make the wife's rights 
        equal to her husband's. Therefore, the interest of the debtor spouse 
        can be conveyed, subject to the other spouse's contingent right of 
        survivorship. (There are other possible gender neutral 
        solutions---e.g., groups 3 and 4, below).
        \item \emph{Group 3}: the MWPA means that the estate is not subject to 
        the debts of either spouse. An attempted conveyance by either spouse 
        is void.
        \item \emph{Group 4}: the MWPA means that each spouse's right of 
        survivorship is alienable and attachable by creditors.
    \end{enumerate}
    \item The court here decided to join group 3, holding that neither 
    spouse's interest in an estate by the entireties is subject to the claims 
    of creditors of either spouse. The reason is that husband and wife are 
    legally unified. They have a single ownership interest in a tenancy by the 
    entirety. ``Neither husband nor wife has a separate divisible interest in 
    the property held by the entirety that can be conveyed or reached by 
    execution.''\footnote{Casebook p. 364.}
    \item This is not unfair to creditors because in extending credit, ``the 
    creditor presumably had notice of the characteristics of the estate which 
    limited his right to reach the property.''\footnote{Casebook p. 365.}
    \begin{enumerate}
        \item But: in this case, the ``creditors'' didn't know that they were 
        dealing with tenants in the entirety.
    \end{enumerate}
    \item Endo's conveyance to his sons was valid because it was not subject 
    to attachment by his creditors.
\end{enumerate}

\paragraph{Problems on Marital Property During Marriage}

\begin{enumerate}
    \item Should we get rid of tenancies by the entirety?
        \item There is no TiC in CA because it's a community property state.
    \item Since \emph{Sawado}, the last three adherents to classical tenancy 
    in common (MA, MI, NC), have ``enacted legislation to give equal rights to 
    husband and wife in a tenancy by the entirety.''\footnote{Casebook p. 367.}
    \item The exemption from creditors is probably the main reason why 
    tenancies by the entirety remain.
    \item Which creditors can reach the interest of a debtor spouse in a 
    tenancy by the entirety?
    \begin{enumerate}
        \item \emph{United States v. Craft}: the IRS can attach liens to 
        property held in tenancy by the entirety because federal law 
        determined the relevant property rights.
        \item \emph{Craft} carries implications for other federal 
        issues, e.g., bankruptcy, forfeiture after criminal 
        conviction.\footnote{Casebook pp. 368--369.}
    \end{enumerate}
\end{enumerate}

\paragraph{Homestead Rights}

\begin{enumerate}
    \item ``Most American states have laws designed to preserve the family 
    home (or homestead) from the claims of creditors.~.~.~.~These laws have 
    the purpose of securing shelter for the family and giving it some measure 
    of stability and independence; to achieve this purpose the rights of 
    creditors are sacrificed.''\footnote{Supplement p. 125.}
\end{enumerate}

\paragraph{The Policy of Exempting a Tenancy by the Entirety from Creditors}

\begin{enumerate}
    \item In most states, creditors of one spouse cannot reach a tenancy by 
    the entirety becuase one spouse cannot assign his or her interest.
    \item What policies does this rule serve?
    \begin{enumerate}
        \item The rule protects the interests of one spouse from the mistakes 
        of another. It wouldn't be fair to hold one spouse liable for 
        the other's costly errors.
    \end{enumerate}
    \item Is it fair to creditors?
    \begin{enumerate}
        \item The court in \emph{Sawada} argued that the rule is not unfair to 
        creditors because creditors should have known the status of the estate 
        before extending credit. This policy makes sense when the creditor 
        voluntary extends credit. In \emph{Sawada}, however, the Sawadas 
        inadvertently became Endo's creditors because he injured them in a car 
        accident and they won tort damages against him. His home appears to 
        have been his only significant asset, and the Sawadas did not have the 
        opportunity to assess his creditworthiness before ecoming his 
        creditors. In this case, the fact that Endo's estate was unreachable 
        left the Sawadas with no compensation for their injuries.
    \end{enumerate}
\end{enumerate}

\paragraph{Termination of Marriage by Death of One Spouse}

\begin{enumerate}
    \item Common law:
    \begin{enumerate}
        \item \emph{Dower} (the wife's right): a surviving widow takes a life 
        estate in one-third (in terms of value, not acreage) of the all of the 
        husband's freehold land.\footnote{Casebook p. 385.}
        \begin{enumerate}
            \item For dower to attach, issue born of the marriage must be able 
            to inherit the property. For instance, dower does not attach to 
            life estates or property held in joint tenancy.
        \end{enumerate}
        % TODO: what happens when the dower life estate ends? who has the 
        % future interest?
        % see google group `Dower'
        \item \emph{Curtesy} (the husband's right): a widower is entitled to a 
        life estate in his wife's real property held in freehold estate. 
        Unlike dower, the husband gets the right only if there is a child from 
        the marriage.
        \item Both have been successfully challenged as violating equal 
        protection. Dower and curtesy exists in only four states, and in three 
        of them curtesy has been abolished and dower has been extended to 
        husbands.
    \end{enumerate}
    \item The modern elective forced share:
    \begin{enumerate}
        \item Forced share legislation gave the surviving spouse an elective 
        share in all property that the decedent spouse owned at death. 
        The surviving spouse can renounce the will and elect to take a 
        statutory share, usually one-half or one-third.
        \begin{enumerate}
            \item The surviving spouse asks: what property was subject to the 
            husband's probate estate? Then: the surviving spouse can choose to 
            be entitled to a fractional share of the probate estate (usually 
            one-half or one-third).
            \item It usually comes up when the surviving spouse has been cut 
            out of the will.
            \item Spouses can remove property from the probate estate in 
            several ways---e.g., by holding it in joint tenancy with someone 
            else.
        \end{enumerate}
        \item An alternative to the traditional forced elective share 
        approach: the Uniform Probate Code. See below.
    \end{enumerate}
\end{enumerate}

\paragraph{Problems on Termination of Marriage by Death of One Spouse}

\begin{enumerate}
    \item During O's marriage to W, O conveys property to A and B as joint 
    tenants.\footnote{Casebook p. 386 problem 1.}
    \begin{enumerate}
        \item O dies. Is O's widow, W, entitled to dower?
        \begin{enumerate}
            \item Yes. W retains her dower right unless she consented to give 
            it up. She has a life estate in one third of the property, and A 
            and B's joint tenancy shrinks by one third.
        \end{enumerate}
        \item Suppose that A dies survived by his wife, X. Does X have a right 
        to dower in the property?
        \begin{enumerate}
            \item No. A and B own the property as joint tenants. Dower does 
            not attach to property held in joint tenancy. B would be the sole 
            owner, subject to W's life estate.
        \end{enumerate}
        \item A conveys his interest in the property to C. A dies survived by 
        his wife, X, who did not join in the deed to C. Is X entitled to dower 
        in the property? If C dies, is C's widow, Y, entitled to dower?
        \begin{enumerate}
            \item [y/n] By conveying his interest to C, A severed the joint 
            tenancy with B, so B and C became tenants in common.
            % TODO: google group, `Dower problem'
            % I think yes, because A, B, and C all had freehold estates as 
            % tenants in common
        \end{enumerate}
    \end{enumerate}
    \item H dies in a state that gives the surviving spouse a choice of dower 
    or an elective forced share of one-half of the decedent's property passing 
    by will or intestacy. During his life H took out a life insurance policy 
    in the face amount of \$60,000 payable to W. H and W also bought a house, 
    worth \$60,000 at H's death, and took title as joint tenants. H dies 
    owning Blackacre, worth \$90,000, stocks and bonds worth \$20,000, and a 
    \$10,000 savings account. H's will bequeaths all his estate to his 
    daughter by a first marriage, D.\footnote{Assignment sheet 8, 2/25/2013.}
    % TODO verify -- google group `Dower problem #2'
    \begin{enumerate}
        \item How is H's estate distributed?
        \begin{enumerate}
            \item The life insurance policy is not part of H's freehold 
            estate, so W gets the \$60,000. Similarly, the \$60,000 house was 
            not part of the probate estate, so W gets the house by right of 
            survivorship.
            \item The remainder includes Blackacre (\$90,000), \$20,000 in  
            investments, and a \$10,000 savings account.
            \begin{enumerate}
                \item If W chooses dower, she takes a life estate in one third 
                of H's freehold land, or \$45,000. She does not get the 
                investments or savings.
                \item If W chooses the elective forced share, she gets half of 
                H's probate estate, which includes all of the remainder. So 
                she would get (\$90,000 / 2) + (\$20,000 / 2) + (\$10,000 / 2) 
                = \$60,0000.
            \end{enumerate}
        \end{enumerate}
        \item If you were advising H before he died, how would you advise him 
        to carry out his wishes?
        \begin{enumerate}
            \item H should have conveyed Blackacre, his investments, and his 
            savings to his daughter before his death.
        \end{enumerate}
    \end{enumerate}
\end{enumerate}

\paragraph{Termination of Marriage by Divorce}

\begin{enumerate}
    \item At common law, upon divorce property remained with the spouse 
    holding the title. Property in TiC or JT remained so, and property held in 
    TbE was converted into a TiC because of the destruction of the unity of 
    marriage. The husband usually owed the wife alimony.
    \item Recent divorce law reforms (e.g., no fault) have led to 
    \textbf{equitable distribution}, in which the court divides property in a 
    way it thinks is fair.\footnote{Casebook p. 370.}
    % TODO: clarify three types of equitable distribution--p. 370`
    \item Equitable division statutes usually only allow courts to divide 
    \textbf{marital property}.
    \item Alimony has also largely disappeared, except where necessary for the 
    other spouse to become self-sufficient (``rehabilitative 
    alimony'').\footnote{Casebook p. 370.}
    \item Sometimes equal (not just equitable) division is required, but not 
    always.
\end{enumerate}

\paragraph{M.B.A. as Non-Marital Property: \emph{In re Marriage of Graham}}

\begin{enumerate}
    \item During a six year marriage, the wife supported the husband while he 
    earned a bachelor's degree and an M.B.A. After divorce, she claimed that 
    the M.B.A. was marital property and that she was entitled to a share of 
    its value.
    \item The trial court found that education can be marital property. It 
    awarded \$33,134 to the wife.
    \item The appellate court reversed, holding that an education is not 
    property, though it could be ``considered in determining maintenance or in 
    arriving at an equitable property division.''\footnote{Casebook p. 372.}
    \item The court here held that property is everything that has an exchange 
    value. Degrees have no exchange value, so they are not property. The court 
    could, in theory, take it into account when calculating alimony or the 
    division of marital property---but here, the wife sought neither, so there 
    was no way to account for the value of her contribution.
    \item Judge Carrigan, dissenting: the M.B.A. was the couple's most 
    valuable asset. The wife had made a significant investment in it. The 
    issue is not the degree itself, but the increased earning capacity it gave 
    the husband. It would be fair to compensate the wife for her contribution.
\end{enumerate}

\paragraph{Celebrity Status as Marital Property: \emph{Elkus v. Elkus}}

\begin{enumerate}
    \item The plaintiff became a famous opera singer during her marriage with 
    the defendant. The question was whether her celebrity status was marital 
    property.
    \item The trial court held that it was not marital property.
    \item The court here found that the husband had contributed substantially 
    to the wife's career (as photographer, voice coach, and more). New York 
    conceptualized marriage as an ``economic partnership.'' The plaintiff's 
    celebrity status increased her earning capacity, which earlier cases held 
    to be within the scope of marital property. The court here held that 
    celebrity status would have to be marital property to be consistent with 
    the ``economic partnership'' concept of marriage.
\end{enumerate}

\paragraph{J. Thomas Oldham, ``Putting Asunder in the 1990s''}

\begin{enumerate}
    \item New York developed a system in which ``human capital accumulations'' 
    (e.g., professional degrees) are counted as marital property, valued by 
    the increase in lifetime earning capacity. Upon divorce, the person with 
    the degree must compensate the other spouse for the value of the increased 
    earning capacity. But if the degree holder remarries, the earnings will 
    also be counted as marital property of the second marriage. The degree 
    holder's earnings would then be counted twice.
\end{enumerate}

\subsubsection{The Community Property System}

\begin{enumerate}
    \item Nine states (including CA) have community property systems, with 
    Alaska as an elective community property state.
    \item ``~.~.~.~the fundamental idea of community property is that 
    \emph{earnings} of each spouse during marriage should be owned equally in 
    undivided shares by both spouses.''\footnote{Casebook p. 388.}
    \item All other property is separate---e.g., property acquired before 
    marriage, property acquired by gift, devise, or descent, and property 
    owned by one spouse by agreement of both spouses.
\end{enumerate}

\paragraph{Community Property Compared with Common Law Concurrent Interests}

\begin{enumerate}
    \item In community property states, there is no dower, curtesy, or tenancy 
    by the entirety.\footnote{Casebook p. 389.}
    \item Neither spouse can convey his or her undivided share, except to the 
    other spouse. Neither spouse can unilaterally convert community property 
    into separate property.
    \item Upon death, each spouse can dispose of half of the community 
    property by will. There is no right of survivorship. Usually (but not 
    always), if one spouse dies intestate, that spouse's share passes to the 
    surviving spouse.\footnote{Casebook p. 390.}
\end{enumerate}

\paragraph{Management of Community Property}

\begin{enumerate}
    \item Community property can only be conveyed to a third person as an 
    undivided whole. But which spouse has the authority to manage the 
    property?
    \item Today, all community property states give both spouses equal 
    managerial authority. Usually either spouse can manage the property, but 
    in certain cases only one can---for instance, if the title is only in one 
    spouse's name, or if one of the spouses is operating a business on the 
    property.
    \item If the husband and wife are equal managers, the creditors of either 
    can reach the community property.
\end{enumerate}

\paragraph{Mixing Community Property with Separate Property Problems}

\begin{enumerate}
    \item Community property states are not in agreement about what happens 
    when separate property is mixed with community property. For instance, 
    what happens when the wife buys property before marriage and pays 1/3 of 
    the installments, and both spouses pay the remaining 2/3 after marriage?
    \item There are several approaches:\footnote{Casebook p. 392.}
    \begin{enumerate}
        \item \emph{``Inception of right'' rule}: the house is the wife's 
        separate property because it was separate when she signed the 
        purchase contract.
        \item \emph{``Time of vesting'' rule}: the house is community property 
        because it vests once all the installments are paid (here, during 
        marriage).
        \item \emph{Pro rata rule (California)}: the community payments ``buy 
        in'' a pro rata share of the title.
    \end{enumerate}
\end{enumerate}

\paragraph{Migrating Couples}

\begin{enumerate}
    \item When couples move, their property retains its status as community 
    property or common law property.
\end{enumerate}
