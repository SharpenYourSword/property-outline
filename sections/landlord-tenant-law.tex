\section{Landlord-Tenant Law}

\subsection{Leasehold Estates}

\subsubsection{The Term of Years}

\begin{enumerate}
    \item Lasts for a fixed period of time.
    \item Some states set upper limits.
\end{enumerate}

\subsubsection{The Periodic Tenancy}

\begin{enumerate}
    \item Continues for succeeding periods until the tenant or landlord gives 
    notice.
    \item Common law rules required six months' notice to terminate a 
    year-to-year tenancy. For shorter periods, notice of termination must 
    equal the period, or six months---whichever is less (i.e., the notice 
    period cannot exceed six months).
\end{enumerate}

\subsubsection{The Tenancy at Will}

\begin{enumerate}
    \item No fixed period. It endures as long as the landlord and tenant want.
    \item Some modern statutes have notice requirements.
\end{enumerate}

\subsubsection{Problems on Leasehold Estates}

\begin{enumerate}
    \item On October 1, L leases Whiteacre ``to T for one year, beginning on 
    October 1.'' T moves out the following September 30 without giving notice. 
    What are L's rights?\footnote{Casebook p. 422 problem 1.}
    \begin{enumerate}
        \item T has a term of years. T has no obligation to give notice. L has 
        no remedy.
    \end{enumerate}
    \item What if the lease had been ``from year to year, beginning October 1''?
    \begin{enumerate}
        \item T has a periodic tenancy. Under the common law rules, T was 
        required to give notice of termination six months in advance. L should 
        be able to recover.
    \end{enumerate}
    \item What is the lease had been for no fixed term ``at an annual rental 
    of \$24,000, payable \$2,000 per month''?
    \begin{enumerate}
        \item T has a tenancy at will. T was under no obligation to give 
        notice, unless there was a state statutory notice requirement. L may 
        also argue that the ``annual rental'' provision established a term of 
        years or a periodic tenancy.
    \end{enumerate}
\end{enumerate}

\paragraph{``Lease for Life'': \emph{Garner v. Gerrish}}

\begin{enumerate}
    \item Gerrish rented from Donovan under a lease that allowed him to 
    terminate at any time. Donovan died a few years later. The executor of his 
    estate, Garner, tried to remove Gerrish.
    \item Garner argued that the lease created a tenancy at will because it 
    did not set a definite term. Gerrish argued that he had a tenancy for 
    life.
    \item The trial court granted summary judgment to Garner and the appellate 
    court reversed.
    \item The early common law was that the landlord and tenant must have had 
    equal rights. For instance, it wasn't allowable for the tenant to have a 
    right of termination but not the landlord. But, this rule has been mostly 
    discarded.
    \item Held: ``~.~.~.~the lease expressly and unambiguously grants to the 
    tenant the right to terminate, and does not reserve to the landlord a 
    similar right. To hold that such a lease creates a tenancy terminable at 
    the will of either party would violate the terms of the agreement and the 
    express intent of the contracting parties.''\footnote{Casebook p. 425.}
    \item The \emph{numerus clausus} principle forbids parties and courts from 
    developing new types of estates. Does the court here sanction the creation 
    of a new type of estate, the ``lease for life''?
\end{enumerate}

\subsubsection{The Tenancy at Sufferance: Holdovers}

\begin{enumerate}
    \item Arises when a tenant in possession remains in possession after 
    termination of the tenancy.
    \item At common law, landlords had two options: eviction or consent.
    \item What happens when a holdover tenant gives rent checks to the 
    landlord and the landlord cashes them?
    \begin{enumerate}
        \item Usually it creates a periodic tenancy or a term. The period can 
        be based on the period in the original lease or on how rent was 
        calculated in the original lease.\footnote{Casebook p. 427.}
    \end{enumerate}
    \item States have adopted varying legislation addressing holdover tenants. 
    Some specificy the length of the holdover tenancy. Others convert it to a 
    tenancy at will. Still others may demand double rent.
    \item The common law rule aimed to protect the rights of incoming tenants 
    and the landlord's right to predictable terms. On the other hand, 
    penalties imposed on holdover tenants often far exceed the injury to the 
    landlord or the incoming tenant.
\end{enumerate}

\paragraph{\emph{Crechale v. Polles, Inc. v. Smith}}

\begin{enumerate}
    \item % TODO reader
\end{enumerate}

\paragraph{Question on Holdover Tenants}

\begin{enumerate}
    \item % TODO reader march 11
\end{enumerate}

\subsection{The Lease}

\begin{enumerate}
    \item An instrument that resembles a lease might in fact be something 
    else, like a license or a life estate. The difference matters because 
    leases give specific rights to the landlord and tenant.
    \item A lease is both conveyance and contract. Courts often invoke 
    contract principles in addressing lease disputes.
\end{enumerate}

\subsection{Delivery of Possession}

\subsubsection{No Implied Covenant to Deliver Possession: \emph{Hannah v. 
Dusch}}

\begin{enumerate}
    \item Dutsch leased to Hannah for a 15-year term. Another tenant was 
    occupying the land when Hannah was supposed to move in.
    \item The question was whether the landlord is responsible for removing 
    trespassers and wrongdoers at the beginning of the lease term. Is there an 
    ``implied covenant to deliver possession''?\footnote{Casebook p. 439.}
    \item The ``English rule'': there is a covenant. The ``American rule'': no 
    covenant.
    \item The court here followed the American rule, holding that Hannah had a 
    remedy against the wrongdoer but not against the landlord.
\end{enumerate}

\subsubsection{Problems on Delivery of Possession}

\begin{enumerate}
    \item L and T execute a lease for a specified term. T takes possession and 
    pays rent for several months. T then learns that L had earlier leased the 
    premises to another tenant for the same term. T remains in possession but 
    stops paying rent. L sues T for unpaid rent; T counterclaims for rent 
    already paid. What result?\footnote{Casebook p. 442 problem 2.}
    \begin{enumerate}
        \item T has a term of years. As long as the other tenant did not 
        displace T, T was in possession and he should be bound by the terms of 
        the lease. L's other lease has not affected T's interests.
    \end{enumerate}
\end{enumerate}

\subsection{Subleases and Assignments}

\subsubsection{Assignment or Sublease: \emph{Ernst v. Conditt}}

\begin{enumerate}
    \item Ernst leased to Rogers for a term of one year and seven days. One of 
    the terms of the lease was that Rogers could not assign or sublet without 
    Ernst's permission.
    \item Rogers built a racetrack and other improvements. The lease 
    stipulated that at the end of the term, Rogers would remove all 
    above-ground improvements.
    \item A month after signing the lease, Rogers sold the racetrack business 
    to Conditt. Conditt and Rogers negotiated a lease extension with Ernst. 
    The modification granted subletting permissions to Conditt but held that 
    Rogers would remain liable for performance of the original terms of the 
    lease (i.e., removing improvements).
    \item Conditt stopped paying rent. Ernst demanded payment of past rent 
    from Conditt if Conditt did not remove the improvements from the property. 
    Conditt did not reply, so Ernst sued.
    \item Ernst argued that the agreement between Ernst, Rogers, and Conditt 
    \emph{assigned} the lease to Conditt, which meant that Conditt was 
    primarily liable to Ernst.
    \item Conditt argued that the agreement was a \emph{sublease}, and that 
    therefore Rogers was directly liable to Ernst.
    \item The trial court held that the agreement was an assignment and so 
    Conditt was directly liable to Ernst.
    \item The revised agreement involving Conditt specifically used the word 
    ``sublet.'' It held Rogers responsible for performane of the original 
    lease's terms. Finally, Rogers retained a reversion.
    \item The general rule is that an assignment conveys the entire term, 
    leaving no interest or reversion in the assignor. In this case, however, 
    Rogers retained an interest, so the conveyane was a sublease.
    \begin{enumerate}
        \item Common law rule: ``If the instrument purports to transfer the 
        lessee's estate for the entire remainder of his term it is an 
        assignment, regardless of its form or the parties' intention. 
        Conversely, if the instrument purports to transfer the lessee's estate 
        for less than the entire term---even for a day less---it is a 
        sublease, regardless of its form or of the parties' 
        intention.''\footnote{Casebook p. 446.}
        \item Modern rule: follow the parties' intention.
    \end{enumerate}
    \item Held: under both the common law rule and the modern rule, the 
    conveyance from Rogers to Conditt was a sublease, not an assignment. 
    Therefore, Conditt is not directly liable to Ernst. Reversed.
\end{enumerate}

\subsubsection{Problems on Subleases and Assignments}

\begin{enumerate}
    \item L leases to T for a term of three years at \$1,000/month. One year 
    later T ``subleases, transfers, and assigns'' to T1 for ``a period of one 
    year from date.'' Neither T nor T1 pays rent.\footnote{Casebook p. 448 
    problem 3(a).}
    \begin{enumerate}
        \item What rights does L have against T?
        \begin{enumerate}
            \item T retained an interest in the property, so his conveyance 
            created a sublease with T1, not an assignment. T is therefore 
            directly liable to L for rent.
        \end{enumerate}
        \item What rights does L have against T1?
        \begin{enumerate}
            \item L has no rights directly against T1.
        \end{enumerate}
    \end{enumerate}
    \item \footnote{Casebook p. 449 problem 3(b).}
    \begin{enumerate}
        \item L leases to T for three years at \$1,000/month, with a clause 
        that indicates that T agrees to pay rent before the first of each 
        month. It also prevents T from subletting or assigning without L's 
        permission. Six months later, T, with L's permission, transfers to T1 
        for the balance of the term. T1 pays rent directly to L for several 
        months and then defaults. L sues T for the rent due. What result?
        \begin{enumerate}
            \item T's conveyance to T1 was an assignment because T did not 
            retain an interest. (Under the modern rule, it might be considered 
            a sublease if T's intention was to convey a sublease rather than 
            an assignment.) After the assignment, T is no longer directly 
            liable to L for rent. L has to recover directly from T1.
        \end{enumerate}
    \end{enumerate}
    \item \footnote{Casebook p. 449 problem 3(c).}
    \begin{enumerate}
        \item L leases to T for three years at \$1,000/month with a promise to 
        pay rent in advance of the first of each month and to keep the 
        premises in good repair. Six month later T conveys her entire interest 
        to T1, who agrees to assume all covenants in the original lease 
        between L and T. Three months later T1 assigns his entire interest to 
        T2, and three months after that T2 assigns his entire interest to T3. 
        T3 defaults on rent payment and fails to keep the premises in good 
        repair. L sues T, T1, T2, and T3. What are the liabilities of the four 
        tenants to L and to each other?
        \begin{enumerate}
            \item [Does assignmetn of a lessee's entire interest convey that 
            lessee's covenants with the lessor? If yes, only T3 is liable. If 
            not, T is not liable, but T1, T2, and T3 all are.]
        \end{enumerate}
    \end{enumerate}
\end{enumerate}

\subsection{The Tenant Who Defaults}

\subsubsection{The Tenant in Possession}

\begin{enumerate}
    \item Suppose the tenant defaults or abandons. What can the landlord do?
\end{enumerate}

\paragraph{\emph{Berg v. Wiley}}

\begin{enumerate}
    \item Facts:
    \begin{enumerate}
        \item November 11, 1970: Wiley leased to Phillip Berg for 5 years on 
        the condition that he would not remodel the premises without 
        permission.
        \item Early 1971: Berg transferred his interest to his sister 
        Kathleen.
        \item May 1971: Kathleen opened ``A Family Affair Restaurant.''
        \item July 15, 1973: the date by which Wiley demanded remedies for
        health code violations and remodeling. Berg closed the restaurant for 
        remodeling, but it was unclear whether she intened to reopen it.
        \item July 16, 1973: Berg entered and changed the locks.
        torts.
        \item August 1, 1973: Wiley relet the property to another tenant.
        \item December 1, 1975: the lease was due to expire.
    \end{enumerate}
    \item Berg remodeled the restaurant without Wiley's permission and 
    operated the restaurant in violation of health codes.\footnote{Casebook 
    pp. 460--61.}
    \item Berg brought suit for lost profits and other tort damages. Wiley 
    asserted the affirmative defenses of abandonment and surrender, and he 
    counterclaimed for his own damages. 
    \item The trial court found that the lockout had been wrongful. It awarded 
    Berg \$31,000 for lost profits and \$3,540 for loss of chattels. It also 
    found that Berg had neither abandoned nor surrendered the premises.
    \item Held
    \begin{enumerate}
        \item The jury was correct in finding no surrender or abandonment.
        \item Was Wiley's self-help repossession wrongful?
        \begin{enumerate}
            \item Minnesota state law held that a landlord can resort to 
            self-help when he is legally entitled to possession and the means 
            are peacable.
            \item The trial court held that Wiley's entry was not peacable, 
            and even if it had, landlords should resort to judicial process 
            rather than self-help.
            \item The appellate court here agreed that Wiley's means were not 
            peacable. The court also agreed with the trial court's rejection 
            of the common law self-help rule in favor of a rule requiring 
            landlords to use judicial process.
            \item Wiley's reentry was both non-peacable and wrongful as a 
            matter of law.
        \end{enumerate}
    \end{enumerate}
\end{enumerate}

\paragraph{Notes on \emph{Berg v. Wiley}}

\begin{enumerate}
    \item The common law rule allowed the landlord to resort to self-help in 
    reentering, but he risked criminal penalties for forcible 
    entry.\footnote{Casebook p. 465.}
    \item One criticism of the trend away from allowing self-help is that 
    landlord will pass the costs of litigating against deadbeat tenants on to 
    tenants who pay their bills on time.
    \item Should self-help be allowed for commercial but not residential 
    leases? Should the parties be allowed to negotiate in the lease whether 
    self-help is available?
    \item Does self-help violate due process?
\end{enumerate}

\paragraph{Summary Proceedings---Purpose and Problems}

\begin{enumerate}
    \item Summary proceedings allow quick resolution of tenancy disputes. They 
    exclude issues not related to the tenancy, though some allow tenants to 
    raise problems with the conditions of the premises.
    \item But, summary eviction procedures can still be costly and 
    time-consuming.
    \item Landlords argue that judges unfairly draw out the process because of 
    a bias towards tenants. Tenants argue that they can't afford lawyers. 
    Would better access to attorneys improve tenants' rights or create more 
    burdens on landlords?
\end{enumerate}

\paragraph{Landlord's Remedies in Addition to Eviction}

\begin{enumerate}
    \item % TODO 469
\end{enumerate}

\subsubsection{The Tenant Who Has Abandoned Possession}

\paragraph{\emph{Whitehorn v. Dickerson}}

\begin{enumerate}
    \item % TODO reader
\end{enumerate}

\paragraph{\emph{Sommer v. Kridel}}

\begin{enumerate}
    \item % TODO 469
\end{enumerate}

\paragraph{Landlord's Remedies and Security Devices}

\begin{enumerate}
    \item % TODO 476
\end{enumerate}

\subsection{Duties, Rights, and Remedies}

\subsubsection{Landlord's Duties; Tenant's Rights and Remedies}

\paragraph{\emph{Whitehorn v. Dickerson}}

\begin{enumerate}
    \item % TODO supp
\end{enumerate}

\paragraph{\emph{Ingalls v. Hobbs}}

\begin{enumerate}
    \item % TODO supp
\end{enumerate}

\paragraph{\emph{Peterson v. Superior Court}}

\begin{enumerate}
    \item % TODO supp
\end{enumerate}

\paragraph{Landlord Liability for Negligence and Exculpatory Clauses}

\begin{enumerate}
    \item % TODO supp
\end{enumerate}

\paragraph{\emph{Bruckner v. Helfaer}}

\begin{enumerate}
    \item % TODO 
\end{enumerate}

\paragraph{Quiet Enjoyment and Constructive Eviction}

\paragraph{\emph{Reste Realty Corp. v. Cooper}}

\begin{enumerate}
    \item % TODO 483-491
\end{enumerate}

\paragraph{Problems on Quiet Enjoyment and Constructive Eviction}

\begin{enumerate}
    \item % TODO 491-492
\end{enumerate}

\paragraph{The Illegal Lease}

\begin{enumerate}
    \item Tenants can argue that leases for unsanitary premises are illegal 
    because they violate housing regulation. Unlike actions based on quiet 
    enjoyment and constructive eviction, the illegal defense is advantageous 
    for tenants because they can withhold rent while fending off eviction 
    actions.
    \item The doctrine ``is a dead letter,'' but tenants can now rely on the 
    implied warranty of habitability.
\end{enumerate}

\paragraph{Implied Warranty of Habitability in California: \emph{Green v. 
Superior Court}}

\begin{enumerate}
    \item At common law, landlords had no duty to keep leased premises in a 
    habitable condition.
    \item In \emph{Hinson}, a California appellate court held that landlords 
    are bound by an implied warranty of habitability. The California Supreme 
    Court here affirmed.
    \item Sumski, the landlord, sued Green, the tenant, for unlawful 
    detainer. Green admitted failing to pay back rent, but he defended on the 
    ground that Sumski had failed to keep the property in habitable condition. 
    The small claims court found for the landlord.
    \item The Superior Court judge found that the ``repair'' and ``deduct'' 
    provisions of Cal. Civ. Code \S\ 1941 were Greene's exclusive 
    remedies.\footnote{Reader p. 162.}
    \item The transformation of the landlord-tenant relationship:
    \begin{enumerate}
        \item At common law, the lessee's covenant to pay rent was considered 
        independent of the lessor's duties---so the lessee was bound to pay 
        rent even if the landlord didn't maintain the premises.
        \item Today, most renters acquire living space in a building, not 
        farmland.
        \item Apartment buildings are complex, making them difficult to 
        inspect and repair.
        \item Renters are also unlikely to be able to make home repairs 
        themselves.
        \item ``~.~.~.~the mechanism of the `free market' no longer serves as 
        as a viable means for fairly allocating the duty to repair leased 
        premises between landlord and tenant.''\footnote{Reader p. 163.}
        \item Affirm \emph{Hinson}'s holding that there is an implied warranty 
        of habitability.
    \end{enumerate}
    \item The ``repair and deduct'' provision was not an exclusive remedy.
    \item Tenants can raise the implied warranty of habitability as an 
    affirmative defense to actions for unlawful detainer, as in this case.
    \item The tenant's duty to pay rent is now mutually dependent on the 
    landlord's duty to maintain the premises.
    \item How to determine what counts as ``habitable''? Broken blinds and 
    minor water leaks do not make property uninhabitable, but lack of heat and 
    hot water do.\footnote{Reader p. 176 n. 64.}
    \item Damages should be measured as the difference between fair rental 
    value as warranted and fair rental value of the actual conditions during 
    the tenant's occupancy. For instance, if fair rental value would have been 
    \$500/month, but the lack of hot water reduced the value to \$300, the 
    tenant who paid \$500 can recover \$200.
    ``~.~.~.~the body of private property law~.~.~.~more than almost any other 
    branch of law, has been shaped by distinctions whose validity is largely 
    historical.''\footnote{Reader p. 168.}
\end{enumerate}

\paragraph{\emph{Knight v. Hallsthammar}}

\begin{enumerate}
    \item % TODO supp
\end{enumerate}

\subsubsection{The Implied Warranty of Habitability}

% TODO 493-500, 500 (note 3) - 503 (note 5)
% TODO 500 n 2
% TODO 'notes on the implied warranty' in supp

\paragraph{``Cities Deal with a Surge in Shantytowns''}

\begin{enumerate}
    \item % TODO supp
\end{enumerate}

\paragraph{\emph{Edwards v. Habib}}

\begin{enumerate}
    \item % TODO supp
\end{enumerate}

\paragraph{Questions on the Implied Warranty of Habitability}

\begin{enumerate}
    \item In \emph{Green v. Superior Court}, what reasons did the California 
    Supreme Court give for recognizing an implied warranty of habitability?  
    Was the Court's theory that a landlord and a residential tenant implicitly 
    bargain that the rental unit will be habitable throughout the tenancy?  
    \footnote{Syllabus 3/18/2013 problem 1.}
    \begin{enumerate}
        \item % TODO
    \end{enumerate}
    \item In California, does a breach of the implied warranty of habitability 
    occur when the premises become uninhabitable, or when the landlord 
    receives actual or constructive notice of the problem, or when the 
    landlord fails to repair the problem within a reasonable period of time 
    after receiving notice? (Be sure to review \emph{Peterson v. Superior 
    Court} as well as reading \emph{Knight v. 
    Hallsthammar}.)\footnote{Syllabus 3/18/2013 problem 2.}
    \begin{enumerate}
        \item % TODO
    \end{enumerate}
    \item When do you think a breach should be found and why?  Should the 
    landlord be viewed as offering an implied warranty that the premises will 
    be habitable throughout the tenancy, or should liability be based on 
    fault?\footnote{Syllabus 3/18/2013 problem 2.}
    \begin{enumerate}
        \item % TODO
    \end{enumerate}
\end{enumerate}

\subsection{The Problem of Decent Affordable Housing}

\subsubsection{Introduction}

\begin{enumerate}
    \item % TODO 508
    \item % TODO course reader 191-94
\end{enumerate}

\subsubsection{\emph{Chicago Board of Realtors, Inc. v. City of Chicago}}

\begin{enumerate}
    \item % TODO 508-515
\end{enumerate}

\subsubsection{Note on Homelessness}

\begin{enumerate}
    \item % TODO supp
\end{enumerate}

\subsubsection{California Statute regarding Commercial Rent Control}

\begin{enumerate}
    \item % TODO supp
\end{enumerate}

\subsubsection{``Enormous Cost of Rent Control''}

\begin{enumerate}
    \item % TODO supp
\end{enumerate}

\subsubsection{``California Ends Rent Control on Vacant Apartments''}

\begin{enumerate}
    \item % TODO supp
\end{enumerate}

\subsubsection{``S.F. Supes Pass New Rules on Renters' Rights''}

\begin{enumerate}
    \item % TODO supp
\end{enumerate}

