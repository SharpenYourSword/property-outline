\section{Landlord-Tenant Law}

\subsection{Leasehold Estates}

\subsubsection{The Term of Years}

\begin{enumerate}
    \item Lasts for a fixed period of time.
    \item Some states set upper limits.
\end{enumerate}

\subsubsection{The Periodic Tenancy}

\begin{enumerate}
    \item Continues for succeeding periods until the tenant or landlord gives 
    notice.
    \item Common law rules required six months' notice to terminate a 
    year-to-year tenancy. For shorter periods, notice of termination must 
    equal the period, or six months---whichever is less (i.e., the notice 
    period cannot exceed six months).
\end{enumerate}

\subsubsection{The Tenancy at Will}

\begin{enumerate}
    \item No fixed period. It endures as long as the landlord and tenant want.
    \item Some modern statutes have notice requirements.
\end{enumerate}

\subsubsection{Problems on Leasehold Estates}

\begin{enumerate}
    \item On October 1, L leases Whiteacre ``to T for one year, beginning on 
    October 1.'' T moves out the following September 30 without giving notice. 
    What are L's rights?\footnote{Casebook p. 422 problem 1.}
    \begin{enumerate}
        \item T has a term of years. T has no obligation to give notice. L has 
        no remedy.
    \end{enumerate}
    \item What if the lease had been ``from year to year, beginning October 1''?
    \begin{enumerate}
        \item T has a periodic tenancy. Under the common law rules, T was 
        required to give notice of termination six months in advance. L should 
        be able to recover.
        \item [What are the damages?] % TODO: remove
    \end{enumerate}
    \item What is the lease had been for no fixed term ``at an annual rental 
    of \$24,000, payable \$2,000 per month''?
    \begin{enumerate}
        \item T has a tenancy at will. T was under no obligation to give 
        notice, unless there was a state statutory notice requirement. L may 
        also argue that the ``annual rental'' provision established a term of 
        years or a periodic tenancy.
    \end{enumerate}
\end{enumerate}

\paragraph{``Lease for Life'': \emph{Garner v. Gerrish}}

\begin{enumerate}
    \item Gerrish rented from Donovan under a lease that allowed him to 
    terminate at any time. Donovan died a few years later. The executor of his 
    estate, Garner, tried to remove Gerrish.
    \item Garner argued that the lease created a tenancy at will because it 
    did not set a definite term. Gerrish argued that he had a tenancy for 
    life.
    \item The trial court granted summary judgment to Garner and the appellate 
    court reversed.
    \item The early common law was that the landlord and tenant must have had 
    equal rights. For instance, it wasn't allowable for the tenant to have a 
    right of termination but not the landlord. But, this rule has been mostly 
    discarded.
    \item Held: ``~.~.~.~the lease expressly and unambiguously grants to the 
    tenant the right to terminate, and does not reserve to the landlord a 
    similar right. To hold that such a lease creates a tenancy terminable at 
    the will of either party would violate the terms of the agreement and the 
    express intent of the contracting parties.''\footnote{Casebook p. 425.}
    \item The \emph{numerus clausus} principle forbids parties and courts from 
    developing new types of estates. Does the court here sanction the creation 
    of a new type of estate, the ``lease for life''?
\end{enumerate}

\subsection{The Lease}

\begin{enumerate}
    \item An instrument that resembles a lease might in fact be something 
    else, like a license or a life estate. The difference matters because 
    leases give specific rights to the landlord and tenant.
    \item A lease is both conveyance and contract. Courts often invoke 
    contract principles in addressing lease disputes.
\end{enumerate}

\subsection{Delivery of Possession}

\subsubsection{No Implied Covenant to Deliver Possession: \emph{Hannah v. 
Dusch}}

\begin{enumerate}
    \item Dutsch leased to Hannah for a 15-year term. Another tenant was 
    occupying the land when Hannah was supposed to move in.
    \item The question was whether the landlord is responsible for removing 
    trespassers and wrongdoers at the beginning of the lease term. Is there an 
    ``implied covenant to deliver possession''?\footnote{Casebook p. 439.}
    \item The ``English rule'': there is a covenant. The ``American rule'': no 
    covenant.
    \item The court here followed the American rule, holding that Hannah had a 
    remedy against the wrongdoer but not against the landlord.
\end{enumerate}

\subsubsection{Problems on Delivery of Possession}

\begin{enumerate}
    \item L and T execute a lease for a specified term. T takes possession and 
    pays rent for several months. T then learns that L had earlier leased the 
    premises to another tenant for the same term. T remains in possession but 
    stops paying rent. L sues T for unpaid rent; T counterclaims for rent 
    already paid. What result?\footnote{Casebook p. 442 problem 2.}
    \begin{enumerate}
        \item T has a term of years. As long as the other tenant did not 
        displace T, T was in possession and he should be bound by the terms of 
        the lease. L's other lease has not affected T's interests.
    \end{enumerate}
\end{enumerate}

\subsection{Subleases and Assignments}

\subsubsection{Assignment or Sublease: \emph{Ernst v. Conditt}}

\begin{enumerate}
    \item Ernst leased to Rogers for a term of one year and seven days. One of 
    the terms of the lease was that Rogers could not assign or sublet without 
    Ernst's permission.
    \item Rogers built a racetrack and other improvements. The lease 
    stipulated that at the end of the term, Rogers would remove all 
    above-ground improvements.
    \item A month after signing the lease, Rogers sold the racetrack business 
    to Conditt. Conditt and Rogers negotiated a lease extension with Ernst. 
    The modification granted subletting permissions to Conditt but held that 
    Rogers would remain liable for performance of the original terms of the 
    lease (i.e., removing improvements).
    \item Conditt stopped paying rent. Ernst demanded payment of past rent 
    from Conditt if Conditt did not remove the improvements from the property. 
    Conditt did not reply, so Ernst sued.
    \item Ernst argued that the agreement between Ernst, Rogers, and Conditt 
    \emph{assigned} the lease to Conditt, which meant that Conditt was 
    primarily liable to Ernst.
    \item Conditt argued that the agreement was a \emph{sublease}, and that 
    therefore Rogers was directly liable to Ernst.
    \item The trial court held that the agreement was an assignment and so 
    Conditt was directly liable to Ernst.
    \item The revised agreement involving Conditt specifically used the word 
    ``sublet.'' It held Rogers responsible for performane of the original 
    lease's terms. Finally, Rogers retained a reversion.
    \item The general rule is that an assignment conveys the entire term, 
    leaving no interest or reversion in the assignor. In this case, however, 
    Rogers retained an interest, so the conveyane was a sublease.
    \begin{enumerate}
        \item Common law rule: ``If the instrument purports to transfer the 
        lessee's estate for the entire remainder of his term it is an 
        assignment, regardless of its form or the parties' intention. 
        Conversely, if the instrument purports to transfer the lessee's estate 
        for less than the entire term---even for a day less---it is a 
        sublease, regardless of its form or of the parties' 
        intention.''\footnote{Casebook p. 446.}
        \item Modern rule: follow the parties' intention.
    \end{enumerate}
    \item Held: under both the common law rule and the modern rule, the 
    conveyance from Rogers to Conditt was a sublease, not an assignment. 
    Therefore, Conditt is not directly liable to Ernst. Reversed.
\end{enumerate}

\subsubsection{Problems on Subleases and Assignments}

\begin{enumerate}
    \item L leases to T for a term of three years at \$1,000/month. One year 
    later T ``subleases, transfers, and assigns'' to T1 for ``a period of one 
    year from date.'' Neither T nor T1 pays rent.\footnote{Casebook p. 448 
    problem 3(a).}
    \begin{enumerate}
        \item What rights does L have against T?
        \begin{enumerate}
            \item T retained an interest in the property, so his conveyance 
            created a sublease with T1, not an assignment. T is therefore 
            directly liable to L for rent.
        \end{enumerate}
        \item What rights does L have against T1?
        \begin{enumerate}
            \item L has no rights directly against T1.
        \end{enumerate}
    \end{enumerate}
    \item \footnote{Casebook p. 449 problem 3(b).}
    \begin{enumerate}
        \item L leases to T for three years at \$1,000/month, with a clause 
        that indicates that T agrees to pay rent before the first of each 
        month. It also prevents T from subletting or assigning without L's 
        permission. Six months later, T, with L's permission, transfers to T1 
        for the balance of the term. T1 pays rent directly to L for several 
        months and then defaults. L sues T for the rent due. What result?
        \begin{enumerate}
            \item T's conveyance to T1 was an assignment because T did not 
            retain an interest. (Under the modern rule, it might be considered 
            a sublease if T's intention was to convey a sublease rather than 
            an assignment.) After the assignment, T is no longer directly 
            liable to L for rent. L has to recover directly from T1.
        \end{enumerate}
    \end{enumerate}
    \item \footnote{Casebook p. 449 problem 3(c).}
    \begin{enumerate}
        \item L leases to T for three years at \$1,000/month with a promise to 
        pay rent in advance of the first of each month and to keep the 
        premises in good repair. Six month later T conveys her entire interest 
        to T1, who agrees to assume all covenants in the original lease 
        between L and T. Three months later T1 assigns his entire interest to 
        T2, and three months after that T2 assigns his entire interest to T3. 
        T3 defaults on rent payment and fails to keep the premises in good 
        repair. L sues T, T1, T2, and T3. What are the liabilities of the four 
        tenants to L and to each other?
        \begin{enumerate}
            \item [Does assignmetn of a lessee's entire interest convey that 
            lessee's covenants with the lessor? If yes, only T3 is liable. If 
            not, T is not liable, but T1, T2, and T3 all are.]
        \end{enumerate}
    \end{enumerate}
\end{enumerate}

\subsection{The Tenant Who Defaults}

% TODO 459-482

\subsection{Duties, Rights, and Remedies}

% TODO 482-508

\subsection{The Problem of Decent Affordable Housing}

% TODO 508-511
