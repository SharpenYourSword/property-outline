\section{Landlord-Tenant Law}

\subsection{Leasehold Estates}

\subsubsection{The Term of Years}

\begin{enumerate}
    \item Lasts for a fixed period of time.
    \item Some states set upper limits.
\end{enumerate}

\subsubsection{The Periodic Tenancy}

\begin{enumerate}
    \item Continues for succeeding periods until the tenant or landlord gives 
    notice.
    \item Common law rules required six months' notice to terminate a 
    year-to-year tenancy. For shorter periods, notice of termination must 
    equal the period, or six months---whichever is less (i.e., the notice 
    period cannot exceed six months).
\end{enumerate}

\subsubsection{The Tenancy at Will}

\begin{enumerate}
    \item No fixed period. It endures as long as the landlord and tenant want.
    \item Some modern statutes have notice requirements.
\end{enumerate}

\subsubsection{Problems on Leasehold Estates}

\begin{enumerate}
    \item On October 1, L leases Whiteacre ``to T for one year, beginning on 
    October 1.'' T moves out the following September 30 without giving notice. 
    What are L's rights?\footnote{Casebook p. 422 problem 1.}
    \begin{enumerate}
        \item T has a term of years. T has no obligation to give notice. L has 
        no remedy.
    \end{enumerate}
    \item What if the lease had been ``from year to year, beginning October 1''?
    \begin{enumerate}
        \item T has a periodic tenancy. Under the common law rules, T was 
        required to give notice of termination six months in advance. L should 
        be able to recover. % TODO: what are the damages?
    \end{enumerate}
    \item What is the lease had been for no fixed term ``at an annual rental 
    of \$24,000, payable \$2,000 per month''?
    \begin{enumerate}
        \item T has a tenancy at will. T was under no obligation to give 
        notice, unless there was a state statutory notice requirement. L may 
        also argue that the ``annual rental'' provision established a term of 
        years or a periodic tenancy.
    \end{enumerate}
\end{enumerate}

\paragraph{``Lease for Life'': \emph{Garner v. Gerrish}}

\begin{enumerate}
    \item Gerrish rented from Donovan under a lease that allowed him to 
    terminate at any time. Donovan died a few years later. The executor of his 
    estate, Garner, tried to remove Gerrish.
    \item Garner argued that the lease created a tenancy at will because it 
    did not set a definite term. Gerrish argued that he had a tenancy for 
    life.
    \item The trial court granted summary judgment to Garner and the appellate 
    court reversed.
    \item The early common law was that the landlord and tenant must have had 
    equal rights. For instance, it wasn't allowable for the tenant to have a 
    right of termination but not the landlord. But, this rule has been mostly 
    discarded.
    \item Held: ``~.~.~.~the lease expressly and unambiguously grants to the 
    tenant the right to terminate, and does not reserve to the landlord a 
    similar right. To hold that such a lease creates a tenancy terminable at 
    the will of either party would violate the terms of the agreement and the 
    express intent of the contracting parties.''\footnote{Casebook p. 425.}
    \item The \emph{numerus clausus} principle forbids parties and courts from 
    developing new types of estates. Does the court here sanction the creation 
    of a new type of estate, the ``lease for life''?
\end{enumerate}

\subsection{The Lease}

\begin{enumerate}
    \item An instrument that resembles a lease might in fact be something 
    else, like a license or a life estate. The difference matters because 
    leases give specific rights to the landlord and tenant.
    \item A lease is both conveyance and contract. Courts often invoke 
    contract principles in addressing lease disputes.
\end{enumerate}

\subsection{Delivery of Possession}

\subsubsection{No Implied Covenant to Deliver Possession: \emph{Hannah v. 
Dusch}}

\begin{enumerate}
    \item Dutsch leased to Hannah for a 15-year term. Another tenant was 
    occupying the land when Hannah was supposed to move in.
    \item The question was whether the landlord is responsible for removing 
    trespassers and wrongdoers at the beginning of the lease term. Is there an 
    ``implied covenant to deliver possession''?\footnote{Casebook p. 439.}
    \item The ``English rule'': there is a covenant. The ``American rule'': no 
    covenant.
    \item The court here followed the American rule, holding that Hannah had a 
    remedy against the wrongdoer but not against the landlord.
\end{enumerate}

\subsubsection{Problems on Delivery of Possession}

\begin{enumerate}
    \item % TODO p 442 prob 2
\end{enumerate}

\subsection{Subleases and Assignments}

\subsubsection{\emph{Ernst v. Conditt}}

\begin{enumerate}
    \item % TODO 442-447 
\end{enumerate}

\subsubsection{Problems on Subleases and Assignments}

\begin{enumerate}
    \item % TODO problems 3(a)-(c), 448-49
\end{enumerate}


\subsection{The Tenant Who Defaults}

% TODO 459-482

\subsection{Duties, Rights, and Remedies}

% TODO 482-508

\subsection{The Problem of Decent Affordable Housing}

% TODO 508-511
