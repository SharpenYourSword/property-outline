\section{Possessory Estates}

\begin{enumerate}
    \item \textbf{Freehold estate}: normal feudal tenure.
    \item \textbf{Nonfreehold estate}: lease.
    \item ``It is revolting to have no better reason for a rule than that it 
    was laid down in the time of Henry IV. It is still more revolting if the 
    grounds upon which it was laid down have vanished long since, and the rule 
    simply persists from imitation of the past.''---Holmes.\footnote{Casebook 
    p. 183.}
    \item Two types of estates in land:
    \begin{enumerate}
        \item Present possessory estates.
        \item Future interests.
    \end{enumerate}
    \item Today, the estate system makes clear (1) what property is being 
    transfered and (2) what sort of ownership is being transferred, measured 
    in the duration of the transferee's interest.\footnote{Casebook p. 191.}
\end{enumerate}

\subsection{Up from Feudalism}

\subsubsection{Tenure}

\begin{enumerate}
    \item The feudal system began under William the Conqueror's social system 
    (Norman Conquest, 1066). Land tenure was a central feature, in which 
    ``one's position was defined in terms of one's relationship to 
    land.''\footnote{Casebook p. 185.} All were subservient to the king.
    \item \textbf{Tenants in chief} held land in exchange for specific 
    services to the king, usually military, e.g., furnishing knights.
    \item The tenant in chief often provided his services by 
    \textbf{subinfeudation}, by which lower tenants provided services or 
    subinfeudated lower tenants in turn---so, a feudal pyramid developed.
    \begin{enumerate}
        \item King $\rightarrow$ Tenant in chief $\rightarrow$ 
        Mesne\footnote{Pronounced ``mean.''} lord $\rightarrow$ Tenant in 
        demesne.
        \item Each layer owns the services of the layer below.
    \end{enumerate}
    \item The Domesday Book recorded each tract of land in England and the 
    services by which the land is held.
\end{enumerate}

\subsubsection{Feudal Tenures and Services}

\begin{enumerate}
    \item Three major types of tenure with accompanying services:
    \begin{enumerate}
        \item Military tenure: \textbf{knight service} (providing men to fight 
        for the king, or the money equivalent), \textbf{grand sergeantry} 
        (``splendid court life and pageantry''\footnote{Casebook p. 186.}).
        \item Economic tenure: \textbf{socage} (intended to provide 
        subsistence for overlords; e.g., money, agriculture, maintenance, or 
        other economic services). Every land grant included a service, even if 
        nominal like annual delivery of a midsummer rose.
        \item Religious: \textbf{frankalmoign}.\footnote{Casebook p. 187.}
    \end{enumerate}
    \item There were also unfree tenures granted to villeins (peasants), who 
    initially had no legal protection, although courts eventually recognized 
    their holdings as on an equal footing with others.
\end{enumerate}

\subsubsection{Feudal Incidents}

\begin{enumerate}
    \item \textbf{Incidents}: a tenant's duties and liabilities to his lord.
    \item \textbf{Homage and fealty}: ceremonial allegiance.
    \item \textbf{Aids}: financial demands in emergencies, e.g., paying a 
    ransom to the lord's captors.
    \item \textbf{Forfeiture}: if a tenant breached his oath or refused to 
    perform feudal services, he may forfeit his land to the lord. For high 
    treason, he forfeits his land to the king.
    \item When a tenant dies:
    \begin{enumerate}
        \item \textbf{Wardship and marriage}: the lord can 
        hold the land for the tenant's underage heirs. The lord could also 
        sell the heir in marriage.
        \item \textbf{Relief}: the heir has to pay the lord to take control of 
        his inheritance.
        \item \textbf{Escheat}: if a tenant dies without heirs, the land goes 
        to the heirs. 
    \end{enumerate}
\end{enumerate}

\subsubsection{Avoidance of Feudal Incidents}

\begin{enumerate}
    \item Tenants developed ways of evading feudal incidents. Suppose a 
    tenant, T, holds land from L by knight service. T subinfeudated to T1, 
    ``reserving as service one rose at midsummer.''\footnote{Casebook p. 189.} 
    T was still responsible for knight service to L, but the subinfeudation 
    devalued the incidents due to L of warship, marriage, relief, and 
    escheat.
    \item Subinfeudation ended with the Statute Quia Emptores in 
    1290.\footnote{Casebook p. 190.}
\end{enumerate}

\subsubsection{The Decline of Feudalism}

\begin{enumerate}
    \item Quia Emptores prohibited subinfeudation in fee simple, but it 
    required lords to allow tenants to transfer their land to others (with the 
    same obligations due to the lord). It had two major consequences:
    \begin{enumerate}
        \item Established a principle of \textbf{free alienation} of land.
        \item Existing mesne relationships disappeared, so that most land was 
        eventually held directly from the crown.
    \end{enumerate}
    \item Wages rose over time, giving peasants increased independence and 
    legal rights.
\end{enumerate}

\subsection{The Fee Simple}

\begin{enumerate}
    \item Tenants have \textbf{status} as (1) tenant of the fee or (2) tenant 
    for life.
    \item Over time, status became \textbf{estate}, defined by the length of 
    time it may endure.
    \item \textbf{Fee simple} (short for fee simple absolute) estates can 
    endure forever. (Life estate: for the life of the person; a term of years, 
    for some period of time.)
\end{enumerate}

\subsubsection{How the Fee Simple Developed}

\begin{enumerate}
    \item \textbf{Heritability}: under feudalism, land was not owned, but held 
    by the possessor as as tenant of another. The tenant's holding 
    (\textbf{fee} or \textbf{fief}) could not be inherited, although the lord 
    would often give it to the heir on payment of relief. Heritability of land 
    gradually arose as a right over time.
    \item \textbf{Alienability}: By the end of the thirteenth century, Quia Emptores 
    ensured that the fee was freely alienable during the tenant's life.
    \item \textbf{Fee simple estate}:
    \begin{enumerate}
        \item Alienability allowed tenants to pass land to others and their 
        heirs, causing holdings to become \textbf{freehold estates} that were 
        not terminable at the lord's will.\footnote{Casebook p. 193.}
        \item \textbf{Estate in land}: e.g., a fee simple---a legal 
        abstraction that we treat as a physical thing (it can be given, sold, 
        bequeathed; creditors can seize it). Estates ``vest, divest, merge, 
        are destroyed, shift, spring.''\footnote{Casebook p. 193.}
    \end{enumerate}
    \item ``The fee simple absolute is as close to unlimited ownership as our 
    law recognizes. It is the largest estate in terms of duration. It may 
    endure forever.''\footnote{Casebook p. 194.}
\end{enumerate}

\subsubsection{Creation of a Fee Simple}

\begin{enumerate}
    \item At early common law, a fee simple was created by conveying land 
    \textbf{``to A and his heirs.''}
    \item ``And his heirs'' are only \textbf{words of limitation}, meaning 
    that the heirs do not take as ``purchasers,'' i.e., they do not have an 
    interest in the land. They have an interest only after A's death.
    \item ``To A'' are \textbf{words of purchase}, meaning A is the grantee.
    \item Adding ``and his heirs'' is no longer necessary to create a fee 
    simple, but lawyers still do it out of habit and caution.
\end{enumerate}

\paragraph{Problems on Fee Simple Creation}

\begin{enumerate}
    \item In 1600, O conveys property ``to A for life, then to B forever.'' 
    What estates do A and B have? If A dies and then B dies, who owns the 
    property? What if the conveyance happens in 2002?\footnote{Casebook p. 194 
    problem 1. See reader p. 36, I and II.}
    \begin{enumerate}
        \item If the conveyance happened in 1600, A and B would both have life 
        estates, because ``to his heirs'' was not included. If A and B both 
        died, The land would revert to O.
        \item If the conveyance happened in 2002, A would have a life estate 
        and B would have a fee simple absolute (because ``to his heirs'' is no 
        longer required to create a fee simple absolute). If A and B both 
        died, B's heirs would own the land.
    \end{enumerate}
    \item O conveys property to ``to A and her heirs.'' A's only child, B, 
    runs up large bills. B's creditors can attach B's property to satisfy 
    their claims. Does B have an interest in the property, reachable by B's 
    creditors? What if A wants to sell the property and take a trip around the 
    world---can B stop her?\footnote{Casebook p. 195 problem 3.}
    \begin{enumerate}
        \item B does not acquire an interest in the property until A's death. 
        So, B's creditors cannot reach the property, and B cannot stop A from 
        selling the property.
    \end{enumerate}

\end{enumerate}

\subsubsection{Inheritance of a Fee Simple}

\paragraph{Heirs}

\begin{enumerate}
    \item Heirs are intestate successors. A living person has no heirs.
    Heirs are identified only if the person dies intestate.
    \item In all states today, the surviving spouse is an intestate successor. 
    The size of the spouse's share depends on other survivors.
    \item Statutes of descent prefer heirs in this order (after spouses):
    \begin{enumerate}
        \item First issue.
        \item Parents.
        \item Collaterals.
    \end{enumerate}
\end{enumerate}

\paragraph{Issue}

\begin{enumerate}
    \item ``Issue'' = descendants---children, grandchildren, and on down.
    \item ``Per stirpes'' = ``by the stocks,'' meaning if a child of the 
    decedent dies before the decedent but leaves children who survive, the 
    child's share goes to the child's children.
    \item Until 1925, rules of primogeniture controlled inheritance. Children 
    born out of wedlock were \emph{filius nullius}.
\end{enumerate}

\paragraph{Ancestors}

\begin{enumerate}
    \item Parents take as heirs if the decedent leaves no issue.
\end{enumerate}

\paragraph{Collaterals}

\begin{enumerate}
    \item Collateral kin are people related by blood who are neither 
    descendants nor ancestors---brothers, sisters, nephews, nieces, uncles, 
    aunts, and cousins.
\end{enumerate}

\paragraph{Escheat}

\begin{enumerate}
    \item If a person dies without heirs, the property goes to the state.
\end{enumerate}

\subsubsection{Problems on Inheritance of a Fee Simple}

\begin{enumerate}
    \item O has two children, A (daughter) and B (son). B dies testate, 
    leaving all of his property to W, his wife. B is survived by B1 
    (daughter), B2 (son), and B3 (daughter). A1 (son) is born to A. O then 
    dies intestate.\footnote{Casebook p. 196 problem 1, reader p. 196.}
    \begin{enumerate}
        \item Who owns the property in England in 1800?
        \begin{enumerate}
            \item B2, under primogeniture.
        \end{enumerate}
        \item Who owns the property under modern American law?
        \item A gets one-half and B1, B2, and B3 each get one-sixth. W gets 
        nothing because when B devised his property to A, he had only a 
        potential interest in O's property, which he could not devise.
        % TODO technical term for B's potential interest at the time of his 
        % death?
    \end{enumerate}
\end{enumerate}

\subsection{The Fee Tail}

\begin{enumerate}
    \item Created by a conveyance ``to A and the heirs of his 
    body.'' ``The fee tail descends to A's lineal descendants (`heirs of the 
    body') generation after generation'' and expires when all of A's 
    descendants are dead.\footnote{Casebook pp. 198--99.}
    \item When the fee tail ends, the land reverts to the grantor's heirs by 
    reversion, or, if specified, to another branch of the family.
    \item Originally, a tenant in fee tail could alienate his possessory 
    interest for his lifetime, but he could not affect his issues' rights to 
    the land after his death. Now, a fee tail tenant can simply convey a fee 
    simple by deed.\footnote{Casebook pp. 200--201.}
    \item Jefferson convinced the Virginia legislature to abolish the fee tail 
    and primogeniture. Today, it exists only in DE, ME, MA, and RI. ``The fee 
    tail has been replaced by the life estate as a device for controlling 
    inheritance.''\footnote{Casebook p. 201.}
    \item Today, what kind of estate is created through language that would 
    have traditionally created a fee tail (``to A and the heirs of his 
    body, and if A dies without issue to my daughter B and her heirs'')? Some 
    states create a life estate, but most fall into \textbf{two 
    categories}:\footnote{Casebook p. 201.}
    \begin{enumerate}
        \item A gets a fee simple. Neither A's issue nor B take anything.
        \item A gets a fee simple, and B gets a fee simple only if A dies 
        without heirs.
    \end{enumerate}
    \item The Restatement (Third) does not recognize the fee tail.
\end{enumerate}

\subsubsection{Problem on Fees Tail}

\begin{enumerate}
    \item O conveys land ``to A and the heirs of her body, and if A dies 
    without issue, to B and his heirs.'' A then conveys ``to C and heir 
    heirs.'' A then married D and has a son E. A dies intestate and is 
    survived by O, B, C, D, E.\footnote{Syllabus problem 1, 1/25/2013.}
    \item Who owns the land in Massachusetts?
    \begin{enumerate}
        \item Massachusetts is one of four states that continues to recognize 
        the fee tail. % TODO
    \end{enumerate}
    \item Who owns the land in states that fall into category one (above)?
    \begin{enumerate}
        \item % TODO
    \end{enumerate}
    \item Who owns the land in states that fall into category two (above)?
    \begin{enumerate}
        \item % TODO
    \end{enumerate}
    \item If E had predeceased A, who would now own the land in these states?
    \begin{enumerate}
        \item % TODO
    \end{enumerate}
\end{enumerate}

\subsection{The Life Estate}

\begin{enumerate}
    \item At common law, the life estate had two important consequences:
    \begin{enumerate}
        \item The grantor could control the land after the grantee's death, 
        allowing the life estate to supplant the fee tail as a device to 
        control inheritance.
        \item Trust management for life tenants developed.
    \end{enumerate}
    \item Today, most life estates are created in trust.
    \item ``Every life estate is followed by a future interest---either a 
    reversion in the transferor or a remainder in a 
    transferee.''\footnote{Casebook p. 202.}
    \item Pros and cons of life estates:\footnote{Casebook p. 215.}
    \begin{enumerate}
        \item \emph{Sale}: it might become advantageous to sell the land at 
        some point, but the life estate holder will not be able to sell it 
        until all future interest holders consent (or until a court orders the 
        sale---\emph{Baker v. Weedon}).
        \item \emph{Lease}: the life tenant may want to lease the property 
        beyond his lifespan.
        \item \emph{Mortgage}: banks usually don't lend on life estates.
        \item \emph{Waste}: taking minerals or cutting timber might be 
        considered waste and therefore blocked.
        \item \emph{Insurance}: life tenants are under no duty to insure 
        buildings on the land.
    \end{enumerate}
    \item Waste occurs when the life tenant unreasonably interferes with the 
    future interest holder.\footnote{Casebook p. 217.} Some actions change the 
    property but substantially increase its value---see \emph{Woodrick} below.
\end{enumerate}

\subsubsection{Ambiguous Life Estates: \emph{White v. Brown}}

When a conveyance is ambiguous as to whether it creates a life estate or a fee 
simple, courts should interpret it as a fee simple.

\begin{enumerate}
    \item People:
    \begin{enumerate}
        \item Mrs. Jessie Lide, a widow with no children.
        \begin{enumerate}
            \item Twelve nieces and nephews---the defendants.
        \end{enumerate}
        \item Evelyn White, sister-in-law of Jessie Lide.
        \begin{enumerate}
            \item Her daughter, Sandra White Perry.
            \item The Whites lived with Jessie Lide for several decades.
        \end{enumerate}
    \end{enumerate}
    \item Facts:
    \begin{enumerate}
        \item February 15, 1973: Jessie Lide died, leaving a handwritten will, 
        which included the words ``I wish Evelyn White to have my home to live 
        in and not to be sold.''\footnote{Casebook p. 203.}
    \end{enumerate}
    \item Evelyn White brought suit with her daughter Sandra, arguing that the 
    will gave her a fee simple. The defendants argued that it conveyed only a 
    life estate, leaving the remainder to them under the rules of intestate 
    succession.
    \item The language of the will was unclear about the type of estate it 
    created, so the court looked to Lide's intent. The words ``to live in'' 
    indicated an intent to convey a fee simple absolute. The phrase ``not to 
    be sold'' also ``does not evidence such a clear intent to pass only a life 
    estate as is sufficient to overcome the law's presumption that a fee 
    simple interest was conveyed.\footnote{Casebook pp. 205--06.}
    \item ``Accordingly, we conclude that Mrs. Lide's will passed a fee 
    simple absolute in the home to Mrs. White.''\footnote{Casebook p. 206.}
    \item Judge Harbison, dissenting:
    \begin{enumerate}
        \item Lide ``knew how to make an outright gift, if 
        desired.''\footnote{Casebook p. 206.} But here, Lide intentionally 
        created a limitation, and it should have been interpreted as creating 
        a life estate.
    \end{enumerate}
\end{enumerate}

\subsubsection{Restraints on Alienation}

\begin{enumerate}
    \item The rule against direct restraints on alienation dates to at least 
    the fifteenth century.\footnote{Casebook p. 208.}
    \item There are four justifications:
    \begin{enumerate}
        \item Restraints make property unmarketable, which may prevent it from 
        being put to good use.
        \item Restraints perpetuate concentrations of wealth.
        \item Restraints discourage improvements on land.
        \item Restraints prevent owners' creditors from accessing the land.
    \end{enumerate}
    \item Three types:
    \begin{enumerate}
        \item \textbf{Disabling restraint}: a grantee cannot transfer his 
        interest. \emph{White v. Brown}.
        \item \textbf{Forfeiture restraint}: a grantee loses his interest if 
        he attempts to transfer it.
        \item \textbf{Promissory restraint}: a grantee promises not to 
        transfer his interest. If valid, it is enforceable by the contract 
        remedies of damages or an injunction.
    \end{enumerate}
    \item Second Restatement: absolute restraints are void, but conditional 
    restraints (e.g., limited to certain people or for a certain time limit) 
    are valid if reasonable.
\end{enumerate}

\subsubsection{Rights of Contingent Remainder Holders: \emph{Baker v. Weedon}}

Courts can force the sale of land in which there are future interests if the 
sale is ``necessary for the best interest of all the 
parties.''\footnote{Casebook p. 214.}

\begin{enumerate}
    \item People:
    \begin{enumerate}
        \item John Weedon.
        \begin{enumerate}
            \item First wife: Lula Edwards.
            \begin{enumerate}
                \item Kids: Florence Baker, Delette Jones.
                \begin{itemize}
                    \item Florence Baker's children: Henry Baker, Sarah Lyman, 
                    Louise Heck (appellants).
                    \item Delette Jones adopted a daughter, Dorothy, who had 
                    not been heard of for years.
                \end{itemize}
            \end{enumerate}
            \item Second wife: Ella Howell.
            \begin{enumerate}
                \item Child: Rachel. Both Rachel and Ella were dead at the 
                time of the dispute.
            \end{enumerate}
            \item Third wife: Anna Plaxico. No children.
        \end{enumerate}
    \end{enumerate}
    \item Facts:
    \begin{enumerate}
        \item 1905: John Weedon bought Oakland Farm.
        \item 1915: John Weedon married Anna Plaxico. John and Anna worked 
        together on the farm.
        \item 1925: Weedon had little contact with his daughters Florence and 
        Delette. He wrote a will excluding his daughters and bequeathing all 
        of his property to Anna. The will stipulated that if Anna died without 
        issue, the property would go to his grandchildren---so, the 
        grandchildren had contingent remainders.
        \item 1932: John Weedon died.
        \item 1933: Anna Plaxico remarried.
        \item 1955: Anna stopped working on the farm because of her age and 
        began renting it out. She had no children.
        \item 1964: The highway department wanted to buy the farm so it could 
        expand the freeway. It located Florence Baker's three children, who 
        were until then unaware of any inheritance.
    \end{enumerate}
    \item Anna brought suit against the grandchildren to have the property 
    (minus the house) sold so that the court could invest the money and live 
    off the interest. The question was whether a court can order the sale of 
    land in which there are future interests. The lower court held in her 
    favor.
    \item The court here held that courts generally have the power to direct a 
    judicial sale of land, but the scope of the power is not well defined. The 
    sale must be ``necessary for the best interest of all the parties, that 
    is, the life tenant and the contingent remaindermen.''\footnote{Casebook 
    p. 214.}
    \item In this case, selling the property would result in great financial 
    loss for the remaindermen (Florence's three children). Reversed with an 
    order to sell only as much land would be necessary to meet Anna's 
    financial need.
\end{enumerate}

\subsubsection{Rights of Remainder Interest Holders: \emph{Woodrick v. Wood}}

\begin{enumerate}
    \item Catherine and George Wood owned several pieces of land, including 
    lot 105. George died in 1987. In his will, he left all of his property to 
    Catherine Wood, and provided that upon her death, the property should be 
    divided between Sheridan Wood and Patricia Woodrick. Sheridan conveyed his 
    interest in lot 105 to Catherine.
    \item Catherine Wood had a life estate and a 75\% remainder interest in 
    lot 105 and full ownership in fee simple of lot 106. Patricia Woodrick had 
    a 25\% remained interest in lot 105.
    \item The Woods wanted to raze a barn on lots 105 and 106. Patricia 
    Woodrick sued to enjoin them. The trial court held for the Woods.
    \item Can the holder of a remainder interest prohibit the life tenant from 
    destroying structures on the land?
    \item Injunctions were available to those with remainder 
    interests to prevent waste. At common law, ``anything which in any way 
    altered the identify of lease premises was waste.''\footnote{Casebook p. 
    219.} But Ohio did not recognize the common law rule.
    \item ``The relevant inquiry is always whether the contemplated act of the 
    life tenant would result in dimunition of the value of the 
    property.''\footnote{Casebook p. 220.} Here, removing the barn would 
    actually increase the property value. Affirmed.
\end{enumerate}

\subsubsection{Seisin}

\begin{enumerate}
    \item Seisin\footnote{Pronounced ``season.''} is possession of a certain kind with certain consequences. 
    For instance, tenants seised of the land were responsible for feudal 
    services.
    \item Before 1536, a freehold estate could only be transfered through the 
    ceremony of \textbf{feeoffment with livery of seisin}.\footnote{Casebook 
    p. 221.}
\end{enumerate}

\subsection{Leasehold Estates}

\begin{enumerate}
    \item ``Leasehold estates are nonfreehold possessory 
    estates.''\footnote{Casebook p. 222.} Early property law distinguished 
    between the tenant's physical possession and the landlord's seisin, but 
    the distinction is no longer relevant.
\end{enumerate}

\subsection{Defeasible Estates}

\begin{enumerate}
    \item A \textbf{defeasible} estate ends prior to its natural end point 
    when a specified event occurs. For instance, a normal life estate ends 
    when the life tenant dies, but a \emph{defeasible} life estate ends when a 
    condition in the conveyance is satisfied---e.g., O conveys property to A 
    ``so long as the property is only used for residential 
    purposes.''\footnote{Casebook p. 223.} That estate would end automatically 
    when it is no longer used for residential purposes.
    \item Defeasible fees are mainly used to control land use (``this land can 
    only be used for school purposes'') or behavior (``this land is yours as 
    long as you don't drink'').
    \item Three types:
    \begin{enumerate}
        \item \textbf{Fee simple determinable}: ends automatically when a 
        stated event happens. The language must include a ``durational 
        aspect''---for instance, ``\emph{so long as} the land is used for 
        school purposes.''\footnote{Casebook p. 223.}
        \begin{enumerate}
            \item \emph{Future interest}: \textbf{possibility of 
            reverter}---O, the transferor, or his heirs retain the future 
            interest. It arises because O has transferred less than his entire 
            interest when he creates a determinable fee. The conveyance back 
            to O happens automatically.
        \end{enumerate}
        \item \textbf{Fee simple subject to condition subsequent}: ``a fee 
        simple that does not automatically terminate but \emph{may be cut 
        short} or divested at the transferor's election when a stated 
        condition happens.''\footnote{Casebook p. 224.} It is created by 
        conditional language---``but if,'' ``provided,'' ''on condition 
        that.'' 
        \begin{enumerate}
            \item \emph{Future interest}: \textbf{Right of entry} (or power of 
            termination)---O can exercise his right to reclaim possession of 
            the property, but there will not be an automatic conveyance.
        \end{enumerate}
        \item \textbf{Fee simple subject to executory limitation}: created 
        when another defeasible fee creates a future interest in a third 
        party, rather than the original transferor.
        \begin{enumerate}
            \item \emph{Future interest}: \textbf{executory interest}---the 
            third party has the right to take possession of the property if 
            the conditions of the defeasible fee are met.
        \end{enumerate}
    \end{enumerate}
    % TODO: 241-42
\end{enumerate}

\subsubsection{\emph{Transferability of Future Interests: Mahrenholz v. County 
Board of School Trustees}}

% TODO: verify
The right of reentry cannot be transferred,
%%%%%%%%
because the holder of the right 
does not possess the property until he exercises his right of reentry. But it 
might be possible to transfer a possibility of reverter. The difference 
between these two types of future interests depends on the language used to 
create them. Language with a ``durational aspect'' (while, during, until, 
etc.) creates a fee simple determinable with a possibility of reverter, while 
conditional words (upon the condition, provided that) create a fee simple 
subject to condition subsequent with the right of reentry.

\begin{enumerate}
    \item Timeline:
    \begin{enumerate}
        \item March 18, 1941: the Huttons conveyed 1.5 acres (the ``Hutton 
        School grounds'') out of 40 acres to the school district. The deed 
        provided that ``this land to be used for school purpose only; 
        otherwise to revert to Grantors herein.''\footnote{Casebook p. 226.}
        \item July 1941: the Huttons conveyed the remaining 38.5 acres to the 
        Jacqmains. The deed also purported to convey the reversionary 
        interest in the 1.5 acres that had been conveyed to the school.
        \item July 18, 1951: Mr. Hutton died intestate.
        \item February 18, 1969: Mrs. Hutton died intestate. The Huttons' only 
        legal heir was their son Harry.
        \item October 9, 1959: the Jacqmains conveyed the 38.5 acres and the 
        reversionary interest in the 1.5 acres to the Mahrenholzes.
        \item May 30, 1973: the school stopped holding classes at the school 
        that was built on the land. The school district continued to use the 
        buildings for storage.
        \item May 7, 1977: Harry Hutton conveyed to the Mahrenholzes all of 
        his interest in the Hutton School land. The document was filed on May 
        7, 1977. On September 6, Hutton disclaimed his interest in the 
        property in favor of the Mahrenholzes.
    \end{enumerate}
    \item The Mahrenholzes sued to quiet title to the school property in 
    themselves. The trial court dismissed the complaint, holding that the 
    conveyance from the Huttons to the school district created a fee simple 
    subject to a condition subsequent followed by the right of entry for 
    condition broken, rather than a determinable fee followed by a possibility 
    of reverter.\footnote{Casebook p. 227.}
    \item ``The basic issue presented by this appeal is whether the trial 
    court correctly concluded that the plaintiffs could not have acquired any 
    interest in the school property from the Jacqmains and Harry 
    Hutton.''\footnote{Casebook p. 227.}
    \item The type of defeasible fee created by the original conveyance to the 
    school board is in dispute. It depends on how the court interprets the 
    language in the conveyance. But the only two possible future interests 
    that could have been created are (1) a possibility of reverter or (2) a 
    right of reentry. Neither of these future interests are transferable 
    inter vivos under Illinois law, so ``the trial court correctly ruled that 
    the plaintiffs could not have acquired any interest in that property from 
    the Jacqmains by the deed of October 9, 1959.''\footnote{Casebook p. 227.}
    \item The next question is whether the Mahrenholzes could have acquired in 
    interest in the Hutton School grounds directly from Harry Hutton. It 
    depends on the language of the original 1941 deed conveying the land from 
    the Huttons to the school district. The Mahrenholzes argued that it 
    created a fee simple determinable followed by a possibility of reverter. 
    The defendants argued (and the trial court held) that it created a fee 
    simple subject to a condition subsequent followed by a right of reentry 
    for condition broken.
    \item Neither future interest is alienable or devisable, but both are 
    inheritable.
    \item Critically, if Harry Hutton had a right of reentry, he did not own 
    the school property until he exercised his right. But if he had a 
    possibility of reverter, the property became his automatically as soon as 
    it was no longer used for school purposes. So, the Mahrenholzes could only 
    have an interest in the property if Harry had acquired a possibility of 
    reverter, not a right to reentry.
    \item The difference between a fee simple determinable and a fee simple 
    subject to condition subsequent hangs on the terminology. Words with a 
    durational aspect (so long as, while, until) would have created a fee 
    simple determinable. Words creating a condition (upon condition that, 
    provided that) would have created a fee simple subject to condition 
    subsequent.
    \item Here the court held that the word ``only'' was intended to create a 
    fee simple determinable followed by the possibility of reverter. Thus, 
    when the property was no longer used for school purposes, Harry Hutton 
    automatically became the owner. It was therefore possible that he could 
    have conveyed his interest to the Mahrenholzes, though the appellate court 
    here left that issue for the trial court to resolve.
    % TODO: 231-34
\end{enumerate}

\subsubsection{Walch, ``Maeser School Crisis Over''}

\begin{enumerate}
    \item % TODO 234-36
\end{enumerate}


\subsubsection{Transferability of Future Interests}

\begin{enumerate}
    \item At common law, possibility of reverter and right of entry descended 
    to the heirs of the owner of the interests. But neither interest was 
    transferable during the owner's life. This was because a possibility of 
    reverter was viewed as a possibility, not a transferable thing. Similarly, 
    a right of entry was not a thing, either---instead, it was like a right to 
    sue, which was not transferable.\footnote{Casebook p. 232.}
    \item Today, most states (with a few exceptions) allow inter vivos 
    transfer of possibility of reverter and right of entry.
\end{enumerate}

\subsubsection{Problems on Defeasible Estates}

\begin{enumerate}
    \item Does it make sense to continue the distinction between the fee 
    simple determinable and the fee simple condition to condition subsequent? 
    The distinction turns on subtle differences in language (``so long as'' 
    vs. ``upon condition that'').
    \begin{enumerate}
        \item California and Kentucky have abolished the fee simple 
        determinable by statute. In those states, language that would have 
        created a fee simple determinable now creates a fee simple subject to 
        condition subsequent.
        \item The draft of the Third Restatement abolishes all three 
        defeasible fees. It replaces them with the ``fee simple defeasible,'' 
        defined as ``a present interest that terminates upon the happening of 
        a stated event that might or might not occur.''\footnote{Casebook p. 
        233.}
        \item ~\\\\\\TODO problems? see p. 233 and syllabus\\\\\\
    \end{enumerate}
\end{enumerate}

\subsubsection{Restraints on Alienation and Use Restrictions: \emph{Mountain 
Brow Lodge No. 82, Independent Order of Odd Fellows v. Toscano}}

A use restriction creates a valid FSSCS even if it effectively creates an 
absolute restraint on alienation.

\begin{enumerate}
    \item The Toscanos conveyed the land to Mountain Brow Lodge in 1950 ``for 
    the use and benefit of the [Lodge] only; and in the event that the same 
    fails to be used by the second party or in the event of sale or transfer 
    by the second party of all or any part of said lot, the same is to revert 
    to the first parties herein, their successors, heirs or 
    assigns.''\footnote{Casebook p. 237.}
    \item The Lodge apparently stopped using the land for Lodge purposes. The 
    Toscanos' heirs argued that the conveyance created a FSSCS and was 
    enforceable. The Lodge argued that it created an absolute restraint on 
    alienation and was void.
    \item The trial court held for the Toscanos' heirs.
    \item On appeal, the court separated the restraint on alienation, which it 
    held to be void, and the use restriction, which it held to create an 
    enforceable FSSCS.
    \item The dissent argued that the use restriction, which only allowed 
    I.O.O.F. Lodge No. 82 to use the land, was effectively identical to a 
    restraint on alienation and should have been held void.
\end{enumerate}

\subsubsection{Defeasible Estates and Personal Conduct Restraints}

\begin{enumerate}
    \item % TODO 249-50
\end{enumerate}

