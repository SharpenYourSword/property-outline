\section{Acquisition by Find}

\begin{enumerate}
    \item \textbf{Bailment}: Voluntary or involuntary delivery of personal 
    property to another, e.g., dropping of laundry at the laundromat 
    (voluntary) or losing a jewel in a chimney to be later found by a chimney 
    sweep (involuntary---\emph{Armory}).
    \item \textbf{Trover}: ``forced purchase''\footnote{Casebook p. 
    99.}---action for recovery of damages for conversion of personal property, 
    generally measured by the property's value---e.g., the chimney sweep could 
    recover the value of the jewel, but not the jewel itself. \emph{Armory}.
    \item \textbf{Replevin}: action for recovery of personal property that the  
    defendant wrongfully took. \emph{Anderson}.
\end{enumerate}

\subsection{Finder's Rights: \emph{Armory v. Delamirie}}

The finder's right of ownership trumps all but the original owner's right.

\begin{enumerate}
    \item A chimney sweep found a jewel in a chimney. He brought it to a 
    goldsmith for appraisal, who offered a small amount of money, and when the 
    sweep refused, returned the socket without the jewel.
    \item The chimney sweep brought a trover action against the goldsmith. The 
    court found for the chimney sweep, holding that a finder has a right to 
    ownership above all but the property's rightful owner.
    \item \textbf{Winkfield Doctrine}: In cases of voluntary bailment, the 
    bailor (i.e., the original owner) does not have an action against the 
    present possessor if the bailee has recovered in full from the present 
    possessor.\footnote{Casebook p. 99.}
    \item Earlier finders' rights trump later finders'.
    \item Applies to both real and personal property.
\end{enumerate}

\subsection{Relative Rights: \emph{Anderson v. Gouldberg}}

The plaintiff's property right must only be valid relative to the defendant's 
right. Property rights are relative.

\begin{enumerate}
    \item Plaintiffs harvested lumber while trespassing and took the logs to 
    the defendants' mill, where the defendants took them.
    \item The court found for the plaintiffs, holding that the plaintiff's 
    right to the property must only have been lawful against the defendant, 
    not all possible parties. The fact that the plaintiff may have originally 
    obtained the property illegally is irrelevant to its right to the property 
    relative to these defendants.
\end{enumerate}

\subsection{Landowner's Rights to Found Property: \emph{Hannah v. Peel}}

Landowners do not have rights to property that others found on their property 
if the landowner was not in physical possession of the premises---but ``the 
authorities are in an unsatisfactory state.''

\begin{enumerate}
    \item The house of defendant, Peel, was requisitioned, released, and 
    requisitioned again. Peel never occupied the house. Plaintiff Hannah, one 
    of the soldiers stationed there, found a brooch in a window frame. He 
    turned it over to the police. Two years later, the original owner had 
    still not showed up, so the police gave the brooch to Peel, who sold it 
    for 66\emph{l}.
    \item Hannah argued that he had a right to possession above all others 
    except the original owner's, and the original owner could not be located 
    (\emph{Armory}).
    \item The court reviewed conflicting precedent, concluding that ``the 
    authorities are in an unsatisfactory state.~.~.~'':\footnote{Casebook p. 
    106.}
    \begin{enumerate}
        \item \emph{Bridges v. Hawkesworth}: A shop customer found a package 
        of bank notes and turned it over to the shopkeeper. Nobody claimed the 
        package, and the customer sued to recover from the shopkeeper. The 
        court, following \emph{Armory}, found for the plaintiff.
        \item \emph{South Staffordshire Water Co. v. Sharman} A pool cleaner 
        found two rings in the mud. The court held that ``if something is 
        found on [an owner's] land, whether by an employee of the owner or by 
        a stranger, the presumption is that the possession of that thing is in 
        the owner of the locus in quo.''\footnote{Casebook p. 105.}
        \item \emph{Elwes v. Brigg Gas Co.}: A gas company with a 99 year 
        lease discovered an ancient boat buried in the soil. The court held 
        that the lessors would have had a right to the boat if it had been a 
        mineral (presumably because their purpose for using the land was to 
        draw minerals from it). But it held that the boat was a chattel and 
        that, since the original owner was long gone, the landowner had the 
        stronger right. The landowner's obliviousness was irrelevant: ``In my 
        opinion it makes no difference, in these circumstances, that the 
        plaintiff was not aware of the existence of the 
        boat.''\footnote{Casebook p. 105.}
    \end{enumerate}
    \item The court held that Peel was ``never in possession of the 
    premises,'' and therefore he did not have prior possession of the 
    brooch.''
\end{enumerate}

% \subsection{\emph{McAvoy v. Medina}}
% 
% \begin{enumerate}
%     \item % TODO
% \end{enumerate}
