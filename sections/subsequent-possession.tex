\section{Subsequent Possession}

\subsection{Finders}

\begin{enumerate}
    \item \textbf{Bailment}: Voluntary or involuntary delivery of personal 
    property to another, e.g., dropping of laundry at the laundromat 
    (voluntary) or losing a jewel in a chimney to be later found by a chimney 
    sweep (involuntary---\emph{Armory}).
    \item \textbf{Trover}: ``forced purchase''\footnote{Casebook p. 
    99.}---action for recovery of damages for conversion of personal property, 
    generally measured by the property's value---e.g., the chimney sweep could 
    recover the value of the jewel, but not the jewel itself. \emph{Armory}.
    \item \textbf{Replevin}: action for recovery of personal property that the  
    defendant wrongfully took. \emph{Anderson}.
    \item In finders cases, courts weight the relative strength of the 
    finder's claim to the property against his opponent's claim to that 
    property.
    \item \textbf{Relativity of title}: TO \textgreater\ F1 \textgreater\ F2. 
    ``It's more mine than yours.''
\end{enumerate}

\subsubsection{Prior Possessor Prevails: \emph{Armory v. Delamirie}}

The finder's right of ownership trumps all but the original owner's right (or 
earlier finders' rights).

\begin{enumerate}
    \item A chimney sweep found a jewel in a chimney. He brought it to a 
    goldsmith for appraisal, who offered a small amount of money, and when the 
    sweep refused, returned the socket without the jewel.
    \item The chimney sweep brought a trover action against the goldsmith. The 
    court found for the chimney sweep, holding that a finder has a right to 
    ownership above all but the property's rightful owner.
    \item \textbf{Winkfield Doctrine}: In cases of voluntary bailment, the 
    bailor (i.e., the original owner) does not have an action against the 
    present possessor if the bailee has recovered in full from the present 
    possessor.\footnote{Casebook p. 99.}
    \begin{enumerate}
        \item The justification is that it's not fair to the present 
        possessor to be doubly liable.
        \item Courts often apply the Winkfield Doctrine, but not always.
        \item Bailees have a duty to take care of the property. If a wrongdoer 
        takes the property and the bailee recovers in trover, it's an open 
        question whether the bailee should also have to take care of the money 
        in case the true owner (bailor) wants it back.
        \item Should Winkfield apply to involuntary bailment?
        \begin{enumerate}
            \item In voluntary bailment, the true owner consents to the 
            bailee's possession, and the bailee represents the owner's 
            interests in the property against wrongdoers. In involuntary 
            bailment, there's no such consent.
        \end{enumerate}
    \end{enumerate}
    \item Earlier finders' rights trump later finders'.
    \item Applies to both real and personal property.
    \item Why should prior possessors prevail?
    \begin{enumerate}
        \item Effort in finding the property.
        \item Reliance on the property once acquired.
        \item Emotional attachment.
        \item Cultural habit (vs. other cultures, e.g. Japan, where turning in 
        found property is the norm).
        \item Predictability of the law. Finders should know their 
        obligations.
    \end{enumerate}
    % TODO question 3 p. 99
    % TODO questoin 5 pp. 100-101 
\end{enumerate}

\subsubsection{Relative Rights: \emph{Anderson v. Gouldberg}}

The plaintiff's property right must only be valid relative to the defendant's 
right. Property rights are relative.

\begin{enumerate}
    \item Plaintiffs harvested lumber while trespassing and took the logs to 
    the defendants' mill, where the defendants took them.
    \item The court found for the plaintiffs, holding that the plaintiff's 
    right to the property must only have been lawful against the defendant, 
    not all possible parties. The fact that the plaintiff may have originally 
    obtained the property illegally is irrelevant to its right to the property 
    relative to these defendants.
\end{enumerate}

\subsubsection{Landowner's Rights to Found Property: \emph{Hannah v. Peel}}

Landowners do not have rights to property that others found on their property 
if the landowner was not in physical possession of the premises---but ``the 
authorities are in an unsatisfactory state.''

\begin{enumerate}
    \item The house of defendant, Peel, was requisitioned, released, and 
    requisitioned again. Peel never occupied the house. Plaintiff Hannah, one 
    of the soldiers stationed there, found a brooch in a window frame. He 
    turned it over to the police. Two years later, the original owner had 
    still not showed up, so the police gave the brooch to Peel, who sold it 
    for 66\emph{l}.
    \item Hannah argued that he had a right to possession above all others 
    except the original owner's, and the original owner could not be located 
    (\emph{Armory}).
    \item The court reviewed conflicting precedent, concluding that ``the 
    authorities are in an unsatisfactory state.~.~.~'':\footnote{Casebook p. 
    106.}
    \begin{enumerate}
        \item \emph{Bridges v. Hawkesworth}: A shop customer found a package 
        of bank notes and turned it over to the shopkeeper. Nobody claimed the 
        package, and the customer sued to recover from the shopkeeper. The 
        court, following \emph{Armory}, found for the plaintiff.
        \item \emph{South Staffordshire Water Co. v. Sharman} A pool cleaner 
        found two rings in the mud. The court held that ``if something is 
        found on [an owner's] land, whether by an employee of the owner or by 
        a stranger, the presumption is that the possession of that thing is in 
        the owner of the locus in quo.''\footnote{Casebook p. 105.}
        \item \emph{Elwes v. Brigg Gas Co.}: A gas company with a 99 year 
        lease discovered an ancient boat buried in the soil. The court held 
        that the lessors would have had a right to the boat if it had been a 
        mineral (presumably because their purpose for using the land was to 
        draw minerals from it). But it held that the boat was a chattel and 
        that, since the original owner was long gone, the landowner had the 
        stronger right. The landowner's obliviousness was irrelevant: ``In my 
        opinion it makes no difference, in these circumstances, that the 
        plaintiff was not aware of the existence of the 
        boat.''\footnote{Casebook p. 105.}
    \end{enumerate}
    \item The court held that Peel was ``never in possession of the 
    premises,'' and therefore he did not have prior possession of the 
    brooch.''
    \item How to reconcile \emph{Hannah} with \emph{Anderson}?
        % TODO: ask: who are the competing claimants? who is the wrongdoer?
    \item Would Peel have won if Hannah had been a trespasser or renter?
        % TODO
\end{enumerate}

\subsection{\emph{Popov v. Hayashi}} 

Normally, if a possessor can prove by a preponderance of the evidence that he 
has a stronger claim than the defendant, he wins the entire property. But 
\emph{Popov} stands for the proposition that each possessor's interest in the 
property is proportional to the strength of their claims. When their claims are 
roughly equal, as they were here, equitable division is the right solution. 
Physical control is not always the bottom line.

% \begin{enumerate}
%     \item % TODO
%     \item % TODO who should have prevailed and why?
% \end{enumerate}

\subsection{Adverse Possession}

\begin{enumerate}
    \item ``Squatters' rights.'' Lay people often see it as ``a variety of 
    rip-off.''\footnote{Casebook p. 119.}
    \item A is the true owner. Adverse possession happens when B becomes the 
    possessor without A's consent. B might be the owner or have other rights 
    (e.g., easement).
    \item \textbf{Four elements of adverse possession}:
    \begin{enumerate}
        \item Actual entry giving exclusive possession (to trigger the statute 
        of limitations and to stake out the property being claimed).
        \item Open and notorious possession (to penalize the owner for 
        sleeping on his rights; the test of notoriety is objective).
        \item Possession that is adverse.
        \item Possession that is continuous for the statutory period (but not 
        literally constant---it need only match the average owner's use).
        \item (Sometimes: payment of property taxes.)
    \end{enumerate}
    \item \textbf{Color of title}: evidence (like a written deed) that appears 
    to establish title but actually does not (e.g., because the original 
    seller lacked a valid title, or because the seller was not competent to 
    transfer the property).
    \item \textbf{Tacking}: joining consecutive periods of possession by 
    multiple people as a single period of possession.\footnote{See 
    \emph{Howard v. Kunto} and reader p. 30 problem IV.}
    \begin{enumerate}
        \item Can be \emph{by} a series of adverse possessors or \emph{against} 
        a series of owners.
        \item Tacking usually requires \textbf{privity in estate}. There is no 
        privity when:
        \begin{enumerate}
            \item A2 forced A1 off the property.
            \item A1 abandoned his claim and A2 just happened onto the 
            property.
        \end{enumerate}
    \end{enumerate}
    \item \textbf{Ailing title}: say A enters adversely on O1's land. Five 
    years later, O1 transfers the property to O2. A can tack against O2---so, if 
    the statutory period for adverse possession is ten years, A only needs to 
    occupy the property for another five years. Since from O2's perspective the 
    statutory period appears to be only five years, O2 has acquired an ailing 
    title.
    \item \textbf{Disabilities}: some disabilities render property owners 
    incompetent to transfer ownership of their property, e.g., dementia.
    \begin{enumerate}
        \item A disability is immaterial unless it existed at the time when 
        the cause of action accrued (i.e., when the adverse possessor first.
        took possession of the land).
        \item Traditionally, the three Is: infancy, insanity, imprisonment.
    \end{enumerate}
    \item \textbf{Relation back}: the adverse possessor acquires a ``new title'' 
    that relates back to the time of the original entry. It comes up most often 
    when the owner tries to recover back rent payments from the adverse 
    possessor. Once the adverse possessor acquires a title, he is considered to 
    have been the owner from the time of his original entry---thus, the original 
    owner is not entitled to back payments.
    \begin{enumerate}
        \item Once the owner no longer has the power to eject the adverse 
        possessor, he also no longer has the power to recover the rental value.
        \item One exception is the Colorado legislation passed in reaction to 
        the Dick-and-Edie controversy.
    \end{enumerate}
    % TODO \item \textbf{Adverse possession of personal property}: 
    \item \textbf{Life estate}: an estate held only for the duration of a 
    person's life.
    \item \textbf{Fee simple absolute}: an estate of indefinite or potentially 
    infinite duration.
    \item Statutes of limitation vary between three and 30 years, but usually 
    run between six and 10.\footnote{Casebook p. 120 n. 4. Peterson mentioned 
    that it may be as high as 60 years for undeveloped land in New Jersey.}
    \item Adverse possession doctrine is a mix of judge-made law (e.g., the open 
    and notorious requirement) and statutory law (e.g., time period 
    requirements).
    \item \textbf{Adverse possession of personal property}:
    \begin{enumerate}
        \item It exists, and is in many was the same doctrine as adverse 
        possession for real property.
        \item The main difference is that it's easy to check on 
        land; not so much with personal property, even if the use is open and 
        notorious (e.g., works of art). Hence the \textbf{discovery rule}, which 
        provides that the statutory period for possession does not begin until 
        the owner knows (or should know) about the adverse possession.
    \end{enumerate}
    \item Synonyms: adversity, hostility, claim of right.
    \item \textbf{Standards for adversity}:
    \begin{enumerate}
        \item If you're there with permission, it's not adverse. (E.g., 
        pasture owners will put up ``permission to graze here'' signs.)
        \item \textbf{Objective test} (most courts):
        \begin{enumerate}
            \item Intent is irrelevant.
            \item Often, the occupant is required to \emph{appear} to be 
            claiming the land as her own.
            \item Prof. Helmholz study: even in jurisdictions that use the 
            objective test, courts usually find a way to make bad faith 
            possessors lose. There's a disconnect between stated law and 
            reality.
        \end{enumerate}
        \item \textbf{Subjective test} (some courts):
        \begin{enumerate}
            \item Good faith is required (you thought the land was yours); or
            \item Good faith is ok, but not required; or
            \item A \emph{lack} of good faith is required---a ``mentality of 
            thievery.'' This appears to be the majority's view in \emph{Lutz}.
        \end{enumerate}
    \end{enumerate}
    \item Questions:
    \begin{enumerate}
        \item What justifies the doctrine of adverse possession? What is the 
        ideal proposal? Is that proposal internally consistent?
        \begin{enumerate}
            \item See below, ``Possession and the Common Law Method.''
        \end{enumerate}
        \item Is it fair, always, that a person who makes better use of property 
        should be entitled to become its owner?
        \item If you ``steal slowly,'' does that make it ethical?
        \item Does notice to the owner matter? What are the owner's duties? 
        Does it matter whether the notice is real or constructive?
        \item When is B in possession of A's property?
        \item Why does the legal system recognize B's possession?
        \item Does it matter whether the owner had notice of B's possession?
        \item How long should the statute of limitations be? Five years, 100 
        years?
        \item What should be the standard for adversity? See competing 
        definitions above.
        \item Do we want land to be put to productive use?
        \begin{enumerate}
            \item Sprankling v. Posner.
        \end{enumerate}
        \item Is it worth having a doctrine of adverse possession at all? If 
        not, what would you tell a family who's been living in a house for 
        thirty years, but it turns out that the earlier owner (say, an 
        ancestor) did not have a valid title?
        \item If you use the good faith standard for adversity, should the 
        good faith be reasonable?
        \item Should adverse possessors have to pay fair market value to the 
        true owner?
        \item Should adverse possessors have to pay property taxes?
        \item How much land should you get in an adverse possession claim?
        \item Should adverse possessors be allowed to tack against multiple 
        owners?
        \begin{enumerate}
            \item Yes---mere ``paper delivery'' should not restart the statute 
            of limitations. The opposite would be unfair to the adverse 
            possessor.
        \end{enumerate}
        \item How should the disability doctrine work? Should there be an 
        extension for disabled owners? How long? What types of disability 
        qualify? What would happen if the law contained no disability 
        provisions? How should the law handle disabilities that occur after the 
        adverse possessor's entry? Should natural disasters count as 
        disabilities?
        \item Can a first adverse possessor sue to eject a second possessor, 
        even if the first adverse possessor does not yet have title?
        \begin{enumerate}
            \item Yes.
        \end{enumerate}
        \item What counts as continuous use?
        \begin{enumerate}
            \item The standard is the normal owner's behavior---e.g., if the 
            property is a summer beach house, the adverse possessor need only 
            occupy it during the summer.
            \item What could the owner do to interrupt occupation?
            \begin{enumerate}
                \item Eject by court action (the strongest).
                \item Ask the occupant to leave.
                \item Grant permission.
            \end{enumerate}
        \end{enumerate}
        \item Should color of title be part of adverse possession law? Under 
        what circumstances, if at all?
        \begin{enumerate}
            \item Color of title does not exist in England. In the US, it 
            developed in the context of the frontier. Does it still make sense 
            today?
        \end{enumerate}
        \item Is the complexity of color of title doctrine an argument against 
        its existence?
        \item Problems:
        \begin{enumerate}
            \item B acquires a new title that \textbf{relates back} to the date of 
            the event that started the statute of limitations running. For 
            instance, by the rule of increase, the owner of an animal owns that 
            animal's offspring.  If B takes possession of A's cow, and then the 
            cow has a calf, and then B  gets title to the cow by adverse 
            possession, B also owns the calf---even though the calf was born 
            before B had title to its mother.\footnote{Casebook p. 119 n. 2.}
            % TODO: how to reconcile adverse possession with the principle of 
            % first in time? p. 119 n. 1.
            \item Suppose you live in a house you inherited long ago. If somebody 
            way back in the family tree acquired the property by adverse 
            possession (inadvertently or not), should you be entitled to 
            possession? Where would you stand without the law of adverse 
            possession?
        \end{enumerate}
    \end{enumerate}
\end{enumerate}

\subsubsection{Powell on Real Property \S\ 91.01}

\begin{enumerate}
    \item Every American jurisdiction has a statute of limitations beyond 
    which the owner of land can no longer recover the land from another's 
    possession.
    \item Adverse possession ``rests upon social judgments that there should 
    be a restricted duration for the assertion of `aging claims,' and that the 
    passage of a reasonable time period should assure security to a person 
    claiming to be an owner.''\footnote{Casebook p. 117.}
\end{enumerate}

\subsubsection{Ballantine, \emph{Title by Adverse Possession}}

\begin{enumerate}
    \item ``~.~.~.~the doctrine apparently affords an anomalous instance of 
    maturing a wrong into a right~.~.~.~''\footnote{Casebook p. 117.}
    \item It's not about the merit of the possessor, but the demerit of the 
    one out of possession. We want people to use land in a useful 
    way.\footnote{Casebook p. 117.}
    \item The purpose of adverse possession is to ``automatically quiet all 
    titles which are openly and consistently asserted, to provide proof of 
    meritorious titles, and correct errors in 
    conveyancing.''\footnote{Casebook p. 117.}
\end{enumerate}

\subsubsection{Holmes, ``The Path of the Law''}

\begin{enumerate}
    \item Why have adverse possession? Loss of evidence and desirability of 
    peace are secondary. It mainly rests on the interests of the possessor, 
    not the interests of the absent owner.
    \item ``A thing which you have enjoyed and used as your own for a long 
    time, whether property or an opinion, takes root in your being and cannot 
    be torn away without your resenting the act and trying to defend yourself, 
    however you came by it. The law can ask no better justification than the 
    deepest instincts of man.''\footnote{Casebook p. 118.}
    \item What are Holmes's justifications for adverse 
    possession?\footnote{Casebook pp. 118--19 n. 1.}
    \begin{enumerate}
        \item \textbf{Economics}: Holmes was invoking diminishing marginal 
        utility of income.
        \item \textbf{Psychology}: Holmes was anticipating prospect theory, 
        which holds that people are more upset by losing something in hand 
        than by forgoing the opportunity to realize an equivalent gain.
        \item \textbf{Morality}: the new possessor depends on the property, 
        and it's wrong for the original owner to establish and then cut off 
        that dependence.
    \end{enumerate}
\end{enumerate}

\subsubsection{Sprankling, ``An Environmental Critique of Adverse Possession''}

\begin{enumerate}
    \item The ``development model'' of adverse possession, which holds that land 
    should be developed, rested on assumptions that might have been appropriate 
    in England, where the property in question would have been down the road. 
    But it is less appropriate in the U.S. where there are large tracts of 
    undeveloped land.\footnote{Reader p. 17.}
\end{enumerate}

\subsubsection{\emph{Van Valkenburgh v. Lutz}}

\begin{enumerate}
    \item Timeline:
    \begin{enumerate}
        \item 1912: the Lutzes bought lots 14 and 15 in Yonkers. Mr. Lutz 
        cleared a ``traveled way'' on lots 19--22.
        \item 1920: the Lutzes cleared their lot and built a house. They also 
        partially cleared lots 19--22 and built a small house for Mr. Lutz's 
        brother, Charlie.
        \item 1928: Mr. Lutz became a fulltime farmer and handyman.
        \item 1937: the Van Valkenburghs bought lots nearby.
        \item 1946: bad blood developed between the Lutzes and Van 
        Valkenburghs.
        \item April 14 1947: the Van Valkenburghs bought lots 19--22 and told the 
        Lutzes to clear their things and buildings.
        \item July 21, 1947: Mr. Lutz agreed to clear off the property but 
        claimed a prescriptive right (i.e., right to use---in this case, an
        easement). The Van Valkenburhgs built a fence across the easement. The 
        Lutzes sued.
        \item January 1948: the trial court ruled for Lutz, granting him a 
        right of way. Affirmed on appeal in June.
        \item April 1948: the Van Valkenburghs brought suit to eject Lutz for 
        failing to remove his belongings from lots 19--22. Lutz argued that he 
        had acquired title by adverse possession for upwards of thirty years 
        (the statutory period was 15 years). The trial and appellate courts 
        found for Lutz.
    \end{enumerate}
    \item Judge Dye (in the Court of Appeals):
    \begin{enumerate}
        \item Under the New York statute, adverse possession required occupation under a ``claim of 
        title,'' which required proof that the premises were (1) protected by a 
        substantial enclosure or (2) ``usually cultivated or 
        improved.''\footnote{Casebook p. 126 \S\ 40 in footnotes.} Thus the 
        key issue here was whether the land was usually cultivated or 
        improved.
        \item Lutz met neither of these requirements.
        \item In the previous action, Lutz conceded that the Van 
        Valkenburgh's legal title conferred actual ownership. He made this 
        concession in order to establish the basis for his easement claim. 
        He cannot now disavow that claim.
        \item Reversed.
        \item Peterson: the court here appears to require that occupant 
        \emph{intend} to occupy the land and take ownership. Under this 
        definition of adversity, the occupant should be openly hostile.
    \end{enumerate}
    \item Judge Fuld, dissenting:
    \begin{enumerate}
        \item There is ample evidence showing that Lutz occupied and 
        cultivated the land.
        \item The fact that Lutz knew that he did not have a title to the 
        property makes no difference as long as he intended to acquire and use 
        the property as his own.
        \item Lutz disclaimed his title \emph{after} the statute of 
        limitations period had run for adverse possession. Thus, he had a 
        valid title. A valid title cannot be transferred by oral disclaimer, 
        but only by ``the formalities prescribed by law.''\footnote{Casebook 
        p. 129.}
        \item The land need not be entirely cultivated. Lutz's partial 
        cultivation was enough.
    \end{enumerate}
    \item The court held Lutz's structure built for Charlie on 
    lots 19--22 did not count because he knew he wasn't building it on his own 
    land. It appears to argue, therefore, that you can only adversely possess land 
    that you think belongs to you.
    % TODO The casebook is stronger: ``you can only adversely possess land 
    % that is already yours. Is this accurate?
    \item \textbf{States of mind}:
    \begin{enumerate}
        \item \textbf{Objective standard} (state of mind is irrelevant): when 
        the new occupant enters the property, the owner has a cause of action. 
        Shouldn't the statute of limitations run at this 
        point?\footnote{Casebook p. 132.}
        \item \textbf{Good faith standard} (``I thought I owned it''): courts 
        often grant title to good faith trespassers but not to trespassers who 
        know the land does not belong to them.\footnote{Casebook pp. 132--133.}
        \item \textbf{Aggressive trespass standard} (``I knew I didn't own it, 
        but I intended to make it mine''): to qualify as adverse possessors, 
        occupants must intend to take the property for themselves. Courts 
        sometimes require the occupant to compensate the owner for the fair 
        market value of the property.
    \end{enumerate}
    \item \emph{Lutz} was overturned in \emph{Walling v. Przybylo}, which 
    held that there can be a claim of right even if the adverse possessor 
    knows that the land belongs to someone else.
    \item Later, the New York legislature passed a statute that defined the 
    claim of right requirement as ``a reasonable basis for the belief that the 
    property belongs to the adverse possessor or property owner, as the case 
    may be''---i.e., good faith is required.
\end{enumerate}

\subsubsection{Boundary Disputes}

\begin{enumerate}
    \item \textbf{Doctrine of agreed boundaries}: if the boundary is 
    uncertain, an oral agreement between neighbors is enforceable if they 
    accept the line for a long time.\footnote{Casebook pp. 140--41.}
    \item \textbf{Doctrine of acquiescence}: long acquiescence is evidence of 
    an agreement fixing the boundary line.\footnote{Casebook p. 141.}
    \item \textbf{Doctrine of estoppel}: the first neighbor makes 
    representations about the property line that the second neighbor relies 
    on. The first neighbor is estopped from denying the validity of his 
    earlier statements or acts.\footnote{Casebook p. 141.}
    \item \textbf{Innocent improver}: someone mistakenly builds on land 
    belonging to another.\footnote{Casebook p. 141.}
    \begin{enumerate}
        \item Common law: harsh. Everything goes to the owner of the land.
        \item Modern courts might force a conveyance of the land at market 
        value from the owner to the improver. The landowner may also have the 
        option to buy the improvement.
        \item Generally no relief if the encroachment is trivial.
        \item What if the encroachment takes up a significant part of the 
        land? In \emph{Amkco Ltd., Co v. Wellborn}, the court developed a 
        two-part test:
        \begin{enumerate}
            \item Plaintiff must show irreparable hardship if relief were 
            denied.
            \item Plaintiff's hardship is balanced against the defendant's 
            hardship.
        \end{enumerate}
    \end{enumerate}
    \item In cases of intentional encroachment, courts usually require removal 
    of the offending structure, no matter the cost.
    \item What if A builds a fence three feet across the line into B's 
    property and occupies the land for the statutory period?\footnote{Problem 
    1 on syllabus, 1/14/2013.}
    % TODO test against the different definitions of adversity
    \begin{enumerate}
        \item Can B successfully sue to eject A? No---the doctrine of 
        acquiescence means that A now has a valid claim of ownership over 
        those three feet of land.
        \item If A believed the fence was on the property line, should it 
        matter whether A intended to claim the land all the way up to the 
        fence, or only up to the actual property line?
        ~\\\\\\\\\\\\\\% TODO
        \item What if A wins title by adverse possession, but to avoid a 
        hassle, moves the fence back three feet? What if three years later A 
        changes her mind and wants to move the fence back?---If three years is 
        enough to satisfy the time period requirement for the doctrine of 
        agreed boundaries, A should lose.
        \begin{enumerate}
            \item Peterson: B could argue that A was making a representation, 
            but it would not be a strong argument. % TODO: why?
        \end{enumerate}
    \end{enumerate}
\end{enumerate}

\subsubsection{Color of Title and Constructive Adverse Possession}

\begin{enumerate}
    \item \textbf{Claim of title}: a means for the adverse possessor to 
    express the requirement of hostility or claim of right.\footnote{Casebook 
    p. 134.}
    \item \textbf{Color of title}: a claim founded on a defective or invalid 
    written instrument---e.g., the grantor doesn't own the land or is 
    incompetent to grant a transfer. Rarely, but occasionally, required for 
    adverse possession claims. Sometimes it confers a shorter statute of 
    limitations.\footnote{Casebook pp. 134--35.} It may also affect how much 
    land you can claim.
    \item Must the adverse possessor have a good faith belief in the validity 
    of the title?
    \begin{enumerate}
        \item Sometimes this question is sorted out in statutes, sometimes in 
        case law. Doubts about the validity are sometimes allowed.
    \end{enumerate}
    \item \textbf{Constructive adverse possession}: adverse possession of only 
    a portion of the land under color of title is \emph{constructive} 
    possession of the entire land. The advantage is that activities that 
    establish adverse possession extend to the boundaries specified in the 
    \begin{enumerate}
        \item Physical possession prevails over constructive adverse 
        possession.\footnote{See reader p. 29 problem III.}
        \item It matters how much of the property the adverse possessor 
        actually possesses. There is no bright line, but generally the adverse 
        possessor must occupy the land ``in reasonable proportion'' to the 
        total area. Courts often consider the circumstances of the possession 
        as well.
    \end{enumerate}
    deed.\footnote{Casebook p. 135.}
    \item O owns a 100-acre farm. A entered the back 40 acres under color of 
    an invalid deed from Z for the entire 100 acres. A occupied and improved 
    the land for the period required by the statute of 
    limitations.\footnote{Casebook p. 135 n. 1.}
    \begin{enumerate}
        \item A brings suit to evict O, arguing constructive adverse 
        possession. What result?---A should be able to adversely possess the 
        40 acres that she has occupied and improved, but not the other 
        60 acres, unless O has not occupied or improved his part of the land.
        \item What if O had originally acquired the farm by adverse possession 
        under color of title? Should the outcome be different?---No---O's 
        claim was valid once he satisfied the requirements for adverse 
        possession.
    \end{enumerate}
\end{enumerate}

\subsubsection{Interruption of Continuity}

\begin{enumerate}
    \item A and B occupy adjacent property. B built a fence twenty-five feet 
    in on his side. A occupied the 25-foot strip for fifteen years. In the 
    process of rebuilding his gas station, B tore down the fence, stored 
    building materials in the 25-foot strip, and then rebuilt the fence in the 
    same spot.\footnote{Problem 3 on syllabus, 1/14/2013.}
    \begin{enumerate}
        \item Should B's actions have interrupted the continuity of A's 
        adverse possession?
        \begin{enumerate}
            \item No. Many courts impose a high threshold for interruption of 
            occupancy. For instance, conducting a survey of the property lines 
            often isn't enough. The interruption should be open and notorious. 
            It also matters whether the adverse possessor knows he's on 
            someone else's land.
            \item But, the cases are inconsistent. Some courts might hold that 
            B's actions (storing materials on the strip) were ``an act of 
            dominion.''
        \end{enumerate}
        \item Should subjective intent or objective acts be determinative?
        ~\\\\\\\\\\\\\\% TODO
    \end{enumerate}
\end{enumerate}

\subsubsection{Color of Title, Tacking, and Seasonal Use: \emph{Howard v. 
Kunto}}

First, tacking is allowed for two adverse possessors in privity with each 
other. Second, seasonal use satisfies the requirements for adverse possession 
if the ordinary use of the property is seasonal.

\begin{enumerate}
    \item There were three adjacent fifty-foot-wide lots on the Hood Canal. 
    The Kuntos bought what they thought was the middle lot from McCall, which 
    contained the house that the Kuntos then inhabited. It turned out that the 
    Kuntos' deed actually described the lot directly 
    to the west.
    \item The Howards, wanting to subdivide their lot, undertook a survey. It 
    turned out that the Kuntos' house was on the Moyer's deeded property (in the 
    middle) and that the Moyers' house was on the Howards' deeded property (to 
    the east).
    \item The Howards and Moyers swapped ownership of their deeded land, 
    leaving the Howards in control of the land on which the Kuntos' house was 
    located.
    \item Up until this point, neither Moyer nor his predecessors took 
    possession of the land actually possessed by the Kuntos.
    \item Howard successfully sued to quiet title over Kunto. The trial court 
    denied Kunto's claim of adverse possession because (1) there was not 
    continuity of possession to permit tacking and (2) possession was not 
    continuous because it involved only a summer occupancy. Thus there were 
    two issues on appeal:
    \begin{enumerate}
        \item Is summer occupancy of a summer home continuous?
        \item Is tacking allowed when the claimant's predecessors also held 
        title to property A but mistakenly occupied property B?
    \end{enumerate}
    \item Holding:
    \begin{enumerate}
        \item Possession by the Kunto and predecessors exceeded the ten-year 
        statutory period for adverse possession.
        \item Summer occupancy is continuous because the standard is the 
        conduct of ordinary owners. In this case, ordinary owners would occupy 
        a summer house only during the summer.
        \item Generally, tacking is allowed if successive possessors are in 
        privity. The reason is that squatters should not be able to profit 
        from trespassing.\footnote{Casebook p. 146.} But in this case, because 
        all occupants acted in good faith on an erroneous deed, privity should 
        not be required. Tacking is allowed here.
        \item Kunto's title to the property where their house is located is 
        quieted. Reversed.
    \end{enumerate}
    \item What about Howard? Is he left without property?
\end{enumerate}

\subsubsection{Problems on Color of Title}

\begin{enumerate}
    \item Opal owned lots 1-4. It sold the lots to A, but the deed was invalid 
    because the company president had not signed off. A occupied and improved 
    lot 1 for the period prescribed by the statute of limitations. B entered 
    lot 2.\footnote{Problem \#1 in syllabus, 1/16/2013.}
    \begin{enumerate}
        \item What rights does A have against B and against Opal under N.Y. 
        Civ. Prac. Act \S\ 38 (cultivated/improved or substantially 
        enclosed)?\footnote{Casebook p. 125 in footnotes.}
        \begin{enumerate}
            \item Against B: A cannot remove B. A did not cultivate, inclose, 
            or improve lot 2, so it has no claim of title.
            \item Against Opal: A can maintain title on lot 1 but on none of 
            the others because A did not cultivate, inclose, or improve any 
            lots but lot 1.
        \end{enumerate}
        \item What rights does A have at common law?
        \begin{enumerate}
            \item Against B: the common law has no requirement for 
            cultivation, improvement, or inclosure. Since A held title to all 
            four lots and occupied them for the statutory period, A's claim of 
            title is valid.
            \item Against Opal: same.
        \end{enumerate}
        \item What if the property had not been divided into lots?
        \begin{enumerate}
            \item Constructive adverse possession would have given A a valid 
            claim to the entire property. The claim would have been valid 
            against both B and Opal.
        \end{enumerate}
    \end{enumerate}
    \item X and Y own lots 1 and 2, respectively. Neither is in possession. 
    The lots are conveyed by invalid deed from Z to A, who occupies lot 1 for 
    the statutory period. A sues X and Y to quiet title for lots 1 and 
    2.\footnote{Problem 2, p. 135.}
    \begin{enumerate}
        \item What result?
        \begin{enumerate}
            \item A wins against X because A met the requirements 
            for adverse possession on lot 1. But A loses against Y because A 
            has not met the statutory requirements for adverse possession of 
            lot 2.
        \end{enumerate}
        \item Would it matter if X had executed the deed?
        \begin{enumerate}
            \item No. A still did not adversely possess lot 2.
        \end{enumerate}
        \item Would it matter if X had executed the deed and A had entered lot 
        2?
        \begin{enumerate}
            \item Yes---in that case, A would have a valid claim of title 
            against Y for lot 2. But A would not have a valid claim of title 
            against X for lot 1 because the deed was invalid for lot 1 and A 
            did not adversely possess it.
        \end{enumerate}
    \end{enumerate}
\end{enumerate}

\subsubsection{Problems on Tacking}

\begin{enumerate}
    \item Assume a ten year statute of limitations.\footnote{Casebook p. 149 
    problems 2--3 at top. See reader p. 33.}
    \begin{enumerate}
        \item In 1994, A enters adversely on O's property. O dies in 1995, 
        leaving a will that gives the property to B for life, remainder to C. 
        B dies in 2010 without ever having entered onto the property. Who owns 
        the property?
        \begin{enumerate}
            \item A. A acquired an FSA in 1994.
        \end{enumerate}
        \item O dies in 1995, leaving a will that devises the property to B 
        for life, remained to C. A enters adversely in 1996. B dies in 2010. 
        Who owns the property?
        \begin{enumerate}
            \item C. A entered the property \emph{against B}, who held a life 
            estate. When B died, the title went to C. A must then adversely 
            possess against C for the entire ten year statutory period.
            \begin{enumerate}
                \item Principle: the adverse possessor acquires the title of the 
                owner at the time of the adverse possessor's original entry. 
                \textbf{The adverse possessor acquires the property interest she 
                enters against.}
                \item As an owner, if all you have is a life estate, you have 
                nothing to leave your heirs.
            \end{enumerate}
        \end{enumerate}
    \end{enumerate}
\end{enumerate}

\subsubsection{Problems on Disability}

\begin{enumerate}
    \item O's disability must have existed when the adverse possessor originally 
    took possession of the property.
    \item O owns the property in 1984. A enters adversely on May 1, 1984. The 
    age of majority is 18. The statutory period is 21 years, unless the owner 
    is disabled, in which case the AP must still reach the 21-year period and 
    can then gain ownership ten years after the removal of the 
    disability.\footnote{Casebook p. 149 nn. 1--3.}
    % TODO: why phrase the minimum statutory period as ``within 21 years'' if 
    % 21 years is the minimum?
    \begin{enumerate}
        \item O is insane in 1984. O dies insane and intestate in 2007.
        \begin{itemize}
            \item O's heir, H, is under no disability on 2007.
            \begin{enumerate}
                \item 2017. The adverse possession accrues at the point when 
                the owner is no longer disabled.
            \end{enumerate}
            \item (b) O's heir, H, is six years old in 2007.
            \begin{enumerate}
                \item 2017. O's disability was removed in 2007. A acquires 
                ownership ten years after the removal of the disability. 
                Therefore, A acquires ownership in 2017.
            \end{enumerate}
        \end{itemize}
        \item O has no disability in 1984. O dies intestate in 2002. O's heir, 
        H, is two years old in 2002. When does A acquire title by adverse 
        possession?
        \begin{enumerate}
            \item 2005. A's cause of action accrued in 1984. The statutory 
            period ends in 2005. O's death in 2002 does not reset the clock, 
            even though H is ``disabled.''
        \end{enumerate}
        \item O is five years old in 1984. In 1994 O becomes mentally ill and 
        dies intestate in 2009. O's heir, H, is under no disability. When does 
        A acquire title by adverse possession?
        \begin{enumerate}
            \item 2007. O's initial disability was removed in 1997. A must 
            wait ten years after the removal of the disability. Therefore, A 
            acquires ownership in 2007.
        \end{enumerate}
    \end{enumerate}
\end{enumerate}

\subsubsection{Adverse Possession against the Government}

\begin{enumerate}
    \item Common law: no adverse possession against the government.
    \item Some states have adopted changes to the common law 
    rules.\footnote{Casebook p. 150.}
    \item Should government land be treated differently for purposes of 
    adverse possession?
\end{enumerate}

\subsubsection{Problem on Relation Back}

\begin{enumerate}
    \item % TODO \footnote{See syllabus problem #1, 1/18/2013.}
\end{enumerate}

\subsubsection{Possession and the Common Law Method}

\begin{enumerate}
    \item Easy cases: (1) true owner's claim against a trespasser, (2) good 
    faith possessor's claim against a neglectful owner. But cases are often 
    more ambiguous---e.g., a bad-faith trespasser vs. a careless owner, or a 
    good faith possessor against a disabled owner. The common law accommodates 
    these cases poorly.
    \item The common law generally deals with these cases monolithically---all 
    or nothing. But there are alternatives:
    \begin{enumerate}
        \item Equitable division (cf. \emph{Popov v. Hayashi}).
        \item Equitable escheat: vesting title in the state.
        \item Land bank, where bad-faith adverse possessor's claims are put 
        and then given to good-faith adverse possessors.
        \item Require the adverse possessor to pay market value to the owner.
    \end{enumerate}
\end{enumerate}

\subsubsection{Perspectives on Private Property}

\begin{enumerate}
    \item \textbf{First occupancy theory}: all land was originally held in 
    common. The institution of private property was needed to keep the peace 
    among claimants. Law protects ``whatever an individual has managed to get 
    hold of~.~.~.~''\footnote{Reader p. 133.}
    \item \textbf{Labor-desert theory}: Locke: ``by occupying something one 
    mixes his labor with it.''\footnote{Reader p. 39.}
    \item \textbf{Personality theory}: Hegel: ``property is the 
    \emph{embodiment} of personality, my inward idea and will that something 
    is to be mine is not enough to make it my property; to secure this end 
    occupancy is requisite.''\footnote{Reader p. 41.}
    \item \textbf{Utilitarian theory}: Bentham: ``Property is nothing but a 
    basis of expectation, the expectation of deriving certain advantages from 
    a thing which we are said to possess, in consequence of the relation in 
    which we stand to it.''\footnote{Reader p. 41.} Property law broadens the 
    range of things we can enjoy.
    \item \textbf{Critical legal studies}: recurring themes:
    \begin{enumerate}
        \item Private property rests on coercive use of public authority.
        \item Legal principles---such as minimizing costs---can never be 
        politically neutral.
        \item Legal principles are indeterminate. In a legal dispute, only one 
        side can be based on legal principles. For instance, if someone 
        bequeaths property to another under certain conditions, the heir's 
        freedom of disposition of the property is constrained. Whose freedom 
        prevails? Both parties rely on the principle of freedom of 
        disposition. The outcome must therefore be based on something other 
        than legal principle, presumably politics.
        \item Collective ownership is common.
        \item Should we have rigid rules or flexible standards? These debates 
        mask deeper tensions, e.g., between social solidarity and guarded 
        isolation.
        \item Property rights should be conditional and provisional.
    \end{enumerate}
\end{enumerate}

