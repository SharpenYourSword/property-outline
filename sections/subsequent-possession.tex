\subsection{Acquisition by Find}

\begin{enumerate}
    \item \textbf{Bailment}: Voluntary or involuntary delivery of personal 
    property to another, e.g., dropping of laundry at the laundromat 
    (voluntary) or losing a jewel in a chimney to be later found by a chimney 
    sweep (involuntary---\emph{Armory}).
    \item \textbf{Trover}: ``forced purchase''\footnote{Casebook p. 
    99.}---action for recovery of damages for conversion of personal property, 
    generally measured by the property's value---e.g., the chimney sweep could 
    recover the value of the jewel, but not the jewel itself. \emph{Armory}.
    \item \textbf{Replevin}: action for recovery of personal property that the  
    defendant wrongfully took. \emph{Anderson}.
\end{enumerate}

\subsubsection{Prior Possessor Prevails: \emph{Armory v. Delamirie}}

The finder's right of ownership trumps all but the original owner's right (or 
earlier finders' rights). Relativity of title: TO \textgreater\ F1 
\textgreater\ F2. ``It's more mine than yours.''

\begin{enumerate}
    \item A chimney sweep found a jewel in a chimney. He brought it to a 
    goldsmith for appraisal, who offered a small amount of money, and when the 
    sweep refused, returned the socket without the jewel.
    \item The chimney sweep brought a trover action against the goldsmith. The 
    court found for the chimney sweep, holding that a finder has a right to 
    ownership above all but the property's rightful owner.
    \item \textbf{Winkfield Doctrine}: In cases of voluntary bailment, the 
    bailor (i.e., the original owner) does not have an action against the 
    present possessor if the bailee has recovered in full from the present 
    possessor.\footnote{Casebook p. 99.}
    \begin{enumerate}
        \item The justification is that it's not fair to the present 
        possessor to be doubly liable.
        \item Courts often apply the Winkfield Doctrine, but not always.
        \item Bailees have a duty to take care of the property. If a wrongdoer 
        takes the property and the bailee recovers in trover, it's an open 
        question whether the bailee should also have to take care of the money 
        in case the true owner (bailor) wants it back.
        \item Should Winkfield apply to involuntary bailment?
        \begin{enumerate}
            \item In voluntary bailment, the true owner consents to the 
            bailee's possession, and the bailee represents the owner's 
            interests in the property against wrongdoers. In involuntary 
            bailment, there's no such consent.
        \end{enumerate}
    \end{enumerate}
    \item Earlier finders' rights trump later finders'.
    \item Applies to both real and personal property.
    \item Why should prior possessors prevail?
    \begin{enumerate}
        \item Effort in finding the property.
        \item Reliance on the property once acquired.
        \item Emotional attachment.
        \item Cultural habit (vs. other cultures, e.g. Japan, where turning in 
        found property is the norm).
        \item Predictability of the law. Finders should know their 
        obligations.
    \end{enumerate}
    % TODO question 3 p. 99
    % TODO questoin 5 pp. 100-101 
\end{enumerate}

\subsubsection{Relative Rights: \emph{Anderson v. Gouldberg}}

The plaintiff's property right must only be valid relative to the defendant's 
right. Property rights are relative.

\begin{enumerate}
    \item Plaintiffs harvested lumber while trespassing and took the logs to 
    the defendants' mill, where the defendants took them.
    \item The court found for the plaintiffs, holding that the plaintiff's 
    right to the property must only have been lawful against the defendant, 
    not all possible parties. The fact that the plaintiff may have originally 
    obtained the property illegally is irrelevant to its right to the property 
    relative to these defendants.
\end{enumerate}

\subsubsection{Landowner's Rights to Found Property: \emph{Hannah v. Peel}}

Landowners do not have rights to property that others found on their property 
if the landowner was not in physical possession of the premises---but ``the 
authorities are in an unsatisfactory state.''

\begin{enumerate}
    \item The house of defendant, Peel, was requisitioned, released, and 
    requisitioned again. Peel never occupied the house. Plaintiff Hannah, one 
    of the soldiers stationed there, found a brooch in a window frame. He 
    turned it over to the police. Two years later, the original owner had 
    still not showed up, so the police gave the brooch to Peel, who sold it 
    for 66\emph{l}.
    \item Hannah argued that he had a right to possession above all others 
    except the original owner's, and the original owner could not be located 
    (\emph{Armory}).
    \item The court reviewed conflicting precedent, concluding that ``the 
    authorities are in an unsatisfactory state.~.~.~'':\footnote{Casebook p. 
    106.}
    \begin{enumerate}
        \item \emph{Bridges v. Hawkesworth}: A shop customer found a package 
        of bank notes and turned it over to the shopkeeper. Nobody claimed the 
        package, and the customer sued to recover from the shopkeeper. The 
        court, following \emph{Armory}, found for the plaintiff.
        \item \emph{South Staffordshire Water Co. v. Sharman} A pool cleaner 
        found two rings in the mud. The court held that ``if something is 
        found on [an owner's] land, whether by an employee of the owner or by 
        a stranger, the presumption is that the possession of that thing is in 
        the owner of the locus in quo.''\footnote{Casebook p. 105.}
        \item \emph{Elwes v. Brigg Gas Co.}: A gas company with a 99 year 
        lease discovered an ancient boat buried in the soil. The court held 
        that the lessors would have had a right to the boat if it had been a 
        mineral (presumably because their purpose for using the land was to 
        draw minerals from it). But it held that the boat was a chattel and 
        that, since the original owner was long gone, the landowner had the 
        stronger right. The landowner's obliviousness was irrelevant: ``In my 
        opinion it makes no difference, in these circumstances, that the 
        plaintiff was not aware of the existence of the 
        boat.''\footnote{Casebook p. 105.}
    \end{enumerate}
    \item The court held that Peel was ``never in possession of the 
    premises,'' and therefore he did not have prior possession of the 
    brooch.''
    \item How to reconcile \emph{Hannah} with \emph{Anderson}?
        % TODO: ask: who are the competing claimants? who is the wrongdoer?
    \item Would Peel have won if Hannah had been a trespasser or renter?
        % TODO
\end{enumerate}

% \subsection{\emph{Popov v. Hayashi}}
% 
% \begin{enumerate}
%     \item % TODO
%     \item % TODO who should have prevailed and why?
% \end{enumerate}

\subsection{Acquisition by Adverse Possession}

\begin{enumerate}
    \item ``Squatters' rights.'' Lay people often see it as ``a variety of 
    rip-off.''\footnote{Casebook p. 119.}
    \item A is the true owner. Adverse possession happens when B becomes the 
    possessor without A's consent. B might be the owner or have other rights 
    (e.g., easement).
    \item Adverse possession occurs when there is:
    \begin{enumerate}
        \item An \textbf{entry} (to trigger the statute of limitations and to 
        stake out the property being claimed)
        \item that is \textbf{open and notorious} (to penalize owners for 
        sleeping on their rights; the test of notoriety is objective),
        \item \textbf{continuous} for the statutory period (but not literally 
        constant---it only needs to match the average owner's use), and
        \item \textbf{adverse and under a claim of right} (~.~.~.~the 
        definition is contested).
    \end{enumerate}
    \item When is B in possession of A's property?
    \item Why does the legal system recognize B's possession?
    \item Does it matter whether the owner had notice of B's possession?
    \item B acquires a new title that \textbf{relates back} to the date of the 
    event that started the statute of limitations running. For instance, by 
    the rule of increase, the owner of an animal owns that animal's offspring. 
    If B takes possession of A's cow, and then the cow has a calf, and then B  
    gets title to the cow by adverse possession, B also owns the calf---even 
    though the calf was born before B had title to its 
    mother.\footnote{Casebook p. 119 n. 2.}
    % TODO: how to reconcile adverse possession with the principle of first in 
    % time? p. 119 n. 1.
    \item Suppose you live in a house you inherited long ago. If somebody way 
    back in the family tree acquired the property by adverse possession 
    (inadvertently or not), should you be entitled to possession? Where would 
    you stand without the law of adverse possession?
    \item Statutes of limitation vary between three and 30 years, but usually 
    run between six and 10.\footnote{Casebook p. 120 n. 4.}
    \end{enumerate}

\subsubsection{Powell on Real Property \S\ 91.01}

\begin{enumerate}
    \item Every American jurisdiction has a statute of limitations beyond 
    which the owner of land can no longer recover the land from another's 
    possession.
    \item Adverse possession ``rests upon social judgments that there should 
    be a restricted duration for the assertion of `aging claims,' and that the 
    passage of a reasonable time period should assure security to a person 
    claiming to be an owner.''\footnote{Casebook p. 117.}
\end{enumerate}

\subsubsection{Ballantine, \emph{Title by Adverse Possession}}

\begin{enumerate}
    \item ``~.~.~.~the doctrine apparently affords an anomalous instance of 
    maturing a wrong into a right~.~.~.~''\footnote{Casebook p. 117.}
    \item It's not about the merit of the possessor, but the demerit of the 
    one out of possession. We want people to use land in a useful 
    way.\footnote{Casebook p. 117.}
    \item The purpose of adverse possession is to ``automatically quiet all 
    titles which are openly and consistently asserted, to provide proof of 
    meritorious titles, and correct errors in 
    conveyancing.''\footnote{Casebook p. 117.}
\end{enumerate}

\subsubsection{Holmes, ``The Path of the Law''}

\begin{enumerate}
    \item Why have adverse possession? Loss of evidence and desirability of 
    peace are secondary. It mainly rests on the interests of the possessor, 
    not the interests of the absent owner.
    \item ``A thing which you have enjoyed and used as your own for a long 
    time, whether property or an opinion, takes root in your being and cannot 
    be torn away without your resenting the act and trying to defend yourself, 
    however you came by it. The law can ask no better justification than the 
    deepest instincts of man.''\footnote{Casebook p. 118.}
    \item What are Holmes's justifications for adverse 
    possession?\footnote{Casebook pp. 118--19 n. 1.}
    \begin{enumerate}
        \item \textbf{Economics}: Holmes was invoking diminishing marginal 
        utility of income.
        \item \textbf{Psychology}: Holmes was anticipating prospect theory, 
        which holds that people are more upset by losing something in hand 
        than by forgoing the opportunity to realize an equivalent gain.
        \item \textbf{Morality}: the new possessor depends on the property, 
        and it's wrong for the original owner to establish and then cut off 
        that dependence.
    \end{enumerate}
\end{enumerate}

\subsubsection{``An Environmental Critique of Adverse Possession''}

\begin{enumerate}
    \item % TODO in reader
\end{enumerate}

\subsubsection{\emph{Van Valkenburgh v. Lutz}}

\begin{enumerate}
    \item Timeline:
    \begin{enumerate}
        \item 1912: the Lutzes bought lots 14 and 15 in Yonkers. Mr. Lutz 
        cleared a ``traveled way'' on lots 19--22.
        \item 1920: the Lutzes cleared their lot and built a house. They also 
        partially cleared lots 19--22 and built a small house for Mr. Lutz's 
        brother, Charlie.
        \item 1928: Mr. Lutz became a fulltime farmer and handyman.
        \item 1937: the Van Valkenburghs bought lots nearby.
        \item 1946: bad blood developed between the Lutzes and Van 
        Valkenburghs.
        \item April 14 1947: the Van Valkenburghs bought lots 19--22 and told the 
        Lutzes to clear their things and buildings.
        \item July 21, 1947: Mr. Lutz agreed to clear off the property but 
        claimed a prescriptive right (i.e., right to use---in this case, his 
        easement). The Van Valkenburhgs built a fence across the easement. The 
        Lutzes sued.
        \item January 1948: the trial court ruled for Lutz, granting him a 
        right of way. Affirmed on appeal in June.
        \item April 1948: the Van Valkenburghs brought suit against Lutz. Lutz 
        argued that he had acquired title by adverse possession for upwards of 
        thirty years. The trial and appellate courts found for Lutz.
    \end{enumerate}
    \item Judge Dye (in the Court of Appeals):
    \begin{enumerate}
        \item Adverse possession requires occupation under a ``claim of 
        title,'' which requires proof that the premises are (1) protected by a 
        substantial enclosure or (2) usually cultivated or 
        improved.\footnote{Casebook p. 125.}
        \item Lutz met neither of these requirements.
        \item In the previous action, Lutz conceded that the Van 
        Valkenburgh's legal title conferred actual ownership. He made this 
        concession in order to establish the basis for his easement claim. 
        He cannot now disavow that claim.
        \item Reversed.
    \end{enumerate}
    \item Judge Fuld, dissenting:
    \begin{enumerate}
        \item There is ample evidence showing that Lutz occupied and 
        cultivated the land.
        \item The fact that Lutz knew that he did not have a title to the 
        property makes no difference as long as he intended to acquire and use 
        the property as his own.
        \item Lutz disclaimed his title \emph{after} the statute of 
        limitations period had run for adverse possession. Thus, he had a 
        valid title. A valid title cannot be transferred by oral disclaimer, 
        but only by ``the formalities prescribed by law.''\footnote{Casebook 
        p. 129.}
        \item The land need not be entirely cultivated. Lutz's partial 
        cultivation was enough.
    \end{enumerate}
    \item The court held Lutz's structure built for Charlie on 
    lots 19--22 did not count because he knew he wasn't building it on his own 
    land. It appears to argue, therefore, that you can only adversely possess land 
    that you think belongs to you.
    % TODO The casebook is stronger: ``you can only adversely possess land 
    % that is already yours. Is this accurate?
    \item \textbf{States of mind}:
    \begin{enumerate}
        \item \textbf{Objective standard} (state of mind is irrelevant): when 
        the new occupant enters the property, the owner has a cause of action. 
        Shouldn't the statute of limitations run at this 
        point?\footnote{Casebook p. 132.}
        \item \textbf{Good faith standard} (``I thought I owned it''): courts 
        often grant title to good faith trespassers but not to trespassers who 
        know the land does not belong to them.\footnote{Casebook pp. 132--133.}
        \item \textbf{Aggressive trespass standard} (``I knew I didn't own it, 
        but I intended to make it mine''): to qualify as adverse possessors, 
        occupants must intend to take the property for themselves. Courts 
        sometimes require the occupant to compensate the owner for the fair 
        market value of the property.
    \end{enumerate}
    \item \emph{Lutz} was overturned in \emph{Walling v. Przybylo}, which 
    held that there can be a claim of right even if the adverse possessor 
    knows that the land belongs to someone else.
    \item Later, the New York legislature passed a statute that defined the 
    claim of right requirement as ``a reasonable basis for the belief that the 
    property belongs to the adverse possessor or property owner, as the case 
    may be''---i.e., good faith is required.
\end{enumerate}

