\subsection{Finders}

\begin{enumerate}
    \item \textbf{Bailment}: Voluntary or involuntary delivery of personal 
    property to another, e.g., dropping of laundry at the laundromat 
    (voluntary) or losing a jewel in a chimney to be later found by a chimney 
    sweep (involuntary---\emph{Armory}).
    \item \textbf{Trover}: ``forced purchase''\footnote{Casebook p. 
    99.}---action for recovery of damages for conversion of personal property, 
    generally measured by the property's value---e.g., the chimney sweep could 
    recover the value of the jewel, but not the jewel itself. \emph{Armory}.
    \item \textbf{Replevin}: action for recovery of personal property that the  
    defendant wrongfully took. \emph{Anderson}.
    \item In finders cases, courts weight the relative strength of the 
    finder's claim to the property against his opponent's claim to that 
    property.
    \item \textbf{Relativity of title}: TO \textgreater\ F1 \textgreater\ F2. 
    ``It's more mine than yours.''
\end{enumerate}

\subsubsection{Prior Possessor Prevails: \emph{Armory v. Delamirie}}

The finder's right of ownership trumps all but the original owner's right (or 
earlier finders' rights).

\begin{enumerate}
    \item A chimney sweep found a jewel in a chimney. He brought it to a 
    goldsmith for appraisal, who offered a small amount of money, and when the 
    sweep refused, returned the socket without the jewel.
    \item The chimney sweep brought a trover action against the goldsmith. The 
    court found for the chimney sweep, holding that a finder has a right to 
    ownership above all but the property's rightful owner.
    \item \textbf{Winkfield Doctrine}: In cases of voluntary bailment, the 
    bailor (i.e., the original owner) does not have an action against the 
    present possessor if the bailee has recovered in full from the present 
    possessor.\footnote{Casebook p. 99.}
    \begin{enumerate}
        \item The justification is that it's not fair to the present 
        possessor to be doubly liable.
        \item Courts often apply the Winkfield Doctrine, but not always.
        \item Bailees have a duty to take care of the property. If a wrongdoer 
        takes the property and the bailee recovers in trover, it's an open 
        question whether the bailee should also have to take care of the money 
        in case the true owner (bailor) wants it back.
        \item Should Winkfield apply to involuntary bailment?
        \begin{enumerate}
            \item In voluntary bailment, the true owner consents to the 
            bailee's possession, and the bailee represents the owner's 
            interests in the property against wrongdoers. In involuntary 
            bailment, there's no such consent.
        \end{enumerate}
    \end{enumerate}
    \item Earlier finders' rights trump later finders'.
    \item Applies to both real and personal property.
    \item Why should prior possessors prevail?
    \begin{enumerate}
        \item Effort in finding the property.
        \item Reliance on the property once acquired.
        \item Emotional attachment.
        \item Cultural habit (vs. other cultures, e.g. Japan, where turning in 
        found property is the norm).
        \item Predictability of the law. Finders should know their 
        obligations.
    \end{enumerate}
    % TODO question 3 p. 99
    % TODO questoin 5 pp. 100-101 
\end{enumerate}

\subsubsection{Relative Rights: \emph{Anderson v. Gouldberg}}

The plaintiff's property right must only be valid relative to the defendant's 
right. Property rights are relative.

\begin{enumerate}
    \item Plaintiffs harvested lumber while trespassing and took the logs to 
    the defendants' mill, where the defendants took them.
    \item The court found for the plaintiffs, holding that the plaintiff's 
    right to the property must only have been lawful against the defendant, 
    not all possible parties. The fact that the plaintiff may have originally 
    obtained the property illegally is irrelevant to its right to the property 
    relative to these defendants.
\end{enumerate}

\subsubsection{Landowner's Rights to Found Property: \emph{Hannah v. Peel}}

Landowners do not have rights to property that others found on their property 
if the landowner was not in physical possession of the premises---but ``the 
authorities are in an unsatisfactory state.''

\begin{enumerate}
    \item The house of defendant, Peel, was requisitioned, released, and 
    requisitioned again. Peel never occupied the house. Plaintiff Hannah, one 
    of the soldiers stationed there, found a brooch in a window frame. He 
    turned it over to the police. Two years later, the original owner had 
    still not showed up, so the police gave the brooch to Peel, who sold it 
    for 66\emph{l}.
    \item Hannah argued that he had a right to possession above all others 
    except the original owner's, and the original owner could not be located 
    (\emph{Armory}).
    \item The court reviewed conflicting precedent, concluding that ``the 
    authorities are in an unsatisfactory state.~.~.~'':\footnote{Casebook p. 
    106.}
    \begin{enumerate}
        \item \emph{Bridges v. Hawkesworth}: A shop customer found a package 
        of bank notes and turned it over to the shopkeeper. Nobody claimed the 
        package, and the customer sued to recover from the shopkeeper. The 
        court, following \emph{Armory}, found for the plaintiff.
        \item \emph{South Staffordshire Water Co. v. Sharman} A pool cleaner 
        found two rings in the mud. The court held that ``if something is 
        found on [an owner's] land, whether by an employee of the owner or by 
        a stranger, the presumption is that the possession of that thing is in 
        the owner of the locus in quo.''\footnote{Casebook p. 105.}
        \item \emph{Elwes v. Brigg Gas Co.}: A gas company with a 99 year 
        lease discovered an ancient boat buried in the soil. The court held 
        that the lessors would have had a right to the boat if it had been a 
        mineral (presumably because their purpose for using the land was to 
        draw minerals from it). But it held that the boat was a chattel and 
        that, since the original owner was long gone, the landowner had the 
        stronger right. The landowner's obliviousness was irrelevant: ``In my 
        opinion it makes no difference, in these circumstances, that the 
        plaintiff was not aware of the existence of the 
        boat.''\footnote{Casebook p. 105.}
    \end{enumerate}
    \item The court held that Peel was ``never in possession of the 
    premises,'' and therefore he did not have prior possession of the 
    brooch.''
    \item How to reconcile \emph{Hannah} with \emph{Anderson}?
        % TODO: ask: who are the competing claimants? who is the wrongdoer?
    \item Would Peel have won if Hannah had been a trespasser or renter?
        % TODO
\end{enumerate}

% \subsection{\emph{Popov v. Hayashi}}
% 
% \begin{enumerate}
%     \item % TODO
%     \item % TODO who should have prevailed and why?
% \end{enumerate}

\subsection{Adverse Possession}

\begin{enumerate}
    \item ``Squatters' rights.'' Lay people often see it as ``a variety of 
    rip-off.''\footnote{Casebook p. 119.}
    \item A is the true owner. Adverse possession happens when B becomes the 
    possessor without A's consent. B might be the owner or have other rights 
    (e.g., easement).
    \item \textbf{Four elements of adverse possession}:
    \begin{enumerate}
        \item Actual entry giving exclusive possession (to trigger the statute 
        of limitations and to stake out the property being claimed).
        \item Open and notorious possession (to penalize the owner for 
        sleeping on his rights; the test of notoriety is objective).
        \item Possession that is adverse.
        \item Possession that is continuous for the statutory period (but not 
        literally constant---it need only match the average owner's use).
        \items (Sometimes: payment of property taxes.)
    \end{enumerate}
    \item \textbf{Color of title}: evidence (like a written deed) that appears 
    to establish title but actually does not (e.g., because the original 
    seller lacked a valid title, or because the seller was not competent to 
    transfer the property).
    \item \textbf{Tacking}: joining consecutive periods of possession by 
    multiple people as a single period of possession. % TODO: case or problem?
    \item \textbf{Disabilities}: some disabilities render property owners 
    incompetent to transfer ownership of their property, e.g., dementia.
    % TODO \item \textbf{Doctrine of relation back}: define, and see the 
    % problem below about the cow and the calf.
    % TODO \item \textbf{Adverse possession of personal property}: 
    \item Statutes of limitation vary between three and 30 years, but usually 
    run between six and 10.\footnote{Casebook p. 120 n. 4.}
    \item Synonyms: adversity, hostility, claim of right.
    \item \textbf{Standards for adversity}:
    \begin{enumerate}
        \item If you're there with permission, it's not adverse. (E.g., 
        pasture owners will put up ``permission to graze here'' signs.)
        \item \textbf{Objective test} (most courts):
        \begin{enumerate}
            \item Intent is irrelevant.
            \item Often, the occupant is required to \emph{appear} to be 
            claiming the land as her own.
        \end{enumerate}
        \item \textbf{Objective test} (some courts):
        \begin{enumerate}
            \item Good faith is required (you thought the land was yours); or
            \item Good faith is ok, but not required; or
            \item A \emph{lack} of good faith is required---a ``mentality of 
            thievery.'' This appears to be the majority's view in \emph{Lutz}.
        \end{enumerate}
    \end{enumerate}
    \item What counts as continuous use?
    \begin{enumerate}
        \item The standard is the normal owner's behavior---e.g., if the 
        property is a summer beach house, the adverse possessor need only 
        occupy it during the summer.
        \item What could the owner do to interrupt occupation?
        \begin{enumerate}
            \item Eject by court action (the strongest).
            \item Ask the occupant to leave.
            \item Grant permission.
        \end{enumerate}
    \end{enumerate}
    \item Key questions:
    \begin{enumerate}
        \item When is B in possession of A's property?
        \item Why does the legal system recognize B's possession?
        \item Does it matter whether the owner had notice of B's possession?
        \item What should be the standard for adversity? See competing 
        definitions above.
        \item Do we want land to be put to productive use?
        \begin{enumerate}
            \item Sprankling v. Posner.
        \end{enumerate}
        \item Is it worth having a doctrine of adverse possession at all? If 
        not, what would you tell a family who's been living in a house for 
        thirty years, but it turns out that the earlier owner (say, an 
        ancestor) did not have a valid title?
        \item If you use the good faith standard for adversity, should the 
        good faith be reasonable?
        \item Should adverse possessors have to pay fair market value to the 
        true owner?
        \item Should adverse possessors have to pay property taxes?
        \item How much land should you get in an adverse possession claim?
    \end{enumerate}
    \item Problems:
    \begin{enumerate}
        \item B acquires a new title that \textbf{relates back} to the date of 
        the event that started the statute of limitations running. For 
        instance, by the rule of increase, the owner of an animal owns that 
        animal's offspring.  If B takes possession of A's cow, and then the 
        cow has a calf, and then B  gets title to the cow by adverse 
        possession, B also owns the calf---even though the calf was born 
        before B had title to its mother.\footnote{Casebook p. 119 n. 2.}
        % TODO: how to reconcile adverse possession with the principle of 
        % first in time? p. 119 n. 1.
        \item Suppose you live in a house you inherited long ago. If somebody 
        way back in the family tree acquired the property by adverse 
        possession (inadvertently or not), should you be entitled to 
        possession? Where would you stand without the law of adverse 
        possession?
    \end{enumerate}
\end{enumerate}

\subsubsection{Powell on Real Property \S\ 91.01}

\begin{enumerate}
    \item Every American jurisdiction has a statute of limitations beyond 
    which the owner of land can no longer recover the land from another's 
    possession.
    \item Adverse possession ``rests upon social judgments that there should 
    be a restricted duration for the assertion of `aging claims,' and that the 
    passage of a reasonable time period should assure security to a person 
    claiming to be an owner.''\footnote{Casebook p. 117.}
\end{enumerate}

\subsubsection{Ballantine, \emph{Title by Adverse Possession}}

\begin{enumerate}
    \item ``~.~.~.~the doctrine apparently affords an anomalous instance of 
    maturing a wrong into a right~.~.~.~''\footnote{Casebook p. 117.}
    \item It's not about the merit of the possessor, but the demerit of the 
    one out of possession. We want people to use land in a useful 
    way.\footnote{Casebook p. 117.}
    \item The purpose of adverse possession is to ``automatically quiet all 
    titles which are openly and consistently asserted, to provide proof of 
    meritorious titles, and correct errors in 
    conveyancing.''\footnote{Casebook p. 117.}
\end{enumerate}

\subsubsection{Holmes, ``The Path of the Law''}

\begin{enumerate}
    \item Why have adverse possession? Loss of evidence and desirability of 
    peace are secondary. It mainly rests on the interests of the possessor, 
    not the interests of the absent owner.
    \item ``A thing which you have enjoyed and used as your own for a long 
    time, whether property or an opinion, takes root in your being and cannot 
    be torn away without your resenting the act and trying to defend yourself, 
    however you came by it. The law can ask no better justification than the 
    deepest instincts of man.''\footnote{Casebook p. 118.}
    \item What are Holmes's justifications for adverse 
    possession?\footnote{Casebook pp. 118--19 n. 1.}
    \begin{enumerate}
        \item \textbf{Economics}: Holmes was invoking diminishing marginal 
        utility of income.
        \item \textbf{Psychology}: Holmes was anticipating prospect theory, 
        which holds that people are more upset by losing something in hand 
        than by forgoing the opportunity to realize an equivalent gain.
        \item \textbf{Morality}: the new possessor depends on the property, 
        and it's wrong for the original owner to establish and then cut off 
        that dependence.
    \end{enumerate}
\end{enumerate}

\subsubsection{``An Environmental Critique of Adverse Possession''}

\begin{enumerate}
    \item % TODO in reader
\end{enumerate}

\subsubsection{\emph{Van Valkenburgh v. Lutz}}

\begin{enumerate}
    \item Timeline:
    \begin{enumerate}
        \item 1912: the Lutzes bought lots 14 and 15 in Yonkers. Mr. Lutz 
        cleared a ``traveled way'' on lots 19--22.
        \item 1920: the Lutzes cleared their lot and built a house. They also 
        partially cleared lots 19--22 and built a small house for Mr. Lutz's 
        brother, Charlie.
        \item 1928: Mr. Lutz became a fulltime farmer and handyman.
        \item 1937: the Van Valkenburghs bought lots nearby.
        \item 1946: bad blood developed between the Lutzes and Van 
        Valkenburghs.
        \item April 14 1947: the Van Valkenburghs bought lots 19--22 and told the 
        Lutzes to clear their things and buildings.
        \item July 21, 1947: Mr. Lutz agreed to clear off the property but 
        claimed a prescriptive right (i.e., right to use---in this case, his 
        easement). The Van Valkenburhgs built a fence across the easement. The 
        Lutzes sued.
        \item January 1948: the trial court ruled for Lutz, granting him a 
        right of way. Affirmed on appeal in June.
        \item April 1948: the Van Valkenburghs brought suit against Lutz. Lutz 
        argued that he had acquired title by adverse possession for upwards of 
        thirty years. The trial and appellate courts found for Lutz.
    \end{enumerate}
    \item Judge Dye (in the Court of Appeals):
    \begin{enumerate}
        \item Under the New York statute, adverse possession required occupation under a ``claim of 
        title,'' which required proof that the premises were (1) protected by a 
        substantial enclosure or (2) ``usually cultivated or 
        improved.''\footnote{Casebook p. 126 \S\ 40 in footnotes.} Thus the 
        key issue here was whether the land was usually cultivated or 
        improved.
        \item Lutz met neither of these requirements.
        \item In the previous action, Lutz conceded that the Van 
        Valkenburgh's legal title conferred actual ownership. He made this 
        concession in order to establish the basis for his easement claim. 
        He cannot now disavow that claim.
        \item Reversed.
        \item Peterson: the court here appears to require that occupant 
        \emph{intend} to occupy the land and take ownership. Under this 
        definition of adversity, the occupant should be openly hostile.
    \end{enumerate}
    \item Judge Fuld, dissenting:
    \begin{enumerate}
        \item There is ample evidence showing that Lutz occupied and 
        cultivated the land.
        \item The fact that Lutz knew that he did not have a title to the 
        property makes no difference as long as he intended to acquire and use 
        the property as his own.
        \item Lutz disclaimed his title \emph{after} the statute of 
        limitations period had run for adverse possession. Thus, he had a 
        valid title. A valid title cannot be transferred by oral disclaimer, 
        but only by ``the formalities prescribed by law.''\footnote{Casebook 
        p. 129.}
        \item The land need not be entirely cultivated. Lutz's partial 
        cultivation was enough.
    \end{enumerate}
    \item The court held Lutz's structure built for Charlie on 
    lots 19--22 did not count because he knew he wasn't building it on his own 
    land. It appears to argue, therefore, that you can only adversely possess land 
    that you think belongs to you.
    % TODO The casebook is stronger: ``you can only adversely possess land 
    % that is already yours. Is this accurate?
    \item \textbf{States of mind}:
    \begin{enumerate}
        \item \textbf{Objective standard} (state of mind is irrelevant): when 
        the new occupant enters the property, the owner has a cause of action. 
        Shouldn't the statute of limitations run at this 
        point?\footnote{Casebook p. 132.}
        \item \textbf{Good faith standard} (``I thought I owned it''): courts 
        often grant title to good faith trespassers but not to trespassers who 
        know the land does not belong to them.\footnote{Casebook pp. 132--133.}
        \item \textbf{Aggressive trespass standard} (``I knew I didn't own it, 
        but I intended to make it mine''): to qualify as adverse possessors, 
        occupants must intend to take the property for themselves. Courts 
        sometimes require the occupant to compensate the owner for the fair 
        market value of the property.
    \end{enumerate}
    \item \emph{Lutz} was overturned in \emph{Walling v. Przybylo}, which 
    held that there can be a claim of right even if the adverse possessor 
    knows that the land belongs to someone else.
    \item Later, the New York legislature passed a statute that defined the 
    claim of right requirement as ``a reasonable basis for the belief that the 
    property belongs to the adverse possessor or property owner, as the case 
    may be''---i.e., good faith is required.
\end{enumerate}

\subsubsection{Boundary Disputes}

\begin{enumerate}
    \item \textbf{Doctrine of agreed boundaries}: if the boundary is 
    uncertain, an oral agreement between neighbors is enforceable if they 
    accept the line for a long time.\footnote{Casebook pp. 140--41.}
    \item \textbf{Doctrine of acquiescence}: long acquiescence is evidence of 
    an agreement fixing the boundary line.\footnote{Casebook p. 141.}
    \item \textbf{Doctrine of estoppel}: the first neighbor makes 
    representations about the property line that the second neighbor relies 
    on. The first neighbor is estopped from denying the validity of his 
    earlier statements or acts.\footnote{Casebook p. 141.}
    \item \textbf{Innocent improver}: someone mistakenly builds on land 
    belonging to another.\footnote{Casebook p. 141.}
    \begin{enumerate}
        \item Common law: harsh. Everything goes to the owner of the land.
        \item Modern courts might force a conveyance of the land at market 
        value from the owner to the improver. The landowner may also have the 
        option to buy the improvement.
        \item Generally no relief if the encroachment is trivial.
        \item What if the encroachment takes up a significant part of the 
        land? In \emph{Amkco Ltd., Co v. Wellborn}, the court developed a 
        two-part test:
        \begin{enumerate}
            \item Plaintiff must show irreparable hardship if relief were 
            denied.
            \item Plaintiff's hardship is balanced against the defendant's 
            hardship.
        \end{enumerate}
    \end{enumerate}
    \item In cases of intentional encroachment, courts usually require removal 
    of the offending structure, no matter the cost.
    \item What if A builds a fence three feet across the line into B's 
    property and occupies the land for the statutory period?\footnote{Problem 
    1 on syllabus, 1/14/2013.}
    % TODO test against the different definitions of adversity
    \begin{enumerate}
        \item Can B successfully sue to eject A? No---the doctrine of 
        acquiescence means that A now has a valid claim of ownership over 
        those three feet of land.
        \item If A believed the fence was on the property line, should it 
        matter whether A intended to claim the land all the way up to the 
        fence, or only up to the actual property line?
        ~\\\\\\\\\\\\\\% TODO
        \item What if A wins title by adverse possession, but to avoid a 
        hassle, moves the fence back three feet? What if three years later A 
        changes her mind and wants to move the fence back?---If three years is 
        enough to satisfy the time period requirement for the doctrine of 
        agreed boundaries, A should lose.
        \begin{enumerate}
            \item Peterson: B could argue that A was making a representation, 
            but it would not be a strong argument. % TODO: why?
        \end{enumerate}
    \end{enumerate}
\end{enumerate}

\subsubsection{Color of Title and Constructive Adverse Possession}

\begin{enumerate}
    \item \textbf{Claim of title}: a means for the adverse possessor to 
    express the requirement of hostility or claim of right.\footnote{Casebook 
    p. 134.}
    \item \textbf{Color of title}: a claim founded on a defective or invalid 
    written instrument---e.g., the grantor doesn't own the land or is 
    incompetent to grant a transfer. Rarely, but occasionally, required for 
    adverse possession claims. Sometimes it confers a shorter statute of 
    limitations.\footnote{Casebook pp. 134--35.} It may also affect how much 
    land you can claim.
    \item \textbf{Constructive adverse possession}: adverse possession of only 
    a portion of the land under color of title is \emph{constructive} 
    possession of the entire land. The advantage is that activities that 
    establish adverse possession extend to the boundaries specified in the 
    deed.\footnote{Casebook p. 135.}
    \item O owns a 100-acre farm. A entered the back 40 acres under color of 
    an invalid deed from Z for the entire 100 acres. A occupied and improved 
    the land for the period required by the statute of 
    limitations.\footnote{Casebook p. 135 n. 1.}
    \begin{enumerate}
        \item A brings suit to evict O, arguing constructive adverse 
        possession. What result?---A should be able to adversely possess the 
        40 acres that she has occupied and improved, but not the other 
        60 acres, unless O has not occupied or improved his part of the land.
        \item What if O had originally acquired the farm by adverse possession 
        under color of title? Should the outcome be different?---No---O's 
        claim was valid once he satisfied the requirements for adverse 
        possession.
    \end{enumerate}
\end{enumerate}

\subsubsection{Interruption of Continuity}

\begin{enumerate}
    \item A and B occupy adjacent property. B built a fence twenty-five feet 
    in on his side. A occupied the 25-foot strip for fifteen years. In the 
    process of rebuilding his gas station, B tore down the fence, stored 
    building materials in the 25-foot strip, and then rebuilt the fence in the 
    same spot.\footnote{Problem 3 on syllabus, 1/14/2013.}
    \begin{enumerate}
        \item Should B's actions have interrupted the continuity of A's 
        adverse possession?
        \begin{enumerate}
            \item No. Many courts impose a high threshold for interruption of 
            occupancy. For instance, conducting a survey of the property lines 
            often isn't enough. The interruption should be open and notorious. 
            It also matters whether the adverse possessor knows he's on 
            someone else's land.
            \item But, the cases are inconsistent. Some courts might hold that 
            B's actions (storing materials on the strip) were ``an act of 
            dominion.''
        \end{enumerate}
        \item Should subjective intent or objective acts be determinative?
        ~\\\\\\\\\\\\\\% TODO
    \end{enumerate}
\end{enumerate}



